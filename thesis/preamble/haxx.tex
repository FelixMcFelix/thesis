% Hack command to insert tables.
\makeatletter\let\expandableinput\@@input\makeatother

% Wingdings for tables etc.
\newcommand{\cmark}{\ding{51}}%
\newcommand{\xmark}{\ding{55}}%

% \paragraph, without the paragraph.
\newcommand{\fakepara}[1]{\noindent\textbf{#1:}}

% Get Booktab style ruling in algorithm blocks.
% Values shamelessly taken from egreg's answer in 
% https://tex.stackexchange.com/a/345745/82917
\makeatletter
\renewcommand*{\@algocf@pre@ruled}{\hrule height\heavyrulewidth depth0pt \kern\belowrulesep}
\renewcommand*{\algocf@caption@ruled}{\box\algocf@capbox\kern\aboverulesep\hrule height\lightrulewidth\kern\belowrulesep}
\renewcommand*{\@algocf@post@ruled}{\kern\aboverulesep\hrule height\heavyrulewidth\relax}
\makeatother

% Needed to put emph, uline, bm into mathy places like an SI-format table
\robustify\bfseries
\robustify\emph
%\robustify\uline

% Nicer et al.
\DefineBibliographyStrings{english}{%
	andothers = {\emph{et al}\adddot}
}

% `Encourage' cref to be nicer.
\newcommand{\crefrangeconjunction}{--}
\crefname{table}{table}{tables}

% Terms used in the OPaL paper so that I could rename it down the line if I so chose.
\newcommand{\approach}{On Path Learning}
\newcommand{\approachshort}{OPaL}
\newcommand{\Coopfw}{\emph{CoOp}}
\newcommand{\coopfw}{\Coopfw}
\newcommand{\Indfw}{\emph{Ind}}
\newcommand{\indfw}{\Indfw}
\newcommand{\inring}{\textsc{In}}
\newcommand{\outring}{\textsc{Out}}

% Insight environment; pretty, but pretentious!
\newcounter{insightc}
\newenvironment{insight}
	{
		\begin{tipblock}\refstepcounter{insightc}\textbf{Insight \theinsightc:}\em
	}
	{
		\end{tipblock}
	}

% ToC lettering: change page numbers to be SF family.
\makeatletter
\let\oldl@chapter\l@chapter
\def\l@chapter#1#2{\oldl@chapter{#1}{\sffamily{\itshape#2}}}

\let\old@dottedcontentsline\@dottedtocline
\def\@dottedtocline#1#2#3#4#5{%
	\old@dottedcontentsline{#1}{#2}{#3}{#4}{{\sffamily{\itshape#5}}}}
\makeatother

% Figure label formatting
\captionsetup[figure]{labelfont=bf}
\captionsetup[subfigure]{labelfont={bf,it},textfont={it}}

% \sidenotetext
% Used in \sidenote
% Changed to meet the question of user J. Bratt (https://tex.stackexchange.com/questions/361622).
\makeatletter
\RenewDocumentCommand\sidenotetext{ o o +m }{%      
	\IfNoValueOrEmptyTF{#1}{%
		\@sidenotes@placemarginal{#2}{\textsuperscript{\thesidenote}{}~\footnotesize#3}%
		\refstepcounter{sidenote}%
	}{%
		\@sidenotes@placemarginal{#2}{\textsuperscript{#1}~#3}%
	}%
}
\makeatother

% Consistent acronym expansion
\renewcommand*{\glsfirstlongdefaultfont}[1]{\emph{#1}}

%
% Below here are some default additions from Stephen Strowes's template.
%
\hyphenation{Internet analysis analyse}

\newcommand\TLSins[1]{
  \cbstart{}
  {
    \color{blue}
    \ul{#1}
  }
  \cbend{}
}

\newcommand\TLSdel[1]{
  \cbdelete{}
  {
    \color{red}
    \st{#1}
  }
}

\newenvironment{codelisting}
{\begin{list}{}{\setlength{\leftmargin}{1em}}\item\scriptsize\bfseries}
{\end{list}}

\newcommand{\note}[1]{\emph{\textbf{Note:}[#1]}}
