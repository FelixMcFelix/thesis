% \usepackage[no-math]{fontspec}

% \renewcommand{\rmdefault}{minlibertine}
% \usepackage[libertine,upint]{newtxmath}
% \usepackage[scr=rsfso]{mathalfa}% helps with loading of math alphabets

%\usepackage{Alegreya}
\usepackage[oldstyle]{AlegreyaSans}
%\usepackage[sfdefault]{AlegreyaSans}

%\usepackage{tgpagella}
% \usepackage[oldstyle]{libertine}
% \usepackage[oldstyle]{libertine}
\usepackage[oldstyle]{libertinus}
% \usepackage{libertinust1math}
%\usepackage{newpxtext}
%\usepackage{newpxmath}
%\usepackage{FiraSans}

\setmathfont{Libertinus Math}

%\usepackage[
%    math-style=ISO,
%    bold-style=ISO,
%    partial=upright,
%    nabla=upright
%]{unicode-math}

% \usepackage{unicode-math}
% \setmathfont[Scale=MatchUppercase]{libertinusmath-regular.otf}

% \setmainfont{Libertinus Serif}
% \setsansfont{Libertinus Sans}
% \setmathfont{Libertinus Math}

%\DeclareSymbolFont{liningdigits}{\encodingdefault}{LinuxLibertineT-TLF}{m}{n}
%\SetSymbolFont{liningdigits}{bold}{\encodingdefault}{LinuxLibertineT-TLF}{b}{n}
%\DeclareMathSymbol{0}{\mathalpha}{liningdigits}{`0}
%\DeclareMathSymbol{1}{\mathalpha}{liningdigits}{`1}
%\DeclareMathSymbol{2}{\mathalpha}{liningdigits}{`2}
%\DeclareMathSymbol{3}{\mathalpha}{liningdigits}{`3}
%\DeclareMathSymbol{4}{\mathalpha}{liningdigits}{`4}
%\DeclareMathSymbol{5}{\mathalpha}{liningdigits}{`5}
%\DeclareMathSymbol{6}{\mathalpha}{liningdigits}{`6}
%\DeclareMathSymbol{7}{\mathalpha}{liningdigits}{`7}
%\DeclareMathSymbol{8}{\mathalpha}{liningdigits}{`8}
%\DeclareMathSymbol{9}{\mathalpha}{liningdigits}{`0}

% \AtBeginDocument{%
%   \Umathcode`0="7 "0 `0
%   \Umathcode`1="7 "0 `1
%   \Umathcode`2="7 "0 `2
%   \Umathcode`3="7 "0 `3
%   \Umathcode`4="7 "0 `4
%   \Umathcode`5="7 "0 `5
%   \Umathcode`6="7 "0 `6
%   \Umathcode`7="7 "0 `7
%   \Umathcode`8="7 "0 `8
%   \Umathcode`9="7 "0 `9
% }

% \SetTracking{encoding={*}, shape=sc}{40}

% must be loaded after ams math...
\usepackage[nameinlink]{cleveref}
