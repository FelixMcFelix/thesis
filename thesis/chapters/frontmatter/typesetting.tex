\chapter*{Notes on Typesetting}
\addcontentsline{toc}{chapter}{Notes on Typesetting}
%\def\theHchapter{3}

Body text is set in \emph{Libertinus Serif}, a free, libre (and more visually pleasing) Times New Roman alternative, widely used in ACM publications.
Headings, the table of contents, and some text elements (such as complexity classes) are set in sans-serif using the free, humanist \emph{Alegreya Sans} typeface.

This thesis is typeset, first and foremost, around readability.
As such, the main body width is set to \qty{11.75}{\centi\metre}, which provides a comfortable \numrange{60}{65} characters per line.
The remaining page width up to the University specification (\qty{15.5}{\centi\metre}) is used to provide lengthier asides in the margins\sidenote{Much like this one. This allows for more interesting comments and/or explanations without disrupting the main flow of the document.}, which I find considerably less disruptive than footnotes.
Line-spacing is set to \qty{136.8}{\percent} to compensate, which achieves similar words per page to the standard (\numrange{490}{520}, excluding sidenotes).

Some colour coding is used for links and markers of various classes:
\begin{itemize}
	\item {\color{kthesis-cite}Citations} to other referenced works,
	\item {\color{kthesis-internal}Section and Acronym} references, which are all functioning links,
	\item {\color{kthesis-url}URLs}, particularly in the bibliography,
	\item and {\color{kthesis-glance}Markers} indicating key elements of other works which benefit from glance value (e.g., \rllitstate, \rllitact).
\end{itemize}