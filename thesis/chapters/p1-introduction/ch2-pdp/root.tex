\chapter{Programmable Computer Networks}\label{sec:ch-networks}\label{chap:nets}
\section{Computer Networks}

?? What are networks?

?? Possibly discuss the internet

?? Lead in from ARPAnet et al. (2 paras). Scientific comms -> general use,

\section{Operation and Management}

?? network-network comms? BGP

?? IGP?

?? packets routed on a per-hop basis from their perspective: may be higher level in practice (MPLS, path switching within a gateway)

?? How can we examine this? High-level (above), mid-level ()

?? Two axes: end-to-end protocol and fabric behaviour. interact in a very delicate way (i.e., host )

?? named-data networking as potential structure of the Internet?~\parencite{DBLP:journals/ccr/0001ABJcCPWZ14}
?? Can I use this to suggest/outline problems which might be solved/encountered in a future Internet?

?? Talk about \gls{acr:as} families here: \glspl{acr:isp}, hypergiants~\parencite{DBLP:conf/sigcomm/GigisCMNKDKS21}...
?? Data centres: refer to e.g. google Espresso [sigcomm 2017] as big SDN deployments

\subsection{Fixed-Function Hardware}

\subsection{Software-Defined Networking}

?? Run through the historical context. Why? What led into P4 (OpenFlow, network operating systems...)

?? \gls{acr:ovs}~\parencite{DBLP:conf/nsdi/PfaffPKJZRGWSSA15} huge here.

?? A survey to mine for stuff~\parencite{DBLP:journals/comsur/NunesMNOT14}

?? Tie into PDP here: active networking and TPP . Think about the concept of protocol boosters~\parencite{DBLP:journals/jsac/FeldmeierMSBMR98}. (NOT READ)

\section{Traffic Characteristics}

?? Can (and should probably) discuss different traffic classes here: congestion-aware, -unaware...

?? Note, explain that this is NOT just TCP vs UDP due to existence of SCTP over UDP (See: DTLS in WebRTC), QUIC over UDP, ...

?? Chain this into \glspl{acr:cca}

?? Explain why you need what.

?? Historical context for their inclusion?

?? Discussion of evolution of traffic: what's come before, what's coming next.
?? Look for older in my old notes, but recent cite here~\parencite{DBLP:conf/anrw/BauerJHBC21}.

?? Describe all my CAIDA analysis here
?? analysis of CAIDA datasets~\parencite{caida-2018-passive}
?? congestion-aware traffic makes up at least \qtyrange{73}{82}{\percent} of packets\sidenote{\url{https://github.com/FelixMcFelix/caida-stats}}
?? Also talk about QUIC's prevalence here

\subsection{Evolution}

\subsection{Emerging Protocols}

?? QUIC~\parencite{DBLP:conf/sigcomm/LangleyRWVKZYKS17}

?? QUIC carries \gls{acr:http} traffic, mostly...

\subsection{Limitations}

\section{Problems in Modern Networks}\label{sec:problems-in-modern-networks}

\subsection{Attacks on the Internet}

?? DDoS -> split by type.

?? Other attacks?

\subsection{Scaling}

\subsection{Fairness}

\section{Summary}
Eh.


\section{Programmable Data-Planes}

?? 2021 --- crucial to sell the why! What can be gained moving to this level, new state measures enable new applications, in-nic/switch exec reduces latencies, increases throughputs, etc...
?? Probably want a worked example showing how in-network compute helps. I.e., network graph, show processing at nodes

?? Explain \gls{acr:npu} here...

?? Understanding Host Network Stack Overheads~\parencite{DBLP:conf/sigcomm/CaiCVH021}

?? Evidence of deployment in real, huge transit networks~\sidenote{\url{https://wiki.geant.org/display/RARE/Home}, \url{https://wiki.geant.org/display/NETDEV}}

?? arbitrarily reconfigurable hardware located directly on the data path

?? mention how it came to this: this was an alternate solution to latency or throughput concerns which plague VNF approaches as they scale (see Metron paper for good discussion of RSS, etc., which solve these problems in their own way)

?? new ways to implement stateful + stateless packet processing.

?? History? RMT~\parencite{DBLP:conf/sigcomm/BosshartGKVMIMH13}, ClickNP, GPU offload papers. See Metron paper for cites.

?? NetFPGA~\parencite{DBLP:conf/fpga/IbanezBMZ19}

?? Getting even faster?~\parencite{nokia-fp5}

\sidenote{Test text hello.}

?? RMT was the first big innovation over OpenFlow --- read this!!

\section{Control and Management}

?? Refer back to History through OpenFlow -- what prompted the evolution?

?? Control of these devices has a lot in common with OpenFlow -- controller, except using commodity hardware to install firmwares, and so on,

\section{Software Frameworks}

\subsection{eBPF}
?? eBPF

?? BPFabric

\subsection{P4}
?? P4

?? Others who lost?

?? How do these differ? What do they share?

?? Popular frameworks now support this -- ONOS, eBPF translators, behavioural model software switch...

?? Things like Lucid built on top of P4?~\parencite{DBLP:conf/sigcomm/SonchackLRW21}

?? P4~\parencite{DBLP:journals/ccr/BosshartDGIMRSTVVW14} and \emph{programmable dataplane} (PDP) hardware~\parencite{DBLP:journals/micro/ZilbermanACM14, netronome-smartnic, xilinx-alveo, barefoot-intel}

?? Cool NIC-CPU co-design~\parencite{DBLP:conf/osdi/IbanezMAJ0KM21}

\fakepara{PDP design for asynchronous compute}
\emph{PANIC}~\parencite{DBLP:conf/hotnets/StephensAS18} places a routing fabric between distinct packet/data processing elements \emph{in a SmartNIC}.
Such designs would enable general, asynchronous, and novel compute in SmartNICs and switches, for instance offering consistent and easy to use communication between workers versus hard-coded ME relationships.
Event-driven versions of P4 have been suggested~\parencite{DBLP:conf/hotnets/IbanezABM19}.
Timer events and device state changes would empower in-network RL use-cases, signalling timesteps for RL agents or new, effective, fine-grained sources of input state.

\section{Use Cases}

?? On-switch (Tofino) DDoS detection and defence \url{https://www.usenix.org/conference/usenixsecurity21/presentation/liu-zaoxing}

\subsection{In-Network Computation}\label{sec:in-network-computation}
?? In-network computation.\sidenote{Test text hello.}

iSwitch~\cite{DBLP:conf/isca/LiLYCSH19} uses programmable switches to combine model updates between RL agents acting as part of a distributed RL training system.
Note that we discuss this in section \ref{sec:wtf}.\sidenote{Make sure that this is properly explained in the PDP seciton: source is `report-yr2.tex'.}

?? Taurus moved out of representations section~\parencite{DBLP:journals/corr/abs-2002-08987}

?? A recent line of research in the community has been to investigate \emph{Binarised/Bitwise Neural Networks} (BNNs)~\parencite{DBLP:conf/nips/HubaraCSEB16,DBLP:journals/corr/KimS16,DBLP:journals/corr/MiyashitaLM16} for line-rate packet classification.
?? \emph{BaNaNa SPLIT} shows this as an offload mechanism~\parencite{DBLP:conf/sigcomm/SanvitoSB18,DBLP:journals/corr/abs-1801-05731}; DNN inference is often carried out on the \emph{CPU} to reduce latency imposed by GPU batching and transfer, but fully-connected layers can be accelerated further by NICs.

\subsection{Network Telemetry}
?? Others (try to avoid DDN cases here).

?? The P4 ecosystem already presents novel, openly-available, fine-grained traffic measurement techniques that can be installed and controlled with ease~\parencite{DBLP:conf/sigcomm/GuptaHCFRW18,DBLP:conf/sigcomm/ChenFKRR18,DBLP:conf/sosr/GhasemiBR17}

\subsection{Transport Layer Optimisation}
?? Hi~\parencite{DBLP:conf/nsdi/ArashlooLGRWW20} Network stacks being moved into NICs to reduce latency/CPU utilisation, mainly for datacentre use-cases---otherwise, \SI{100}{\giga\bit\per\second} can't be hit. New API \emph{Tonic} for transport layer in user code (send, data management), DMA to NIC and let it handle all (de)packetisation. ``Transport logic'' goes to Tonic. Main design is datacentres, so not very high BDPs (long-fat) $\rightarrow$ \si{\kilo\byte} inflight data.

\subsection{Active Queue Management}
Turns out that you can't just write it in P4, you need to co-design for the target environment---with meaningful performance cost to boot, based on the tradeoffs you need to make~\parencite{Kunze-P4-AQM}.

\subsection{KV Stores}
NetChain~\parencite{DBLP:conf/nsdi/JinLZFLSKS18}.

\subsection{ML in PDP}
?? ML in the dataplane~\parencite{DBLP:conf/hotnets/XiongZ19,DBLP:conf/sigcomm/SanvitoSB18,DBLP:journals/corr/abs-1801-05731,DBLP:journals/corr/abs-2009-02353,langlet-ml-netronome,DBLP:journals/corr/abs-2002-08987}

\section{Hardware Designs}

?? Mention FPGA vs many-core

?? P4 etc. can actually all run on commodity hardware, which offers a third (suboptimal) hardware class.

?? Ref the paper that Haruna presented \parencite{DBLP:conf/icc/MafiolettikDMRV20}: pareto front of work-division optimality for SmartNICs (i.e., addition of high-latency cores).

?? How do these limit and influence what code can be run on different device classes? The time taken to adapt the network?

\subsection{Models of Parallelism}

\subsection{Flexibility}

\subsection{Mapping Software Frameworks}

\section{Summary}
Eh. \sidenote{Test text hello. Test text hello. Test text hello. Test text hello. Test text hello. Test text hello. Test text hello. Test text hello. Test text hello.}
