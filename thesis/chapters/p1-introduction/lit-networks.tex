% TODO: split file by chapter.

\chapter{Computer Networks}

?? What are networks?

?? Possibly discuss the internet

\section{Operation}

?? network-network comms? BGP

?? IGP?

?? packets routed on a per-hop basis from their perspective: may be higher level in practice (MPLS, path switching within a gateway)

?? How can we examine this? High-level (above), mid-level ()

?? Two axes: end-to-end protocol and fabric behaviour. interact in a very delicate way (i.e., host )

\section{Traffic Characteristics}

?? Can (and should probably) discuss different traffic classes here: congestion-aware, -unaware...

?? Historical context for their inclusion?

?? Discussion of evolution of traffic: what's come before, what's coming next.

\section{Software-Defined Networking}

?? Run through the historical context. Why? What led into P4 (OpenFlow, network operating systems...)

\section{Attacks on the Internet}

?? DDoS -> split by type.

?? Other attacks?

\section{Summary}

% -------------------------

\chapter{Data-driven Networking}

?? Firstly, what is data-driven networking? Probably: making use of measurements, observations, and statistics (run-time, simulation, pen-and-paper) to enhance and improve the operation of a computer network.

?? enhance in what way? what metrics?

\section{Use Cases}

?? Optimisation

?? Design

?? Detection / telemetry / inference

?? Refer back to the computer networks chapter: topologies, routing, defence, ... all present problems who are often served and solved by the use of heuristics.

\section{Function approximation}

?? Explain how different approximators work, I suppose?

?? Linear coding

?? Neural networks

\section{Learning an approximation}

?? How do we learn a nice policy?

\subsection{Gradient Descent}

?? The standard. Probably space here to reference a good many techniques.

\subsection{Federated Learning}

?? Describe it here

?? Issue? Only works on certain problems (explicitly unsupervised, or easy to acquire local supervised measurements).

\subsection{Reinforcement Learning}

?? RL works in tandem with other mathematical training approaches: the key insight is that the structure of an MDP allows external information and model(-free) observation to strengthen or weaken different function responses.

\section{Hardware considerations}

?? Put in all the stuff about quantisation here?

\section{Security}

?? Discuss attacks on ML models, techniques, paradigms here.

\subsection{Attacks on Data-Driven Techniques}

?? White-box attacks -- evasion, adversarial examples, etc.

\subsection{Defences on above}

?? DIstillation, other papers I read ages ago.

?? Mine the relevant conferences for more...

\section{Summary}

% -------------------------

\chapter{Programmable Data-Planes}

?? arbitrarily reconfigurable hardware located directly on the data path

?? mention 

\section{Control and Management}

?? Refer back to History through OpenFlow -- what prompted the evolution?

?? Control of these devices has a lot in common with OpenFlow -- controller, except using commodity hardware to install firmwares, and so on,

\section{Frameworks}

?? eBPF

?? BPFabric

?? P4

?? Others who lost?

?? How do these differ? What do they share?

\section{Use Cases}

?? In-network computation.

?? Others (try to avoid DDN cases here).

\section{Hardware designs}

?? Mention FPGA vs many-core

?? P4 etc. can actually all run on commodity hardware, which offers a third (suboptimal) hardware class.

?? Ref the paper that Haruna presented: pareto front of work-division optimality for SmartNICs (i.e., addition of high-latency cores).

?? How do these limit and influence what code can be run on different device classes? The time taken to adapt the network?

\section{Summary}

% -------------------------
