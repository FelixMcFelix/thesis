\gls{acr:ml} models, introduce particular security issues in their training, use, and how we expose them or their decisions to users.
Recall that, in general, their operation is \emph{entirely governed by their parameter set $\wvec{}$}, and that we currently face great difficulty in understanding exactly what transformations or logic they encode.
What additional concerns might arise from this?
The most obvious challenge is that an attacker might construct input samples which appear to a human to have one label, but produce a strong response in a \gls{acr:ddn} model for \emph{another label}.
We call such inputs \emph{adversarial examples} or \emph{evasion attacks}.
Changing focus, the idea of online or \emph{active learning}~\parencite{active-learning-report} can seem like a powerful capability to have in the administration of a network for saving operator time.
In introducing this, we now need to ask how an attacker might aim to covertly modify our model's behaviour, either to change the label for a handful of samples (e.g., ensure a malware sample always evades inspection), adjust the entire decision surface (e.g., to incur a \gls{acr:dos} or performance degradation by incorrectly handling \emph{all} flows), or to encode an input pattern which always triggers a given output.
These behaviours fall under the umbrella of \emph{poisoning attacks}.
In tandem, we must also ask whether attackers are able to reverse engineer our model parameters from queries or environmental observation, and the privacy implications of a parameter vector $\wvec{}$ being leaked or extracted---\emph{data extraction} attacks.
These classes of attack, interestingly, mostly mirror those that have historically threatened classifiers in the security domain~\parencite{DBLP:conf/ccs/BarrenoNSJT06}.

%?? valid concerns in is the model behaving as intended? Can an attacker change our model's behaviour? Can sensitive data be extracted from an ML model?

I introduce in this section the techniques and procedures for undertaking these classes of attack, in addition to defences and present suggestions on how and why they work.
%Finally, I ...
It must be stated that the attack and defence surface of \gls{acr:ml} models is very much subject to the same game of cat-and-mouse as any other security domain, i.e. malware or \gls{acr:ddos} design and detection.
This field in particular moves very quickly due to the larger reach and impact of \glspl{acr:dnn} in society as a whole, motivating constant scrutiny by the security community.
As a result, any defences listed are certain to have been invalidated by the time this thesis is read; I hope this section at least provides an illustration of the \emph{classes} of input, output and model transforms that have held some promise.

%?? Discuss attacks on ML models, techniques, paradigms here.

\subsection{Evasion attacks and adversarial examples}\label{sec:evasion-attacks}
\emph{Adversarial examples} are input data which have been subtly modified to trick a machine learning model into producing an incorrect output~\parencite{DBLP:conf/eurosp/PapernotMSW18,DBLP:conf/eurosp/PapernotMJFCS16}.
This problem has been known to security experts for a much longer time under the moniker of \emph{evasion attacks}~\parencite{DBLP:conf/ccs/BarrenoNSJT06}.
The context for these evasions includes cases as simple as spam filter avoidance, and as complex as self-modifying and virtualisation-aware malware~\parencite{DBLP:conf/acsac/CoptyDEEMZ18}.
The term does not purely cover \gls{acr:ml}-based approaches in this context, though there are similarities in the sense that the transformed output must maintain a key property (i.e., it remains a functioning malware payload).

For instance, assume we have in input vector $\mathbf{x}$ with a ground truth label $w$ that the classifier correctly outputs.
An attacker wishes to add some \emph{perturbation} $\symbfit{\delta}$ such that the adversarial example $\mathbf{x}+\symbfit{\delta}$ produces a new output $w'$ from a classifier but still appears to belong to $w$ according to a human observer.
They may require that $w'$ is a specific label, or simply that $w \ne w'$.
These attacks typically assume a white-box attacker (i.e., one who has direct read access to the \gls{acr:ml} model's parameters), who is able to use their knowledge of $\wvec{}$ to compute this $\symbfit{\delta}$.
The data extraction techniques discussed shortly (\cref{sec:data-extraction-and-privacy}) offer more concrete tools for mounting a black-box evasion attack.
Typically, this is then formalised as an optimisation problem in terms of the underlying model, which can be solved via a stochastic optimiser like \emph{Adam}~\parencite{DBLP:journals/corr/KingmaB14}.
To ensure that these alterations are subtle enough to be unnoticeable to a human operator, the constraint to be minimised is some distance metric in $\ell_{\{0,1,2,...,\infty\}}$\sidenote{These specific metrics are the \emph{Hamming} metric $\ell_0$ (number of altered elements in $\mathbf{x}$), \emph{Manhattan} metric $\ell_1$, \emph{Euclidean} metric $\ell_2$, and the \emph{Chebyshev} metric $\ell_\infty$ (the largest change to any element).} between the altered data and its original.
For instance, in a pixel image a bounded $\ell_0$ limits the number of pixels that may be changed, while $\ell_2$ limits the overall strength of noise added.
These adversarial examples typically occur very close to the decision hyperplane; applying too much noise can either accidentally `push' the data into a classification the attacker did not desire, or it may become humanly perceptible.
Since their inception~\parencite{DBLP:journals/corr/SzegedyZSBEGF13}, they have been shown to generalise between models and input vectors~\parencite{DBLP:journals/corr/GoodfellowSS14}.
In the image domain, they have been made transform-resilient~\parencite{DBLP:journals/corr/KurakinGB16}, to transfer to textural information on 3D-printed objects~\parencite{DBLP:journals/corr/AthalyeS17}, and to persist through projection onto surfaces~\parencite{DBLP:journals/corr/abs-2108-06247}.

%?? Indeed, others~\parencite{DBLP:conf/sp/PierazziPCC20} are noticing that most research remains in \emph{feature-space} (i.e., techniques and mathematics), rather than \emph{problem-space}---and most of these are in malware.

A more recent formalisation and strengthening of attacks based on raw input data was recently presented by \textcite{DBLP:conf/sp/Carlini017}.
Around the time of publication, distillation~\parencite{DBLP:conf/sp/PapernotM0JS16} was proposed as a form of hardening for neural networks expected to perform in adversarial settings where evasion attacks might be common.
This work reveals that existing approaches for generating adversarial examples \emph{weren't strong enough} and, accordingly, approaches like defensive distillation are shown to be ineffective.
Some future works refer to the methods they propose as CW-$\ell_{\{0, 2, \infty\}}$ attacks.
Their attacks exceed existing work based on these three well-understood metrics by a more in-depth analysis of the construction of cost functions, a reworked box constraint built around $\tanh(\cdot)$ (as in HDR image tone mapping), and a more nuanced treatment of the effects of discretisation error.
By introducing a \emph{confidence factor} $\kappa$, they are able to explicitly design attacks which are \emph{transferable} between one classifier and its distilled form, or a network derived from the original by black box inference.
%Their work currently establishes the benchmark for future mitigation techniques.

%?? Yeah these are a thing (note: haven't actually read most of these aside from the wonderful turtle-rifle 3d printer one).
%?? The one, the only, the original \parencite{DBLP:journals/corr/SzegedyZSBEGF13}, the application \parencite{DBLP:journals/corr/GoodfellowSS14}.
%?? Second one here suggests that part of the weakness is that models fall back on their heavily linear components -- verify this.

%?? Where are we now? Transform invariant (i.e., photographs \cite{DBLP:journals/corr/KurakinGB16}, 3D model \& 2D transforms \cite{DBLP:journals/corr/AthalyeS17}): what does this mean with regards to the transforms we apply to our input data?

In practice, inputs to \gls{acr:ml} classifiers are often heavily pre-processed or undergo some statistical transformation; either to achieve a fixed-size and compact representation or to increase accuracy.
In this sense we can refer to the `true' inputs as belonging to the \emph{problem space}, while the transformed input given to the \gls{acr:dnn}/\gls{acr:ml} classifier belongs to the \emph{feature space}\sidenote{Image and audio processing are something of an exception to this, where the feature space \emph{is} the problem space. As a result, most adversarial \gls{acr:ml} research targets these domains for simplicity.}.
A malware detector would not, for instance, take an executable's binary as its input, and would instead process behavioural features extracted by static and dynamic analysis tools. 
Of course, these transforms are non-invertible and often non-differentiable, the need to maintain input functionality (in, e.g., malware), and when combined with input validity checks this makes it difficult to create adversarial examples.
A recent frontier on enabling such attacks is a framework for expressing input validity and transformation constraints~\parencite{DBLP:conf/sp/PierazziPCC20}; if feature transformations are approximately differentiable then, otherwise falling back on genetic algorithms and Monte Carlo tree search.

\gls{acr:drl} algorithms are equally vulnerable to this class of attacks, despite the fact that their stochastic nature greatly influences the trajectories gathered during training.
The meaning of an attacker manipulating the environment is, again, problem space dependent, and most work focuses to some extent on reducing agent performance rather than invoking specific actions.
\Textcite{DBLP:journals/corr/HuangPGDA17} have shown that this vulnerability to adversarial inputs extends between \gls{acr:rl} algorithms in white-box settings, while perturbations acquired in a black-box setting on the same \gls{acr:nn} architecture require greater error bounds to invoke the same loss of reward.
An alternative strategy is to directly modify the PPO algorithm, training agents to choose actions with the highest likelihood of making another victim agent perform suboptimally~\parencite{DBLP:conf/uss/Wu0WX21}---i.e., through this adversarial agent's effect on shared environment state via valid actions.

%?? RL algorithms based on NNs are just as vulnerable~\parencite{DBLP:journals/corr/HuangPGDA17}, now~\parencite{DBLP:conf/uss/Wu0WX21}!

%?? Very respectable source \cite{DBLP:conf/eurosp/PapernotMJFCS16} that they're bad I guess? (Not Read)
%
%?? The full summary paper \cite{DBLP:conf/eurosp/PapernotMSW18}.

\paragraph{Defences}
How to not get pwned?

?? \textcite{DBLP:conf/sp/PapernotM0JS16}---distillation---not read, and defeated by CW attacks.

In ensemble classification, if many of the individual classifiers disagree then this can represent a high degree of uncertainty about the observed data.
\textcite{DBLP:conf/ndss/SmutzS16} realise that this uncertainty can act as a powerful indicator of an evasion attack in progress, and propose \emph{mutual agreement analysis} as a defence.
When an insufficient amount of the constituent classifiers return the same result, the result returned is that the sample is `uncertain'---suggesting either a new class of data or evidence of attempted evasion.
The approach naturally suits ensemble methods such as \emph{random forests}, but an extension to SVMs is proposed (an ensemble of feature-bagged SVMs); both are shown to be more effective than standard learning in the face of evasion, mimicry and reverse mimicry attacks.
Moreover, the addition of the `uncertain' classification acts as a useful metric for continuous training and evolution.

Adversarial examples typically occur very close to the decision hyperplane; applying too much noise can either accidentally `push' the data into a classification the attacker did not desire, or it may become humanly perceptible.
This principle is exploited by \textcite{DBLP:conf/acsac/CaoG17}, who propose ensemble classification of potentially adversarial data by sampling from the local hypercube---\emph{region-based} classification, rather than standard \emph{point-based} classification.
This differs from the previous approach; we consider here an ensemble of \emph{classifications} around one data point, rather than an ensemble of \emph{classifiers}.
This approach is remarkable because it may rely on any existing classifier $\mathcal{C}$.
This draws from the observation that most of the volume of the surrounding hypercube (of length $r$) lies within the true class region, even for adversarial examples.
To learn the true class of an example, we must then choose the class which admits the largest volume of overlap with the sample region: we may approximate this by drawing samples uniformly from this hypercube (e.g., \num{10000} points), labelling them with the chosen $\mathcal{C}$ and observing the output histogram.
$r$ is chosen such that classifier performance on benign examples does not degrade below that of $\mathcal{C}$.
Given that this design is non-differentiable, an attacker cannot compose an adversarial attack directly even if they know $r$, the sample count and $\mathcal{C}$ exactly---they must craft examples against $\mathcal{C}$ (which \emph{is} differentiable) and use these.
Detection of the standard set of attacks \cite{DBLP:conf/sp/Carlini017} is shown by very convincing results, meaning that attackers must amplify the applied noise, again risking an incorrect target class or very perceptible distortion.
An interesting area that isn't discussed is how different sampling distributions might affect robustness in the face of even this.

?? Newest on these?~\parencite{DBLP:journals/corr/abs-1902-06705}. Attempts to offer concrete methodology in response to very variable analysis/testing and slow research on defences. Most defences posited in response to the rapid development of attacks are shown to be incorrectly or incompletely evaluated. ?? Human and machine classification/decision-making performance are tied in efficacy, while there remains a huge gap between sensitivity to adversarial examples. The norm is to assume attackers have white-box access, because this includes black-box defence.

?? NN structure analysis~\parencite{DBLP:conf/eurosp/SperlKCLB20}

?? With DiffPriv?~\parencite{DBLP:conf/sp/LecuyerAG0J19}

?? Adversarial Training~\parencite{DBLP:conf/iclr/MadryMSTV18,DBLP:journals/corr/abs-1712-09196}

\Textcite{DBLP:conf/ndss/SmutzS16} examined an ensemble-based defence on top of the PDFrate (PDF malware) and Drebin (Android executables) malware detection systems.
Both of these platforms had well-established adversarial attacks~\parencite{DBLP:conf/ccs/MaiorcaCG13,DBLP:conf/sp/SrndicL14}, built around the constraint that core exploit functionality must be preserved.

Recent work~\parencite{DBLP:journals/corr/abs-2002-04599} suggests an inherent balancing act between sensitivity/invariance-based attacks---in that defence against one creates a vulnerability against the other.
Sensitivity attacks are what we usually consider in this family (a small change which doesn't impact the input's true label), while invariance attacks use a change which \emph{would} change the true label, but is performed in such a way that the model still outputs the old label.
The defence in question would be against attacks within some $\ell_p$ norm ball (i.e., similar pixel/state similarity)---with the findings suggesting that a `robust' neural network is even more sensitive than an undefended one.

?? Adversarial patches?~\parencite{DBLP:conf/uss/0001BSM21}

?? New work~\parencite{DBLP:journals/corr/abs-2002-04599} covering the balancing attack between sensitivity/invariance-based attacks. Sensitivity is what we usually think of (small change which doesn't impact the true label), invariance is a change which \emph{would} change the true label, but is performed in such a way that the model still outputs the old label. Implication: defending against one makes you more sensitive to the other (specifically, attacks within some $\ell_p$ norm ball)---even more sensitive than an undefended network.
?? Even modern provably robust systems based on $\ell_\infty$~\parencite{DBLP:conf/iclr/ZhangCXGSLBH20} are vuln.

\subsection{Poisoning attacks}
?? Attacker wants to (permanently) alter the behaviour of a system which is still training in some way.
The key intuition is that an attacker wishes to affect which data points are used in training, and in turn affect the decision boundaries to do... something bad?
Either sneak something by, or degrade model performance.

?? For systems which assume a stable (stationary) form of normality, it takes an exponential amount of packets with respect to how far the mean must move.
Yet in systems which use a finite number of data points (i.e. modelling non-stationarity) an attacker requires only a linear amount of data if they control a sufficient fraction of the network throughput.
Further findings are more optimistic: if the attacker controls insufficient traffic, then they cannot succeed in appreciably shifting the mean with even an infinite amount of traffic.
It isn't made clear how these findings relate to more complex systems or models, but it will remain an important consideration. \cite{DBLP:journals/jmlr/KloftL10}

?? Backdooring/Trojan~\parencite{DBLP:journals/corr/abs-1712-05526,DBLP:conf/eurosp/TanS20}

?? Online/active learning and FL~\parencite{DBLP:conf/aistats/BagdasaryanVHES20} at risk.

?? Semi-supervised~\parencite{DBLP:conf/uss/Carlini21}

?? on GNNs~\parencite{DBLP:conf/uss/XiPJ021}

?? Clean-label?~\parencite{DBLP:journals/corr/abs-2005-00191}

\paragraph{Defences}
How to not get pwned?

?? Poisoning defences. Auror \cite{DBLP:conf/acsac/ShenTS16} relies upon the fact that masked features (i.e. gradient updates) submitted by users tend to have a known distribution.
By performing 2-means cluster detection on \emph{indicative features}, users whose updates consistently fall outside of the benign distribution may be detected and blacklisted.

?? Input validation~\parencite{DBLP:conf/acsac/DoanAR20}

?? Detecting AI Trojans Using Meta Neural Analysis~\parencite{DBLP:conf/sp/XuWLBGL21}

?? Backdoors~\parencite{DBLP:conf/sp/WangYSLVZZ19}

\subsection{Data extraction and privacy}\label{sec:data-extraction-and-privacy}
Eh.

?? NOTE check FLASHE paper for more attacks on e.g. FL.

Furthermore, it should be noted that the \emph{white-box} requirement can be discarded in network-facing or observable models.
\textcite{DBLP:conf/uss/TramerZJRR16,DBLP:conf/uss/JagielskiCBKP20} have shown that visibility of input-output pairs can allow neural network parameters to be reverse-engineered, and that attacks computed on these surrogates transfer to the target.

While it is feasible that an attacker could start with a \emph{white-box} understanding of your model to aid in evasion, a more feasible situation is the case where they do not.
\textcite{DBLP:conf/uss/TramerZJRR16} have shown that attackers can infer many classes of learned models from observation (\emph{model extraction attacks}), allowing evasion attacks, model cloning or discovery of confidential training data characteristics.
They are able to demonstrate model extraction on logistic regressors, SVMs, MLPs and NNs with and without confidence values.
Furthermore, from their retrained models they are able to extract images of faces which are ``visually indistinguishable'' from the real training data of a facial recognition classifier.
As far as defences go, they recommend: hiding or rounding confidence values (to limited success); applying differential privacy to mode parameters (an open question); and ensemble methods (which may still be beaten by evasion attacks).

?? Put this somewhere: can extract portions of training data w/ carefully crafted inputs~\parencite{DBLP:journals/corr/abs-2012-07805}. Not read, though maybe relate to privacy in general? I.e., differentially private methods.

?? state of art in black box?~\parencite{DBLP:conf/uss/HeM0HH21}
This is reduction of the input dataset (assuming you and target model have similar data) to learn the `most valuable points' in your dataset to query from an victim's model.

?? Data-free model extraction~\parencite{DBLP:conf/cvpr/TruongMWP21}

?? Just look at PCIe lol~\parencite{DBLP:conf/uss/ZhuCZL21}

?? Feature leakage?~\parencite{DBLP:conf/sp/MelisSCS19}

\paragraph{Defences}
How to not get pwned?

Link some diffpriv here?

?? Huh~\parencite{DBLP:journals/corr/abs-2103-07101}

?? Machine Unlearning~\parencite{DBLP:conf/sp/BourtouleCCJTZL21}
