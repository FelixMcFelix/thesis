\gls{acr:ml} training and inference work in the domain of real numbers, and thus require a suitable data format for representation of weights, gradients, and values.
\emph{Floating-point arithmetic} allows commodity machines and accelerators to approximate real numbers in a fixed-size representation, dividing available bits among a \emph{sign}, \emph{exponent}, and \emph{mantissa}.
This captures several important properties, principally \emph{dynamic range} (as the exponent describes the magnitude of a number).
For instance, \qty{32}{\bit} floating-point numbers (FP32) use \qtylist{1; 8; 23}{\bit} to store each component, which is sometimes known as a 1-8-23 representation.

However, there are concrete reasons to consider other data formats, particularly on more resource-constrained environments.
Quantisation and alternative data formats have been suggested to make \gls{acr:ml} inference feasible on resource- and power-limited platforms, work around hardware constraints, or compute faster and more efficiently.
Although individual floating-point operations, as compared to integers, have effectively equal latency and reciprocal throughput on modern x86 hardware~\parencite{agner-x86}\sidenote{This leaves aside the performance gains offered by \gls{acr:simd} vectorisation, which is a trickier topic.}, \glspl{acr:fpu} still require additional chip area and power.
Naturally, chip designers don't want to fabricate or plan around unnecessary \glspl{acr:fu}: for instance, (programmable) network hardware and \glspl{acr:asic} require only basic integer arithmetic.
This is not the only reason to be interested in alternative data formats; by reducing the size of any individual number from \qty{32}{\bit} to \qtylist[list-pair-separator = { or }]{16; 8}{\bit}, we reduce the size of parameter sets and input vectors by \qtyrange{2}{4}{\times}.
This reduces the range of numbers we can express (in both magnitude and precision), but can reduce inference latency and memory cost for the benefit of both commodity machines and dedicated accelerators.
Luckily, the task of bringing \gls{acr:ml} models to resource-constrained environments without these capabilities is well-studied, and in general the effect on accuracy is small in spite of the introduced quantisation noise.

\subsection{Floating-point}
While there are standardised floating-point forms designed to target mobile and weaker hardware, such as \emph{half-precision} (FP16, \qty{16}{\bit}, 1-5-10) and \emph{minifloat} (FP8, \qty{8}{\bit}, 1-4-3), these fail to be effective in some \gls{acr:ml} use cases.
At the same time, a key factor in \gls{acr:fpu} chip cost is the size of the \emph{mantissa}---which has been observed to have a quadratic scaling effect on area in \gls{acr:tpu} development~\parencite{bfloat16-blog}.
Accordingly, allocating more bits to the \emph{exponent} can allow for more cores and \glspl{acr:fu} in the same area, or reduce power draw.
\emph{bfloat16}~\parencite{bfloat16-blog} is a \qty{16}{\bit} format employed in (among other devices) Google TPUs~\parencite{DBLP:journals/sigops/XieDMKVZT18} and modern Intel Xeon server \glspl{acr:cpu}~\parencite{intel-bfloat}.
It matches the dynamic range of \qty{32}{\bit} floats (1-8-7), better expressing the smaller end of the dynamic range (for e.g., gradients) while having identical failure modes (subnormal numbers, edge cases) to FP32.
\emph{hfp8}~\parencite{DBLP:conf/nips/SunCCWVSCZG19}, as an \qty{8}{\bit} format, uses different layouts for the forward (1-4-3) and backward (1-5-2) passes, applying a downward bias of \num{4} to the exponent in both cases.
This allows better expression of small values in general, and even smaller values during gradient computation, at an extra \qty{5}{\percent} hardware area cost to support both formats.
While this is an indicative summary , it must be said that there are more formats beyond the scope of this introduction~\parencite{DBLP:journals/corr/abs-2007-01530}.

\subsection{Fixed-point and binary}
While the above techniques are promising and effective, deployment on environments such as \gls{acr:pdp} hardware requires further ingenuity.
\emph{Fixed-point arithmetic} is an approach which makes it possible to represent real numbers as integers, losing dynamic range as a consequence.
$Qm.n$ expresses a real number using an $m$~\unit{\bit} integer part alongside an $n$~\unit{bit} fractional part, which allows us to evaluate and update policies using only integer arithmetic on $k=1+m+n$~\unit{\bit} numbers, assuming the presence of a sign bit.
For instance, the \qty{8}{\bit} number \mintinline{rust}|0b0010_1000| in $Q3.4$ represents the real number \num{2.5}---we can view this as two separate parts ($2 + 8\times2^{-4}$), or as one whole in the fractional base ($40\times2^{-4}$).
In practice, the entirety of the number is stored as a two's complement number in place of a sign bit, and base conversion (i.e., changing $n$) requires only bitshifts.
When $k$ is known, we can simply refer to the representation as $Qn$.
The most useful part of this scheme is that integer addition and subtraction are \emph{unchanged} for two $Q$ numbers, and conversion of a normal integer requires an $n$~\unit{\bit} left shift.
Multiplication and division \emph{by a normal integer} can be performed using the standard \gls{acr:alu} operations, while $Q$ numbers need an additional bitshift (pre-/post-op base correction) and temporary expansion into a $2k$~\unit{\bit} register.

Although these tradeoffs seem to predispose fixed-point towards \emph{only} target environments without \glspl{acr:fpu}, it has still enjoyed application in inference and training.
In particular, \textcite{DBLP:journals/corr/CourbariauxBD14} were able to train Maxout networks without substantial error with $k$ at \qty{19}{\bit}, or \qty{11}{\bit} using dynamic scaling ($m=5$).
In distributed training, fixed-point arithmetic is useful as an intermediate representation for in-NIC gradient aggregation~\parencite{DBLP:conf/nsdi/SapioC0NKKKMPR21,DBLP:conf/nsdi/LaoLMCWAS21}.
Scaling down \glspl{acr:nn} to INT8 requires careful calibration~\parencite{tensorrt-8bit}, binning the activations per layer in an FP32 network to choose optimal saturation thresholds.
Training INT8 \glspl{acr:dnn} directly on mobile/\gls{acr:iot} devices has become more possible~\parencite{DBLP:conf/usenix/Zhou0QGXZGLZ21}, though this still requires floating point hardware to compute simple compensation terms after moving all tensor operations to fixed-point.
To some extent, these can also enable in-\gls{acr:nic} \gls{acr:dnn} inference~\parencite{langlet-ml-netronome}.

Binary representations are used to great effect in \glsxtrfullpl{acr:bnn}~\parencite{DBLP:journals/corr/MiyashitaLM16,DBLP:conf/eccv/RastegariORF16,DBLP:journals/corr/KimS16,DBLP:conf/nips/HubaraCSEB16}; by using 0 to represent to \num{-1}, and 1 to represent $+$\num{1}, we may replace Hadamard product operations between tensors with \textsc{Xnor} operations, and when combined with \textsc{Popcnt} instructions we can easily compute the dot product between vectors.
This offers bit-parallel execution compatible with almost all \glspl{acr:alu}, including \gls{acr:pdp} hardware.
This enables in-\gls{acr:nic} execution of \glspl{acr:nn}: either offloading small portions of fully-connected layers to accelerate inference~\parencite{DBLP:conf/sigcomm/SanvitoSB18} or to enable line-rate packet processing~\parencite{DBLP:journals/corr/abs-2009-02353,DBLP:journals/corr/abs-1801-05731}.
While there is vested interested in running these complex function approximators in \glspl{acr:nic} and switches, they are at present trained on commodity x86 machines using real-valued weights and gradients clamped to $\left[-1, 1\right]$ (i.e., using $\tanh$).
As such, online in-\gls{acr:nic} training remains out of reach.