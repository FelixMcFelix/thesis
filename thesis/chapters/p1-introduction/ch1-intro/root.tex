\chapter{Introduction}\label{chap:intro}

%\epigraphhead{\epigraph{Insert thoughtful quote here.}{Some author---\emph{Some Book}}}

?? Remember: Topic Sentences!

%?? PROBLEM
Computer networks are complex, yet critical infrastructure.
?? layers upon layers.
?? keeping them well-oiled is...
?? why we want DDN: in combination, many aspects of networks complex, heuristic, hand-tuned

?? TOOLS DDN
?? Meteoric rise of \gls{acr:ml}
?? an new idea takes the stage
?? why we want \gls{acr:ddn}: in combination, many aspects of networks complex, heuristic, hand-tuned

?? TOOLS PDP
?? such \gls{acr:pdp} hardware can...
?? The resurgence of PDP
?? an old idea returns to the fore
?? What's fascinating is that these ideas are not entirely novel.

?? HOW TOOLS HELP
?? Why we want to run complex stuff in the network
?? The complexities of running complex stuff in the network

?? SOLUTION
?? The intersection
?? new data to play with
?? data reduction to make it feasible to export
?? in-situ processing

?? cusp of a promising field
?? or networking is on the cusp of two promising fields?

\section{Thesis statement}
This thesis asserts that:
\begin{quotation}
	\noindent
	Data-driven networking---enhancing networks with \gls{acr:ml}---and data plane programmability are key tools in .
	
	Programmable dataplanes and in-network computation can make data-driven networking more efficient, effective, and responsive.
	Programmable dataplanes and in-network computation can enable new data-driven networking use-cases.
	
	
	?? Modern programmable network hardware will allow these key parts of data-driven or RL systems to run within the network fabric, improving performance and response times.
	?? \emph{Reinforcement learning can lead to improved performance in network optimisation and control problems, such as DDoS prevention.}
	?? \emph{Programmable dataplanes and in-network computation can enable new data-driven networking use-cases.}
	?? \emph{Programmable dataplanes and in-network computation can make data-driven networking more efficient, effective, and responsive.}
	
	Taken together, programmable data-driven networks improve computer network operation beyond the sum of their parts.
\end{quotation}

To unpack some of this...

\section{Contributions}
From the main goals: ?? Subdivide these into constituent parts?
\begin{itemize}
	\item \emph{A thorough summary of the literature on modern computer networks (including recent hardware trends) (\cref{chap:nets}), and of machine learning techniques suitable for their control (\cref{chap:ddn})}. Describe in detail... ?? Not just how ML benefits networks, but how creative networking can benefit ML use cases \textbf{Link this to a specific part of the thesis statement.}
	\item \emph{A novel synthesis of best practices, design decisions, and environmental tradeoffs to consider in the design of \gls{acr:ml}-led system control (\cref{sec:ddn-use-takeaways})}. Describe in detail... \textbf{Link this to a specific part of the thesis statement.}
	\item \emph{Marl DDoS (\cref{chap:ddos-rl})}. Describe in detail... \textbf{Link this to a specific part of the thesis statement.}
	\begin{itemize}
		\item Part 1...
	\end{itemize}
	\item \emph{In-Network RL (\cref{chap:in-net-rl})}. Describe in detail... \textbf{Link this to a specific part of the thesis statement.}
	\begin{itemize}
		\item Part 1...
	\end{itemize}
	\item \emph{\seidr{} (\cref{chap:seidr})}. Describe in detail... \textbf{Link this to a specific part of the thesis statement.}
\end{itemize}

At the same time, my intent in writing this thesis is to collect together enough of the literature on \gls{acr:ddn} and \glspl{acr:pdp} to serve as a comfortable introduction to a newer researcher in the field.
To be the hypothetical book I would have wanted to read and use as a starting point when I was setting out on this research venture.

?? Topic of \gls{acr:ddn} in particular bloomed during the course of this---scarcely existed when this work was first undertaken.
?? as such, I'm one case study among many?

In addition to the main goals:
\begin{itemize}
	\item Procedures for modelling and generating traffic similar to modern Opus-based \gls{acr:voip} flows (\cref{adx:opus-traffic}).
\end{itemize}

\section{Thesis Outline and Structure}
Broadly speaking, this thesis is presented in two halves.
The first offers in-depth background on both the fields of \gls{acr:ddn} and \gls{acr:pdp}:
\begin{description}
	\item[\Cref{chap:nets}] describes the evolution of computer networks from purely fixed-function devices towards increased programmability in both the control plane and dataplane---critically examining early research directions in contrast with modern successes. It then describes how modern dataplanes allow new, or more performant, networked applications---offloading and in-network compute---before discussing the threat landscape of volumetric \gls{acr:ddos} attacks.
	\item[\Cref{chap:ddn}] description
\end{description}
The second half presents novel, concrete use cases which each demonstrate a part of the thesis statement (as discussed above):
\begin{description}
	\item[\Cref{chap:ddos-rl}] description
	\item[\Cref{chap:in-net-rl}] description
	\item[\Cref{chap:seidr}] description
\end{description}
\Cref{chap:conclusion} then concludes the thesis by summarising its main takeaways, offering closing thoughts on these fields, and outlines future work specific to the above use cases.

Additional, supplementary details follow:
\begin{description}
	\item[\Cref{adx:caida-traffic}] describes the methodology and results of a small-scale study on the distribution of protocols in \gls{acr:caida} trace data, to establish a rough estimate of congestion-unaware traffic's presence in \gls{acr:isp} networks.
	\item[\Cref{adx:opus-traffic}] provides additional detail on the measurement process used to collect trace data for simulating \gls{acr:voip}-like traffic, as well as the software architecture for packet generation.
	\item[\Cref{adx:nfp-arch}] expands on architectural details for the \gls{acr:nfp} family of SmartNICs to offer some additional context for \approachshort{}'s design constraints.
	\item[\Cref{adx:opal-proto}] contains packet header and protocol descriptions for \approachshort{}'s control protocol.
\end{description}