\chapter{Netronome NFP Architectural Details}
\label{adx:nfp-arch}

As we implement this work on Netronome NFP SmartNICs, it is necessary to explain their basics.
These \emph{system-on-a-chip} (SoC) devices achieve scalable packet processing through sheer parallelism.
Most of the chip is composed of \emph{microengines} (MEs), grouped into \emph{islands} of 4 or 12 MEs.
All 12-ME islands are used by a default P4 pipeline.
Each ME has \numrange{4}{8} \emph{contexts} (hardware threads) which share a code store.
%Contexts and MEs may send one another numbered signals, and MEs have a small \emph{next-neighbour} register file for passing values in one direction to the next ME on the same island.
%MEs run a proprietary instruction set, compiled to via a \emph{(Micro-)C} compiler.
Beyond registers, the platform implements an explicit memory hierarchy scaling in size, location, and access cost:
$\text{LMEM (ME)} < \text{CLS (Island)} < \text{CTM} < \text{IMEM (Chip)} < \text{EMEM}$.

?? Fill this out in more detail.
