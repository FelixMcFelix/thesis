%Traffic is played back from hosts via Tcpreplay at a bandwidth assigned uniformly from a `good' or `bad' distribution, each using the same pcap file with source and destination IP addresses rewritten.

This work is most naturally compared against Marl, introduced by \textcite{DBLP:journals/eaai/MalialisK15}, the state-of-the-art in \gls{acr:rl}-based \gls{acr:ddos} prevention.
We are most interested in seeing how their approach contrasts with the new agent designs across different topologies and workloads.
Different network environments will also impose different levels of host density, where popular web servers may have orders of magnitude more clients than egress points from their network---I aim to show how these characteristics affect performance and learning rate.
Marl is known to outperform the AIMD~\parencite{DBLP:journals/ton/YauLLY05} strategy, yet the state of the art has long since moved on.
To paint a more current picture, I compare this work against an effective modern approach, \emph{SPIFFY}~\parencite{DBLP:conf/ndss/KangGS16}.
SPIFFY tests a proportion of flows by routing them through an alternate path with higher bandwidth, observing how their speed changes some time later.
This comparison lets us position our new agent designs against the state of the art, observing that SPIFFY has a similar mode of interaction to \gls{acr:rl}-based systems (taking action, observing an effect, and acting once again) and does not rely on protocol characteristics or signatures.
In reimplementing SPIFFY, I make the simplifying assumption that a suitable unused path exists (with identical bandwidth to the server's link).
\qty{10}{\percent} of active flows were tested at a time (according to the authors' observation that there is a factor of \qty{10}{\times} difference between the ideal and achieved bandwidth expansion), excluding flows below \qty{50}{\kilo\bit\per\second} and requiring a \qty{3}{\times} expansion from legitimate flows, making a judgement after \qty{5}{\second}.

To test this, I made use of both traffic models introduced in \cref{sec:a-new-normal} (Opus and \gls{acr:tcp}), both topologies discussed below (1-dest vs.\ Fat-Tree), and vary the amount of hosts typically communicating over each agent's ingress/egress node.
Additionally, these new models were evaluated in multi-agent mode (\emph{separate}, no model sharing), and in single-agent mode (\emph{single}, zero-cost perfect information sharing).
In each case, the algorithm's performance was averaged over \num{10} episodes of length \num{10000} timesteps (setting each agent's $\wvec{}=\mathbf{0}$ between episodes).
Host allocations at the beginning of each episode were generated pseudorandomly to ensure fairness between episodes---a host is malicious with probability $\operatorname{P}\left(\mathit{malicious}\right)$, and is benign otherwise.
Benign hosts generate traffic according to either \cref{sec:tcp-http-traffic-model,sec:udp-opus-traffic-model} depending on the experiment, while malicious hosts generate traffic as described in \cref{sec:attack-traffic-model} (both at experiment-dependent rates).

All experiments were executed on Ubuntu 18.04.2 LTS (GNU/Linux 4.4.3-040403-generic x86\_64), using an Intel Core i7-6700K (\qtyproduct[product-units=single]{4 x 4.2}{\giga\hertz}) which had \SI{32}{\gibi\byte} of \gls{acr:ram}.
%All code underpinning these findings is available on a public repository\footnote{\url{https://github.com/FelixMcFelix/rln-dc-ddos-paper}}.
%All code underpinning these findings is available on a public repository.\footnote{Private until publication.}

\subsection{Single destination}\label{sec:single-dest}
%?? Move description of tree topol to here.
The network is tree-structured, where one server $s$ connects through a dedicated switch to $k$ team leader switches, each connected to $\ell$ intermediate switches, which in turn each connect to $m$ egress switches.
We then have $N_{\mathit{hosts}} = k \ell m n$.
\Cref{fig:marl-topol} demonstrates this.
%Although \citeauthor{DBLP:journals/ccr/MahajanBFIPS02a}, the originators of this topology, make it clear that it exists as a fairly unrepresentative example \cite{DBLP:journals/ccr/MahajanBFIPS02a}, it remains the case that such a network topology allows for functional testing, and indeed is illustrative of one way in which attack traffic might aggregate in the network.
%It is hard, however, to argue its relevance to specific classes of victim or to reason about the interactions it might have with dependent applications.
%We aim to address this through \cref{sec:performance-in-an-emulated-environment}.
The network topology was configured using $k=2$ teams, $\ell=3$ intermediate nodes per team, $m=2$ agents per intermediate node, and $n \in \{2, 4, 8, 16\}$ hosts per learner.
This is a slight simplification of \Textcite{DBLP:journals/eaai/MalialisK15}'s \textquote{online} experiment, choosing fewer teams but remaining as a single server with a fan-out network.
%The algorithm parameters were set at $\gamma=0$ (leading to opportunistic behaviour), $\alpha=0.05$, having linearly annealed $\epsilon=0.2 \rightarrow 0$ by $t=3000$.
%Benign and malicious hosts uploaded between \SIrange{0}{1}{\mega\bit\per\second} and \SIrange{2.5}{6}{\mega\bit\per\second} respectively, and hosts were redrawn at each episode's start with $\operatorname{P}(\mathit{malicious})=0.4$.
%$U_s$  $k \ell mn+2$ \si{\mega\bit\per\second}.
%The performance of each choice of $n$ was averaged over \num{10} episodes of length \num{10000} timesteps (setting each agent's $\wvec{}=\bm{0}$ between episodes).
%Host allocations were generated pseudorandomly to ensure fairness between choices of $n$.
%These parameter choices match those of the original study to enable direct comparison, and are (to the best of our knowledge) arbitrary, but we justify our range of $n$ as capturing increasing scales of host activity.

\begin{figure}
	\centering
	\resizebox{0.9\linewidth}{!}{\begin{tikzpicture}[
	texts/.style = {text=black},
	labeltexts/.style = {text=uofgsandstone},
	treeline/.style = {draw=uofgburgundy},
	treenode/.style = {texts, circle, centered, fill=white, treeline},
	load/.style = {fill=uofgcobalt},
	loadhide/.style = {fill=uofgcobalt!40!white},
	external/.style = {fill=uofgrust},
	externalhide/.style = {fill=uofgrust!40!white},
	hideline/.style = {draw=uofgsandstone!40!white},
	hidenode/.style = {treenode, hideline},
	grow'=right
]
	\node[treenode, label={[texts]above:Server}] (root) {}
	child [treeline] { node [treenode, label={[texts]above:Core}] (sswitch) {}
		child [treeline] { node [treenode, label={[texts]above:Leader}] (teaml) {} 
			child [treeline] { node [treenode, label={[texts]above:Intermediate}] (inter) {}
				child [treeline] { node [treenode, load, label={[texts]above:Agent/Egress}] (agent) {}
					child [treeline] { node [treenode, external] (extern) {}
						child [treeline] { node [treenode, external, label={[texts]above:Host}] (host) {} }
						child [hideline] { node [hidenode, externalhide] (endhost) {} }
					}
				}
				child [hideline] { node [hidenode, loadhide] (endagent) {} }
			}
			child [hideline] { node [hidenode] (endinter) {} }
		}
		child [hideline] { node [hidenode] (endteaml) {} }
		edge from parent
		node[below, labeltexts] {$U_s$}
	};
	
	%\draw[-] (teaml) -- (endteaml);
	\node [labeltexts] (kdots) at ($(teaml)!0.5!(endteaml)$) {$\rvdots$};
	\node [labeltexts, right = -0.1cm of kdots] {$k$};
	\node [labeltexts] (ldots) at ($(inter)!0.5!(endinter)$) {$\rvdots$};
	\node [labeltexts, right = -0.1cm of ldots] {$\ell$};
	\node [labeltexts] (mdots) at ($(agent)!0.5!(endagent)$) {$\rvdots$};
	\node [labeltexts, right = -0.1cm of mdots] {$m$};
	\node [labeltexts] (ndots) at ($(host)!0.5!(endhost)$) {$\rvdots$};
	\node [labeltexts, right = -0.1cm of ndots] {$n$};
\end{tikzpicture}}
	\caption[Tree-structured network topology diagram for evaluating a single-destination network.]{
		Network topology diagram, showing how the server and its core switch's $k$ teams are structured, with $\ell$ intermediate routers per team, connected to $m$ agents which each moderate $n$ hosts beyond a single external switch.
		%	Empty nodes are considered to be internal.
		Red nodes are external, and each blue node hosts an agent.
		\label{fig:marl-topol}
	}
\end{figure}

\subsection{Multiple destinations}
The previous topology allows for direct comparison against the state-of-the-art, and indeed is illustrative of one way in which attack traffic might aggregate in the network.
It is hard, however, to argue its relevance to specific classes of victim or to reason about the interactions it might have with dependent applications.
In contrast, the fat-tree topology~\parencite{DBLP:conf/sigcomm/Al-FaresLV08} sees regular use in real-world data centres and scales well horizontally.
%?? Come up with description of fat-tree (multi-dest) topol.
%?? Why fat tree? regularly appears in modern datacentres.
%?? $k=4$ fat-tree , with one pod hosting two servers $s_0,s_1$.
We use a $k=4$ fat-tree, with one pod hosting two servers $s_0$ and $s_1$.
$n$ external hosts connect through each core switch (where agents are hosted), and communicate with $s_0, s_1$ uniformly randomly.
Both servers host identical services.
We set $n \in \{6, 12, 24, 48\}$ hosts per learner (keeping $N_{\mathit{hosts}}$ identical to each tier of the single-host topology), and restrict $U_{s_0} = U_{s_1} = U_s / 2$.

\subsection{Parameters}
The algorithm parameters were set at $\alpha=0.05$, linearly annealing $\epsilon=$ \num{0.2} $\rightarrow$ 0 by $t=$~\num{3000} in the case of Marl (\num{8000} actions per agent in the \emph{Instant/Guarded} models).

Benign hosts each occupied \qtyrange{0}{1}{\mega\bit\per\second}, and hosts were redrawn at each episode's start with $\operatorname{P}(\mathit{malicious})=$~\num{0.4}.
%The original introduction of this approach to direct-control reinforcement learning as introduced by \textcite{DBLP:journals/eaai/MalialisK15} fails to consider key cases: the absence of a suitable heuristic classifier $g(\cdot)$, disjoint ranges of traffic distribution (i.e., the presence of benign heavy-hitters), the accurate simulation of TCP-like behaviour (and its effects on collateral damage), and high densities of hosts at egress points.
%?? Why? ...
%Of these, the latter two are most deserving of a closer investigation, as they have stronger implications for wide-scale deployment.
%These are important issues, particularly when we consider real-world deployment.
%Heuristic estimates of traffic legitimacy come with computational cost and couple the reward function to the accuracy of the estimator, hosts often show diversity in their own traffic patterns (perhaps being multi-modal), and it is known that TCP is the most used transport protocol for Internet traffic \cite{DBLP:conf/saint/ZhangDJC09}.
%?? NEED TO VERIFY VOLUME OF CONGESTION-AWARE PROTOCOLS
Malicious hosts each sent \qtyrange{2.5}{6}{\mega\bit\per\second} when attacking \gls{acr:udp} traffic, though this was increased to \qtyrange{4}{7}{\mega\bit\per\second} when using \gls{acr:tcp}-like traffic (to meaningfully impact benign flows).
Given $n$ and $\operatorname{P}(\mathit{malicious})$, we see an expected malicious bandwidth \numrange{1.27}{1.87} and \qtyrange{2.03}{2.18}{\times} $U_s$ respectively.
%The expected fraction of $U_s$ consumed by each host is \SI{21.5}{\percent} for $n=2$, and \SI{2.84}{\percent} for $n=16$.
For these choices of $n$ in both topologies, we observe $N_{\mathit{hosts}} \in \left\{24, 48, 96, 192\right\}$, and an expected number of malicious hosts $\mathbb{E}\left[N_{\mathit{attackers}}\right] \in \left\{9.6, 19.2, 38.4, 76.8\right\}$.
For the largest choice of $n$, we see an expected total attack traffic $\mathbb{E}\left[V_{\mathit{attack}}\right] =$ \qtylist{334.05;422.4}{\mega\bit\per\second} for Opus and \gls{acr:http} traffic respectively.

$U_s$ was fixed at $N_{\mathit{hosts}}+2$ \unit{\mega\bit\per\second} (to account for burstiness), and each link had a delay of \qty{10}{\milli\second}.
All links had unbounded capacity, save for each server-switch.
These parameters match those of the original study to enable direct comparison, and many are (to the best of our knowledge) arbitrary, but I justify the range of $n$ as capturing increasing scales of host activity.