%?? Try and compare my work here when possible?

\fakepara{DDoS Prevention}
\Textcite{DBLP:conf/lcn/BragaMP10} examine the detection of flooding DDoS attacks through \emph{self-organising maps}, using SDN to gather statistics effectively.
Many of their features aren't overly relevant, as their focus is not active defence or discovering \emph{which} hosts contribute to an attack.
%?? Actually talk about Marl (???) to appease reviewer \#1.
The closest available approach within this field is that of \textcite{DBLP:journals/eaai/MalialisK15} (whom we have positioned our work against), and their contribution in applying RL to the task of intrusion prevention is significant: their work helps to show the viability of live, adaptive, feedback-loop-like control of the network to detect and prevent DDoS attacks.
They create a tree overlay topology (subdivided into teams), where each agent applies packet drop to \emph{all} flows inbound to a protected server.
%?? Recap their flaws, since they've been cut form every other aspect.
Our results show that their technique underperforms at high host density and when congestion-aware traffic dominates---that their results do not demonstrate this suggests an evaluation driven purely by traces (rather than live application dynamics).

\emph{SPIFFY} \cite{DBLP:conf/ndss/KangGS16} aims to remedy transit-link attacks by observing how flows from each source respond to a sudden increase in available bandwidth.
\Citeauthor{DBLP:conf/ndss/KangGS16} realise that bots participating in an attack are often unable to match this bandwidth expansion (having already saturated the capacity of their outbound links), while legitimate flows typically speed up to match the new fair-share rate.
%Attackers must either be detected or reduce the throughput of each bot, increasing the cost of launching an attack.
%Unlike our approach (and due to the class of attacks it is designed to defend against), SPIFFY is intended to be deployed within ASes, although .
A weakness of their approach is that computing a route to measure bandwidth expansion on real networks can be costly (up to \SI{14}{\second} for the Cogent topology), and that the low expansion factors in real network can require more ``rounds'' of filtering.
By contrast, our approach takes a constant time to compute an action for a flow regardless of topology size.
Their assumptions about traffic response to such bandwidth expansion do not hold for constant bitrate flows (e.g., VoIP) and may not extend to HTTP DASH flows, both of which make up a sizeable proportion of network traffic.

\emph{Athena} \cite{DBLP:conf/dsn/LeeKSPY17} is a generalised SDN framework for intrusion detection, but has shown the use of a \emph{k-nearest neighbours} classifier to detect individual attack flows.
Although heavyweight (and proven to be effective compared with \textcite{DBLP:conf/lcn/BragaMP10}), their comparison against SPIFFY lacks the quantitative evidence required to understand how the system compares.
\Textcite{DBLP:conf/sp/SmithS18} use AS-level routing to tackle both transit-link and flooding-based attacks.
This view is taken due to the perceived cost of per-stream classification and inherent sensitivity to adversarial examples.
The approach is creative, relying upon BGP \emph{fraudulent route reverse poisoning} to preserve traffic to a target AS, but unlike SPIFFY the approach doesn't actually \emph{remove} the congestion.
Because of this, flooding-based attacks aren't fully alleviated.

%?? Abuses of RL 
\fakepara{RL in Networks}
Earnest, well-considered application of RL towards the challenge of intrusion prevention has seen comparatively little examination.
Past work treats the paradigm as a traditional classifier for anomaly detection \cite{shamshirband2014anomaly} and DDoS prevention \cite{DBLP:conf/mates/ServinK08}.
Given that the main strengths of RL techniques are the ability to control ongoing interaction and adapt by observing the concrete effects of actions, such works don't apply the rich literature on the subject to its fullest potential.

For categorising how RL fits into solving problems, we label works as direct- or indirect-control RL.
A \emph{direct-control} RL problem is one where the RL agent(s) learn optimal control over a set of actions as the \emph{primary} defence or decision-maker---requiring measurements, reward functions and action sets tailored for this purpose.
%We feel there is a shortage of work in this category at present, at least in the field of networks.
To date, the best-fitting example we have encountered is that of \textcite{DBLP:journals/eaai/MalialisK15}.
An \emph{indirect-control} RL problem is one where agents act in service to \emph{another technique} responsible for decision-making, optimising or generalising aspects of its operation beyond that of hand-coded heuristics.
A past example includes learning when best to share knowledge between \emph{hidden Markov model} anomaly detectors \cite{DBLP:conf/paisi/XuSH07}.
%The position of this work is weakened by its reliance on the problematic `DARPA99' dataset \cite{DARPA-IDD, DBLP:conf/cisda/TavallaeeBLG09, DBLP:conf/sp/SommerP10}, but the idea itself is well-treated and this acts as a driver for improvements in this direction.
This work is weakened by its reliance on the problematic `DARPA99' dataset \cite{DBLP:conf/sp/SommerP10}, but the idea itself is well-treated.
Outside of intrusion detection, there has been growing interest in the use of RL in data-driven networking, such as for intra-AS route optimisation \cite{DBLP:conf/hotnets/ValadarskySST17} and resource-constrained process allocation \cite{DBLP:conf/hotnets/MaoAMK16}.
\textcite{DBLP:conf/sigcomm/MaoNA17} employ client-side observations of network state and video performance with RL to optimise bitrate selection for multimedia streaming.
\emph{AuTO} \cite{DBLP:conf/sigcomm/ChenL0L18} employs deep RL to perform traffic optimisation.
Crucially, they find that the vast majority of flows are short-lived, requiring effective decisions in less than a millisecond.
To overcome the high latency of action computation via a neural network, two agents are trained, handling aspects of short and long flows respectively.
The first learns to optimise the flow size thresholds to demarcate long and short flows; these short flows are routed by ECMP.
The second agent makes bespoke decisions about routing, prioritisation etc.\ for each of the remaining long flows.
