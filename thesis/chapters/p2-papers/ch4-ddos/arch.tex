%?? Here is where I would put my ramblings about system architecture...
We present our design of a system which supports the effective real-world deployment of RL-based DDoS mitigation.
\Cref{fig:sys-arch} displays this, separating system elements which are local to each agent from those which reside elsewhere in the network.
We operate each agent as a \emph{virtualised Network Function} (vNF) adjacent to a software-defined switch.
SDN allows a controller (or authorised hosts) to install actions, forwarding rules and logic into a switch at runtime.
Moreover, networks of this kind more naturally enable the future relocation and easy installation of learners.
Agent vNFs communicate with these co-hosted switches to install probabilistic packet-drop rules.
We describe the main purpose and operation of each module within an agent's vNF, and discuss techniques to make deployment more efficient using existing technologies.

\subsection{Core and RL Executor}\label{sec:core-and-rl-executor}
The core module is the main loop in an agent's architecture.
At each timestep, the core receives information about which flows have arrived and should be acted upon from the \emph{TRS Scheduler}.
The core then retrieves the current and previous state vector associated with each flow from the \emph{Flowstate Database}, passing them into the RL algorithm alongside the last action chosen for that flow (if available).

The RL algorithm then returns an action.
Each action is passed to the database, which computes and returns a packet drop rate according to the agent model (\emph{Instant/Guarded}) while updating flow state.
This is then converted into an OpenFlow message carrying packet drop rules; these are batched to the agent's switch using the same groupings produced by the scheduler.
Finally, timing information is passed back into the scheduler to refine its estimates about how much work should be scheduled in the next timestep.

State space sizes guarantee that an \emph{Instant} policy remains under \SI{520}{\kibi\byte}, though our sparse representation typically leads to far smaller policies: $\sim$\SI{17.8}{\kibi\byte} from our experiments.
\emph{Guarded} policies are \SI{30}{\percent} of this size.
As described earlier, action updates require a constant number of floating point operations---\num{160} floating point additions and \num{32} multiplications per update of $\wvec{}$ with per-tile updates, with \num{160} additions required to choose an action.
The vast majority of these operations can be vectorised.
Action computation for \emph{Guarded} agents is cheaper still, requiring only \num{48} additions per action.

\subsection{Stats API and Collectors}
Agents require information from the network and one another to be effective.
Our agents can act either independently, having no agent-to-agent communication, or cooperatively.
In the latter case agents transfer, when possible, \emph{experience} to one another---lists of state-action-state transitions with associated rewards.
It's noted that a transition may be high-value or surprising to one agent, while well-known to another, causing each to produce different policy updates from the same unit of experience.
For this reason we do not transfer policy deltas between agents, causing each to learn its own policy.
Which scheme achieves better performance is left to future work.

%?? How do stats get from switcehs to the agent?
Load collectors and estimators periodically push observations to each active agent vNF.
In our current implementation, load statistics are gathered via vNFs active at each network switch, though we expect that OpenFlow stats requests, NetFlow or SNMP data may be used to derive these cheaply.
Transferring state to agents and experience sharing can both be made more efficient through effective use of broadcast addressing in a target network.
Depending on the capabilities of switches in the network, the estimator can either send benign traffic estimates or parameters for use in a reward function.

Gathering and transmission of load/flow statistics would be difficult to perform quite as often as an emulated environment allows.
However, the measurements acquired in such a scenario are likely to be less noisy (by being collected over longer periods of time), which could aid effective training.

\subsection{Flowstate Database}
For each flow 5-tuple, we hold two state vectors containing the features described in \cref{sec:feature-space}---the current state, and the state which induced the last action.
To ensure that flow control actions are made with recent information, we combine state vectors for unvisited flows in the current work set.
State vector combination is done by summing deltas and packet counts, updating means via weighted sums, and replacing all other fields with the newest measurements.
For flows outside of the current work list, we simply replace the stored vector.

\subsection{TRS Scheduler}
Acting on an unbounded set of flows in each timestep introduces potential issues: the inability to respond to unexpected changes in flow state, delayed service of new flows, and the risk that flow states become outdated.
At their worst, these risks present additional attack surface to an adversary.
To tackle these problems, we make use of \emph{Timed Random Sequential} updates.

The scheduler begins with a shuffled work list of active flows.
When requested, the scheduler estimates the cost of an action computation using the timing information received from the core, proceeding down the list to send a set of 5-tuples to the core which can be handled in a set time limit.
The scheduler continues until the list is empty, at which point it is repopulated and reshuffled with active flows.

Following \citeauthor{DBLP:conf/sigcomm/ChenL0L18}'s observations on handling short flows \cite{DBLP:conf/sigcomm/ChenL0L18}, we maintain a deadline of \SI{1}{\milli\second}---an agent is typically able to process around 3 flows in this time.
We expect deadlines should be tuned based on the frequency at which statistics arrive.
The amount of processed flows per deadline depends on the agent design (FLOP count, policy size), but also on the amount of flow telemetry data needing processed---our current implementation is written in python, restricting this handling to a single thread.
An implementation in a systems language such as Rust or C++ would allow faster concurrent processing.

There is a risk that so much work can be queued up that an agent is never able to act on some attack flows.
A solution is to impose an upper bound on the amount of action computations/policy updates that can be performed before the work list is regenerated.
This removes the guarantee that all flows will be visited often, but if updates occur regularly then this sampling may be sufficient to achieve good performance.

\subsection{Agent Switches}
%?? How do we do stat collection?
Our agent switches operate a modified version of Open vSwitch, implementing an action which requests that each matched packet be dropped with a certain probability.
To get around the lack of floating-point support in many environments, we represent this probability using a 32-bit unsigned integer (scaling \num{1.0} to $2^{32}-1$).
On commodity hardware, we believe that a similar effect can be achieved using OpenFlow meters (at the expense of these being stateful measures).

We use OpenFlow groups to simplify control messages: premade tables with permitted levels of packet drop.
This saves some overhead compared to using experimenter/extension headers.
Flows are automatically given a group with the default level of packet drop (according to the chosen agent design), meaning that switches don't need to refer to a controller or the agent vNF.

%?? What do our modifications to openflow look like? How do we get around the lack of floating point math?

\begin{figure}
	\centering
	\resizebox{0.8\linewidth}{!}{\begin{tikzpicture}
		\node(remote){Remote};
		
		%%%
		
		\node[below=0.05 of remote](swpos){};
		\node[fill=white!80!uofgcobalt, draw=black, minimum height=1.5cm, minimum width=2cm, below right= 0.1 of swpos.north west](sw1){};
		\node[fill=white!80!uofgcobalt, draw=black, minimum height=1.5cm, minimum width=2cm, below right= 0.1 of sw1.north west](sw2){};
		\node[fill=white!80!uofgcobalt, draw=black, minimum height=1.5cm, minimum width=2cm, below right= 0.1 of sw2.north west](switch){};
		\node[below right, inner sep=2pt] at (switch.north west) {\small Switch};
		\node[fill=white!90!uofgcobalt, draw, rectangle, rounded corners=0.05cm, above=0.1] (oswlc) at (switch.south) {\begin{varwidth}{1.5 cm}\small \centering Load\\Collector\end{varwidth}};
		
		%
		
		\node[right=2 of swpos](epos){};
		\node[fill=white!80!uofgthistle, draw=black, minimum height=1.5cm, minimum width=3.5cm, below right= 0.1 of epos.north west](e1){};
		\node[fill=white!80!uofgthistle, draw=black, minimum height=1.5cm, minimum width=3.5cm, below right= 0.1 of e1.north west](e2){};
		\node[fill=white!80!uofgthistle, draw=black, minimum height=1.5cm, minimum width=3.5cm, below right= 0.1 of e2.north west](egress){};
		\node[below right, inner sep=2pt] at (egress.north west) {\small Egress Switch};
		\node[fill=white!90!uofgthistle, draw, rectangle, rounded corners=0.05cm, above=0.1] (eglc) at ($(egress.south) + (-0.85,0)$) {\begin{varwidth}{1.5 cm}\small \centering Load\\Collector\end{varwidth}};
		\node[fill=white!90!uofgthistle, draw, rectangle, rounded corners=0.05cm, right=0.1] (egest) at (eglc.east) {\begin{varwidth}{1.5 cm}\small \centering Estimator\\$g(\cdot)$\end{varwidth}};
		
		%
		
		\node[right=3.5 of epos](apos){};
		\node[fill=white!60!uofgpumpkin, draw=black, minimum height=1.5cm, minimum width=2cm, below right= 0.1 of apos.north west](a1){};
		\node[fill=white!60!uofgpumpkin, draw=black, minimum height=1.5cm, minimum width=2cm, below right= 0.1 of a1.north west](a2){};
		\node[fill=white!60!uofgpumpkin, draw=black, minimum height=1.5cm, minimum width=2cm, below right= 0.1 of a2.north west](otheragent){};
		\node[below right, inner sep=2pt] at (otheragent.north west) {\small Agent vNF};
		\node[fill=white!90!uofgpumpkin, draw, rectangle, rounded corners=0.05cm, above=0.3] (oasa) at (otheragent.south) {\begin{varwidth}{1.5 cm}\small \centering Stats API\end{varwidth}};
		
		%
		
		\node[below=2.3 of remote.west](linestart){};
		\path let \p1 = (linestart) in node (lineend) at (9,\y1){};
		\draw [dashed] (linestart) -- (lineend);
		
		%%%
		
		\node[below=2.4 of remote](local){Local};
		
		%%%
		
		\node[below=0.05 of local](aswpos){};
		\node[fill=white!80!uofgthistle, draw=black, minimum height=3.5cm, minimum width=2.2cm, below right= 0.1 of aswpos.north west](aswitch){};
		\node[below right, inner sep=2pt] at (aswitch.north west) {\small Agent Switch};
		\node[fill=white!90!uofgthistle, draw, rectangle, rounded corners=0.05cm, above=0.1] (aswoft) at (aswitch.south) {\begin{varwidth}{1.5 cm}\small \centering OpenFlow\\Tables\end{varwidth}};
		\node[fill=white!90!uofgthistle, draw, rectangle, rounded corners=0.05cm, above=0.1] (aswsc) at (aswoft.north) {\begin{varwidth}{1.5 cm}\small \centering Stats\\Collector\end{varwidth}};
		\node[fill=white!90!uofgthistle, draw, rectangle, rounded corners=0.05cm, above=0.1] (aswlc) at (aswsc.north) {\begin{varwidth}{1.5 cm}\small \centering Load\\Collector\end{varwidth}};
		
		%
		
		\node[right=3 of aswpos](avfpos){};
		\node[fill=white!60!uofgpumpkin, draw=black, minimum height=3.5cm, minimum width=4.5cm, below right= 0.1 of avfpos.north west](avf){};
		\node[below right, inner sep=2pt] at (avf.north west) {\small Agent vNF};
		\node[fill=white!90!uofgpumpkin, draw, rectangle, rounded corners=0.05cm, below=0.15] (avfsa) at ($(avf.north) + (0.2,0)$) {\begin{varwidth}{1.5 cm}\small \centering Stats API\end{varwidth}};
		\node[fill=white!90!uofgpumpkin, draw, rectangle, rounded corners=0.05cm, below=0.9] (avfdb) at (avfsa.south west) {\begin{varwidth}{1.5 cm}\small \centering Flowstate\\Database\end{varwidth}};
		\node[fill=white!90!uofgpumpkin, draw, rectangle, rounded corners=0.05cm, right=0.1] (avfsched) at (avfdb.east) {\begin{varwidth}{1.5 cm}\small \centering TRS\\Scheduler\end{varwidth}};
		\node[fill=white!90!uofgpumpkin, draw, rectangle, rounded corners=0.05cm, below=2.3] (avfcore) at (avfsa.south) {\begin{varwidth}{1.5 cm}\small \centering Core\end{varwidth}};
		\node[fill=white!90!uofgpumpkin, draw, rectangle, rounded corners=0.05cm, right=0.8] (avfrl) at (avfcore.east) {\begin{varwidth}{1.5 cm}\small \centering RL\end{varwidth}};
		
		%%%
		
		\tikzset{>=stealth}
		
		\draw[thick, ->] (aswlc) -- (avfsa.west) node[midway,above] {\tiny Current load};
		\draw[thick, ->] (aswsc) -- (avfsa.west) node[midway,sloped, above] {\tiny Flow stats};
		
		\draw[thick, ->] (oswlc) -- (avfsa) node[midway,above, sloped] {\tiny Current load};
		
		\draw[thick, ->] (eglc) -- (avfsa) node[midway,above, sloped] {\tiny Current load};
		\draw[thick, ->] (egest) -- (avfsa) node[midway,above, sloped] {\tiny Estimation data};
		
		\draw[thick, <->] (oasa) -- (avfsa) node[midway,above, sloped] {\tiny Experience};
		
		\draw[thick, ->] (avfsa) -- (avfdb) node[midway,above, sloped] {\tiny State};
		\draw[thick, ->] (avfsa) -- (avfsched) node[midway,above, sloped] {\tiny Live flows};
		
		\draw[thick, ->] (avfcore) -- (aswoft) node[midway,above, sloped] {\tiny Packet drop rules};
		\draw[thick, <->] (avfcore) -- (avfdb) node[midway,below, sloped] {\tiny State};
		\draw[thick, <->] (avfcore) -- (avfsched) node[midway,below, sloped] {\tiny Work};
		\draw[thick, <-] ($(avfcore.east) + (0,0.1)$) -- ($(avfrl.west) + (0,0.1)$) node[midway,above, sloped] {\tiny Actions};
		\draw[thick, ->] (avfcore) -- (avfrl) node[midway,below, sloped] {\tiny State};
		
		\draw[thick, ->] (avfsa.east) to [out=0, in=45] (avfrl.north);
		\node[right=0.3] at (avfsa.east) {\tiny Experience};
	\end{tikzpicture}}
	\caption{
		System architecture  for our RL-driven DDoS defence system.
		\label{fig:sys-arch}
	}
\vspace{-1em}
\end{figure}