\chapter{DDoS Prevention by Multi-agent Reinforcement Learning}\label{chap:ddos-rl}
Network anomaly detection and intrusion detection/prevention are continually evolving problems, compounded by the partial, non-\gls{acr:iid} view of data at each point in the network.
Looking ahead to our discussion of \gls{acr:ddos} attacks in \cref{sec:problems-in-modern-networks}, attacks and anomalous behaviours evolve, becoming more sophisticated or employing new vectors to harm a network or system's confidentiality, integrity, and availability without being detected~\parencite{DBLP:journals/comsur/BhuyanBK14}.
These attacks and anomalies have measurable consequences and symptoms which allow a skilled analyst to infer new signatures for detection by signature-based classifiers, but unseen attacks may only be defended against after the fact.
%This issue is inherent to \emph{misuse-} or \emph{signature-based} intrusion detectors, and it has been long-hoped that \emph{anomaly-based} detectors would surpass this by making effective use of statistical measures \cite{DBLP:journals/comsur/BhuyanBK14}.

%?? ML would seem like a natural fit for this problem.
%\gls{acr:ml} and other statistical approaches have , ... but languished. ?? try chain straight into DDoS: domains where labels and rewards derivable at runtime.
%While \emph{machine learning} (ML) approaches seem like a sensible fit for this problem, in \citeyear{DBLP:conf/sp/SommerP10} \citeauthor{DBLP:conf/sp/SommerP10} identified the `failure to launch' of ML-based anomaly detection systems---a distinct lack of real-world system deployments \cite{DBLP:conf/sp/SommerP10}.
%To quite a large extent, this remains the case today.
%They posit that their use is made difficult due to significant operational differences from standard ML tasks, including: the high cost of errors and extraordinarily low tolerance for false positives inherent to network intrusion detection \cite{DBLP:conf/ccs/Axelsson99}; a general lack of recent, openly available (and high-quality) training data; and diversity of network traffic across varying timescales combined with significant burstiness \cite{DBLP:journals/ccr/LelandWTW95}.
%Above the aggregate level, the constant deployment of new services and protocols means that traffic is \emph{non-stationary} and displays an evolving notion of normality.
%Learning is made harder still by the challenges encountered with unlabelled (often partial) data.
%All of these factors greatly inflate the difficulty of the detection problem.

%?? Make it clearer here what problem I specifically want to solve: principally a particular class of DDoS attacks; volume-based DDoS attacks. Amplification attacks are just a specialisation, this can be made more obvious. I think I need to be clearer about the \emph{intended deployment environment} of service hosts (i.e., not ISPs).

We've already discussed how \gls{acr:ml} and \gls{acr:ddn} were hoped to make this domain easier to solve---e.g., automatic detection of attack signatures, reliable anomaly detection---and the operational limits which have become clear in \cref{sec:ddn-uses-security}.
Consider on one point in particular, namely the availability of useful training data.
In many of these cases, anomalous events still require human expertise to label and detect; to complicate matters, this effort must be sustained in the face of network evolution.
For certain classes of problem, e.g., volumetric \gls{acr:ddos} attacks, system health corresponds to wider load and performance metrics which are typically more easily known.
It is here that \gls{acr:rl} offers another perspective.
%?? Unclear explanation of RL here?
Recall from \cref{chap:ddn} that \gls{acr:rl} agents operate by following a \emph{policy} to interact with or control a system, while at the same time using observed performance metrics and deliberate exploration to dynamically improve this policy.
In this way the role of an \gls{acr:rl} agent differs from that of a standard classifier, adaptively reacting to threats by assuming the role of a feedback loop for network optimisation, typically to safeguard service guarantees.
In a sense, this allows us to ``overcome'' some of the difficulties of the detection problem by monitoring \emph{performance characteristics and consequences} in real-time; by looking for (and controlling) the effect rather than the cause.
%Long-term, we expect that the value of RL-based defence systems will be to augment what existing misuse-based solutions can provide, by automatically alerting, recording and controlling what are believed to be illegal system states.
%The goal of this work is much less general; we aim to prevent volume-based DDoS attacks with the aid of RL-based techniques (an important goal in its own right), while bringing to light the flexibility and applicability of these techniques in the security domain.

%Whether it takes direct control of the network, or is used indirectly to optimise a key part of another system, more powerful `deep' RL techniques (and well-founded action spaces) aren't yet well explored for network IDS/IPS.
%These range from more modern training algorithms \cite{DBLP:journals/corr/SchulmanWDRK17, DBLP:conf/icml/SchulmanLAJM15}, to evolutionary strategies \cite{DBLP:journals/corr/SalimansHCS17, DBLP:journals/corr/abs-1802-08842}, hierarchical action composition \cite{DBLP:journals/corr/abs-1710-09767}, and competitive multi-agent learning \cite{DBLP:journals/corr/abs-1710-03748}.

%To date, there have been few applications of this class of algorithms towards intrusion detection and prevention which make use of their full potential for online control, rather than using them as the basis for a classifier.
%We aim to take steps to redress this and establish their proper capabilities, beyond simple ``blind application''.
%?? Expand as required
%?? THAT IS A MAJOR CONTRIB, MENTION IT EVERYWHERE YOU CAN

%?? Discuss the most important conclusions before the outline.
What \gls{acr:rl} approaches do exist are aimed towards the task of adaptive online \gls{acr:ddos} mitigation, and rely upon learning to control probabilistic packet drop.
These have concrete weaknesses compared to the reality of network traffic.
\Cref{sec:ddn-uses-security} presents my analysis of the existing work for this task---\emph{Marl}~\parencite{DBLP:journals/eaai/MalialisK15}---particularly how it fails to account for congestion-aware traffic (i.e., \gls{acr:tcp}) and environments with high host density per egress point such as \glspl{acr:isp} or datacentre networks.
As a result, they achieve poor protection of legitimate traffic due to an overly coarse view of the network and the dominance of congestion-aware traffic in today's networks (\cref{sec:ddos-contributing-factors,adx:caida-traffic}).
However, there are limits to how we should infer these properties given network evolution---we must remain protocol- and content-agnostic to offer future-proofing against the rollout of protocols like QUIC.

%To remedy this, we make throttling decisions on a per-source basis and present the engineering decisions this mandates: updating RL agents from multiple traces per timestep, timed random sequential action computation and a supporting \emph{software-defined network} (SDN) architecture.
%In tandem with the development and evaluation of an effective state space and model, we provide the design of a second model inspired by past work on algorithmic DDoS prevention, as an example of the integration of domain-specific knowledge.
%Our introduction of per-source decisions improves substantially upon the state-of-the-art when acting upon most internet traffic (i.e., congestion-aware protocols), and we show that our second model achieves excellent performance for high host density in this case.
%Crucially, both models remain protocol- and content-agnostic to offer future-proofing against the rollout of future protocols like QUIC \cite{DBLP:conf/sigcomm/LangleyRWVKZYKS17}.
%?? Also algorithmic enhancements such as multiple actions per timestep, 
%?? PROTOCOL-AGNOSTIC -- HOW WILL THESE THINGS COPE WITH QUIC ET AL.?!?!

%\subsection{Contributions}
%This paper contributes two source-level granularity approaches to RL-driven DDoS prevention (\emph{Instant} and \emph{Guarded} action models), improving upon past aggregate-based models (\cref{sec:ddos-mitigation-with-per-flow-reinforcement-learning}).
%These are designed to make effective decisions irrespective of protocol, and act on individual flows at the edge of any network topology.
%We offer an in-depth investigation into suitable features for automatic DDoS mitigation, with qualitative and quantitative justification (\cref{sec:rethinking-the-state-space}).
%These features have been suggested by past studies, and independently tested in their own contexts.
%Our study is the first attempt to quantify the individual efficacy of each in an RL setting.
%
%We implement reactive simulations of HTTP and VoIP web-server traffic, designed to test system characteristics that packet trace playback fails to capture (\cref{sec:a-new-normal}).
%%To our knowledge, this is the first attempt to study or replicate Opus-based VoIP traffic, which has become commonplace since the codec's release in 2012.
%These new traffic models inform an empirical evaluation of our new models against the state-of-the-art in RL-based DDoS mitigation using (\cref{sec:the-results-of-doing-so}), alongside a discussion of security concerns and real-world deployment (\cref{sec:discussion}).
%We additionally compare our work against SPIFFY \cite{DBLP:conf/ndss/KangGS16}, reuniting two divergent strands of research and grounding the study of RL-based DDoS defences.

This chapter considers how online \gls{acr:rl} can be used to defend networks from volumetric \gls{acr:ddos} attacks, agnostic of the protocols of carried traffic, and is based upon \citetitle{DBLP:journals/tnsm/SimpsonRP20}~\parencite{DBLP:journals/tnsm/SimpsonRP20}.
I first explain the existing threat and defence landscape of such attacks (\cref{sec:problems-in-modern-networks}), then reiterate the motivation for \gls{acr:rl} as a solution (\cref{sec:ddos-motivation}), before defining the threat model for attackers with respect to known \gls{acr:ddos} attack methods and the security context of \gls{acr:ml} (\cref{sec:ddos-threat}).
\Cref{sec:ddos-mitigation-with-per-flow-reinforcement-learning} then outlines the design and rationale behind new agent designs built to improve on the failings of past \gls{acr:rl} works, by making decisions on a per-flow or per-source basis.
This includes algorithmic modifications to learn from and control many traces simultaneously, achieve faster convergence by per-tile updates, and to better learn from individual features.
I describe a feature space built on a mixture of global and local state, reward functions tailored to different attack classes, and contribute two action models and their risks (\emph{Instant} and \emph{Guarded}).
The \emph{Guarded} model is inspired by past work on algorithmic \gls{acr:ddos} prevention, as an example of how the integration of domain-specific knowledge can lead to more effective \gls{acr:rl} agents in shorter timescales.
\Cref{sec:ddos-architecture} then describes how state measurement and installation of actions could be managed in an \gls{acr:sdn} deployment.
To determine \emph{which} per-flow features are worth using for \gls{acr:ddos} control and mitigation, I then present qualitative and quantitative analysis of a large selection of these metrics for different agent designs on varied protected traffic types (\cref{sec:rethinking-the-state-space}).
\Cref{sec:a-new-normal} then details my implementation of reactive simulations of \gls{acr:http} and \gls{acr:voip} web-server traffic, designed to test system characteristics that packet trace playback fails to capture.
\Cref{sec:ddos-evaluation,sec:the-results-of-doing-so} then describe and show empirical performance measurements of the two new agent designs against existing \gls{acr:rl} \gls{acr:ddos} techniques, and algorithmic works via \emph{SPIFFY}~\parencite{DBLP:conf/ndss/KangGS16}, reuniting two divergent strands of research and grounding the study of \gls{acr:rl}-based \gls{acr:ddos} defences.
I conclude with some discussion on the significance of the results and wider security implications of this solution in particular (\cref{sec:ddos-discussion}), and summarise in \cref{sec:ddos-summary}.

%?? \Cref{chap:ddos-rl} investigates using multi-agent \gls{acr:rl} to automatically learn the features of attack traffic online. I explore agent designs informed by past \gls{acr:rl} approaches (and their failures) relative to the realities of Internet traffic. State spaces in particular are experimentally justified to find `per-feature' value. A system architecture as part of a larger \gls{acr:vnf} system is shown, followed by evaluation of efficacy on different traffic classes and scenarios.
%
%?? Although there are variations of each class of attack, flooding attacks are the most relevant to our work.

% \section{Threat Model}
% \section{Agent Design}
% \subsection{Weaknesses in Prior Art}
% \subsection{Feature Spaces}
% \subsection{Reward Functions}
% \subsection{Action Spaces}
% \subsection{Systems and Threat Considerations}
% \section{System Design}
% \section{Modified Learning Algorithms}
% \section{Methodology}
% \subsection{Traffic Modelling}
% \subsection{Topologies and Testing Environments}
% \section{Evaluation}
% \subsection{Congestion-aware Traffic}
% \subsection{Congestion-unaware Traffic}
% \subsection{Increased Attack Volume}
% \subsection{Computational Cost}

%\section{Introduction}
%Network anomaly detection and intrusion detection/prevention are continually evolving problems, compounded by the partial, non-\emph{independent and identically distributed} (IID) view of data at each point in the network.
Attacks and anomalous behaviours evolve, becoming more sophisticated or employing new vectors to harm a network or system's confidentiality, integrity, and availability without being detected \cite{DBLP:journals/comsur/BhuyanBK14}.
These attacks and anomalies have measurable consequences and symptoms which allow a skilled analyst to infer new signatures for detection by misuse-based classifiers, but unseen attacks may only be defended against after-the-fact.
This issue is inherent to \emph{misuse-} or \emph{signature-based} intrusion detectors, and it has been long-hoped that \emph{anomaly-based} detectors would surpass this by making effective use of statistical measures \cite{DBLP:journals/comsur/BhuyanBK14}.

While \emph{machine learning} (ML) approaches seem like a sensible fit for this problem, in \citeyear{DBLP:conf/sp/SommerP10} \citeauthor{DBLP:conf/sp/SommerP10} identified the `failure to launch' of ML-based anomaly detection systems---a distinct lack of real-world system deployments \cite{DBLP:conf/sp/SommerP10}.
To quite a large extent, this remains the case today.
They posit that their use is made difficult due to significant operational differences from standard ML tasks, including: the high cost of errors and extraordinarily low tolerance for false positives inherent to network intrusion detection \cite{DBLP:conf/ccs/Axelsson99}; a general lack of recent, openly available (and high-quality) training data; and diversity of network traffic across varying timescales combined with significant burstiness \cite{DBLP:journals/ccr/LelandWTW95}.
Above the aggregate level, the constant deployment of new services and protocols means that traffic is \emph{non-stationary} and displays an evolving notion of normality.
Learning is made harder still by the challenges encountered with unlabelled (often partial) data.
All of these factors greatly inflate the difficulty of the detection problem.

%?? Make it clearer here what problem I specifically want to solve: principally a particular class of DDoS attacks; volume-based DDoS attacks. Amplification attacks are just a specialisation, this can be made more obvious. I think I need to be clearer about the \emph{intended deployment environment} of service hosts (i.e., not ISPs).

For certain classes of problem e.g., volumetric \emph{distributed denial of service} (DDoS) attacks, \emph{reinforcement learning} (RL) offers another perspective.
%?? Unclear explanation of RL here?
RL agents operate by following a \emph{policy} to interact with or control a system, while at the same time using observed performance metrics and deliberate exploration to dynamically improve this policy.
In this way the role of a RL agent differs from that of a standard classifier, adaptively reacting to threats by assuming the role of a feedback loop for network optimisation, typically to safeguard service guarantees.
In a sense, this allows us to ``overcome'' some of the difficulties of the detection problem by monitoring \emph{performance characteristics and consequences} in real-time; by looking for (and controlling) the effect rather than the cause.
Long-term, we expect that the value of RL-based defence systems will be to augment what existing misuse-based solutions can provide, by automatically alerting, recording and controlling what are believed to be illegal system states.
The goal of this work is much less general; we aim to prevent volume-based DDoS attacks with the aid of RL-based techniques (an important goal in its own right), while bringing to light the flexibility and applicability of these techniques in the security domain.
%Whether it takes direct control of the network, or is used indirectly to optimise a key part of another system, more powerful `deep' RL techniques (and well-founded action spaces) aren't yet well explored for network IDS/IPS.
%These range from more modern training algorithms \cite{DBLP:journals/corr/SchulmanWDRK17, DBLP:conf/icml/SchulmanLAJM15}, to evolutionary strategies \cite{DBLP:journals/corr/SalimansHCS17, DBLP:journals/corr/abs-1802-08842}, hierarchical action composition \cite{DBLP:journals/corr/abs-1710-09767}, and competitive multi-agent learning \cite{DBLP:journals/corr/abs-1710-03748}.

To date, there have been few applications of this class of algorithms towards intrusion detection and prevention which make use of their full potential for online control, rather than using them as the basis for a classifier.
We aim to take steps to redress this and establish their proper capabilities, beyond simple ``blind application''.
%?? Expand as required
What approaches do exist are aimed towards the task of adaptive online DDoS mitigation, and rely upon learning to control probabilistic packet drop.
%?? THAT IS A MAJOR CONTRIB, MENTION IT EVERYWHERE YOU CAN

%?? Discuss the most important conclusions before the outline.
We find that the existing work for this task \cite{DBLP:journals/eaai/MalialisK15} fails to account for congestion-aware traffic (i.e., TCP) and environments with high host density per egress point, achieving poor results due to an overly coarse view of the network.
To remedy this, we make throttling decisions on a per-source basis and present the engineering decisions this mandates: updating RL agents from multiple traces per timestep, timed random sequential action computation and a supporting \emph{software-defined network} (SDN) architecture.
In tandem with the development and evaluation of an effective state space and model, we provide the design of a second model inspired by past work on algorithmic DDoS prevention, as an example of the integration of domain-specific knowledge.
Our introduction of per-source decisions improves substantially upon the state-of-the-art when acting upon most internet traffic (i.e., congestion-aware protocols), and we show that our second model achieves excellent performance for high host density in this case.
Crucially, both models remain protocol- and content-agnostic to offer future-proofing against the rollout of future protocols like QUIC \cite{DBLP:conf/sigcomm/LangleyRWVKZYKS17}.
%?? Also algorithmic enhancements such as multiple actions per timestep, 
%?? PROTOCOL-AGNOSTIC -- HOW WILL THESE THINGS COPE WITH QUIC ET AL.?!?!

\subsection{Contributions}
This paper contributes two source-level granularity approaches to RL-driven DDoS prevention (\emph{Instant} and \emph{Guarded} action models), improving upon past aggregate-based models (\cref{sec:ddos-mitigation-with-per-flow-reinforcement-learning}).
These are designed to make effective decisions irrespective of protocol, and act on individual flows at the edge of any network topology.
We offer an in-depth investigation into suitable features for automatic DDoS mitigation, with qualitative and quantitative justification (\cref{sec:rethinking-the-state-space}).
These features have been suggested by past studies, and independently tested in their own contexts.
Our study is the first attempt to quantify the individual efficacy of each in an RL setting.

We implement reactive simulations of HTTP and VoIP web-server traffic, designed to test system characteristics that packet trace playback fails to capture (\cref{sec:a-new-normal}).
To our knowledge, this is the first attempt to study or replicate Opus-based VoIP traffic, which has become commonplace since the codec's release in 2012.
These new traffic models inform an empirical evaluation of our new models against the state-of-the-art in RL-based DDoS mitigation using (\cref{sec:the-results-of-doing-so}), alongside a discussion of security concerns and real-world deployment (\cref{sec:discussion}).
We additionally compare our work against SPIFFY \cite{DBLP:conf/ndss/KangGS16}, reuniting two divergent strands of research and grounding the study of RL-based DDoS defences.

\section{Distributed denial of service}\label{sec:problems-in-modern-networks}
While computer networks are prone to all manner of operational problems on account of their gradual, continued construction via many complex interlocking systems, we train our focus here on \glsxtrfull{acr:dos} and \glsxtrfull{acr:ddos} attacks.\sidenote{A vast array of other, keenly relevant problems are briefly explained while motivating the \gls{acr:ddn} use cases presented throughout \cref{sec:use-cases}.}
\gls{acr:ddos} attacks are concentrated efforts by many hosts to reduce the availability of a service, typically to inflict financial harm or as an act of vandalism.
Attackers achieve this by either exploiting peculiarities of \gls{acr:os} or application behaviour in \emph{semantic attacks} (e.g., \emph{\texttt{SYN} flooding attacks}), or overwhelming their target through sheer volume of requests or inbound packets (\emph{volume-based attacks})~\parencite{DBLP:conf/imc/JonkerKKRSD17}.
Hosts often participate unwillingly, typically having been recruited into a \emph{botnet} by malware infection to be orchestrated from elsewhere~\parencite{DBLP:conf/uss/AntonakakisABBB17}.

Why focus on this problem in particular?
The primary reason is that its scale and impact presents a constant threat to any Internet service.
Exhausting all of a server's resources (or those of the infrastructure providing a path to it) ensures that it cannot be accessed---causing financial losses, silencing information sources, or creating downstream service breakages.
Some services, such as those associated with game hosting, are likely to be targeted simply for competitive advantage or reputation~\parencite{aws-shield-review-2020}.
Accordingly, \gls{acr:ddos} attacks are often nicknamed an `800-pound gorilla'~\parencite{DBLP:conf/imc/CzyzKGPBK14} on the wider Internet.
Their reach is, however, made all that much greater by the centralisation of many websites and servers to cloud and hypergiant infrastructure.
Consider volumetric attacks on Dyn (\qty{1.2}{\tera\bit\per\second}), who at that time hosted key resources for Twitter, Spotify, and Netflix~\parencite{dyn-ddos-2016}, the web host OVH (\qty{1}{\tera\bit\per\second})~\parencite{ovh-ddos-2016}, and the Github code hosting platform (\qty{1.35}{\tera\bit\per\second})~\parencite{github-ddos-2018}.
Amazon's own services have been an attractive target on several occasions: S3's \gls{acr:dns} servers were hit by an attack of unknown size in October 2019 which was unmitigated~\parencite{amazon-s3-2019-ddos}, while AWS successfully resisted \qty{2.3}{\tera\bit\per\second} of traffic mere months later~\parencite{aws-shield-2020-q1}.
Even individuals' blogs such as KrebsOnSecurity (\qty{665}{\giga\bit\per\second})~\parencite{krebs-ddos-2016} have been high-profile targets.
The more solutions and insight the research community can provide, the better.

The second reason is that \gls{acr:ddos} defence scenarios may be a representative example of the kinds of closed loop control that \gls{acr:ddn} is exceptionally well-suited to.
Target servers and infrastructure expose useful state such as link utilisations, queue depths, and service metrics; an influx of attack traffic has noticeable effects on this state, and taking the `right' control plane actions (e.g., blackholing specific traffic sources or protocols, filtering out attack packets) should move the network's state closer to some degree of normality.
At the same time, \gls{acr:ddos} strategies evolve over the course of a single attack~\parencite{DBLP:conf/spw/KangGS16}, potentially leading to a nuanced (and difficult) measure$\rightarrow$infer$\rightarrow$act loop.
%?? natural choice for the problems and services of future networks
An ideal, human-designed solution to this control loop is made tricky by the complex interplay of attack dynamics with many existing elements of the network: protocol distribution and behaviour, application behaviour, and the gradual evolution of benign traffic.
For this reason, I focus on \gls{acr:ddos} attacks as a particular use case in this thesis---in turn, \cref{chap:ddos-rl} is dedicated to improving automated, data-driven means for their solution.

%?? Attacks nowadays present on the order of \si{\tera\bit\per\second}~\parencite{github-ddos-2018,dyn-ddos-2016,krebs-ddos-2016,ovh-ddos-2016,aws-shield-2020-q1}, and attack strategies vary from direct volume-based attacks~\cite{DBLP:conf/ndss/Rossow14, DBLP:conf/uss/KuhrerHRH14} to those which a victim cannot directly observe~\cite{DBLP:conf/sp/KangLG13, DBLP:conf/esorics/StuderP09}.



%?? ----------------
%?? IDEA: 2022. How to present:
%?? Problem at big scale: DoS attacks
%?? Problem at small scale: Measurement, microburst, fairness etc.
%?? ----------------

%\subsection{Attack classes}
%?? Big survey \cite{DBLP:conf/imc/JonkerKKRSD17}?
\Textcite{DBLP:conf/imc/JonkerKKRSD17} offer an in-depth analysis and taxonomy of the landscape of \gls{acr:dos} attacks.
They observe that Denial-of-Service is most commonly achieved through \emph{resource exhaustion}---either at the target server or the infrastructure serving it.
Attacks may then be classified on two orthogonal axes: \emph{Direct vs.~Reflection} and \emph{Volumetric vs.~Semantic}.
\begin{description}
	\item[Direct] Attackers send packets directly towards their target. Random \gls{acr:ip} spoofing is often used to make blocklisting more difficult, but leaves evidence of an attack and its characteristics due to \emph{backscatter}, visible to network telescopes~\parencite{DBLP:conf/lisa/Moore03,DBLP:conf/imc/RichterB19}.
	\item[Reflection] Attackers send traffic to a \emph{reflector}, spoofing the source \gls{acr:ip} of packets to match that of the target. The reflector sends replies to the target, often \emph{amplifying} them in the process.
	\item[Volumetric] \gls{acr:dos} is achieved by \emph{resource exhaustion}---\gls{acr:cpu} or \gls{acr:ram} usage at a target host, or occupying and overflowing transmission buffers along key traffic routes. These can be service agnostic, but in some cases rely on buggy behaviour of other software as their main mechanism.
	\item[Semantic] \gls{acr:dos} is achieved by \emph{exploiting program logic}, for instance to crash a target application server. These are often tailor-made for a particular service or its deployment environment, such as \emph{teardrop attacks} against a host's \gls{acr:tcp} stack.
\end{description}
We'll focus mainly on volumetric attacks (direct and reflection), as these are the attack vectors applied in the listed, real-world attacks.

%?? How to fit in w/ rest?
%They find that TCP tends to be the leading transport for direct random-spoofing attacks, while reflection and amplification attacks are dominated by UDP-like protocols (NTP$>$DNS$>$CharGen).
%Randomly spoofed direct attacks are found to last longer, and are most intense around ``web ports'' (HTTP, SSH, etc.), evidence is seen to support the existence of ``joint attacks''.
%The scale of all attacks is immense, by their measurements: at least a third of the internet is under attack at any one point in time, with at least \num{30000} attacks \emph{visible} each day.

\subsection{Volumetric attacks}

\paragraph{Amplification attacks}
Amplification attacks abuse network services with small request bodies and large responses, causing a typically benign service to forward traffic on an attacker's behalf by \emph{reflection}---spoofing the source \gls{acr:ip} of requests to that of the intended victim.
An attacker requires that their \gls{acr:as} doesn't prevent \gls{acr:ip} spoofing at egress.
In exchange, they are able to split their upstream bandwidth across many reflectors to gain higher volumes of attack traffic from multiple sources without revealing their own \gls{acr:ip} to the victim.
\gls{acr:udp}-based protocols are the typical basis for such attacks, as the transport is connectionless.

While \gls{acr:dns} is the most well-known vector for amplification, \textcite{DBLP:conf/ndss/Rossow14} presents an in-depth survey of a wide variety of other vulnerable protocols alongside a rough census of abusable servers.
He examines network services (SNMPv2, \gls{acr:ntp}, \gls{acr:dns}, NetBIOS, SSDP), legacy services (CharGen, QoTD), peer-to-peer networks (BitTorrent, Kad), online games (Steam, Quake 3) and externally abusable botnets (ZAv2, Sality, Gameover).
Scanning for \num{e5} amplifiers of a popular service can be done in minutes, making \gls{acr:ntp} particularly dangerous due to its prevalence and high amplification rate.
Furthermore, he notes that DNSSEC can exacerbate the problem by its addition of large signatures to message payloads.

\textcite{DBLP:conf/uss/KuhrerHRH14} build further upon this census; they find significant overlap between servers who expose different vulnerable services, connect these services to \gls{acr:os} fingerprints, and use \gls{acr:dns} proxies to enumerate the \glspl{acr:as} who allow \gls{acr:ip} spoofing.
They find that many eligible reflectors tend to lie behind dynamic \gls{acr:ip} addresses and so undergo significant churn (meaning an attacker must often re-scan every few days/weeks).
This is not the case for certain protocols like \gls{acr:ntp}, where the server list remains far more stable.
The authors also explore the amplification potential of all \gls{acr:tcp}-based services---given that well-known protocols like \gls{acr:http} cannot be blocked in most infrastructures, an attacker can abuse retransmissions of the handshake (\texttt{SYN-ACK}) to achieve an amplification factor up to \qty{20}{\times} if the receiver doesn't send \texttt{RST} responses.
%Differing TCP stacks have varying quirks, so the behaviour of the victim and all reflectors can be hard to predict without prior fingerprinting.
%It can be observed that choosing amplifiers of larger geographic distance might increase the amount of \texttt{SYN-ACK} packets in flight before the well-meaning reflector can receive a \texttt{RST}.

\gls{acr:ntp} quickly became the attack vector of choice, according to \Textcite{DBLP:conf/imc/CzyzKGPBK14}.
They find that most vulnerable amplifiers are \emph{end-hosts}, typically offering \qty{4}{\times} amplification.
At the time of publication, they observed that \gls{acr:ntp} amplification attacks had risen in volume by $\sim$\qty{1000}{\times}, though were slowly declining; \qty{85}{\percent} of attacks over \qty{100}{\giga\bit\per\second} relied upon \gls{acr:ntp} reflection.
The decrease, they posit, stems mostly from vulnerable servers being patched in response to recent bulletins making the risk clear to server operators.
%How are these patched servers distributed?
They observe that, after the patch period, many of the remaining vulnerable servers are sparsely distributed (rather, the patched servers are clustered under \gls{acr:ip} blocks).
This is in line with the (un)cleanliness hypothesis put forth by \textcite{DBLP:conf/imc/CollinsSFJWSK07}.
Of greatest concern was the presence of `mega-amplifiers' offering \qtyrange{e3}{e9}{\times} amplification due to the presence of network loops.
%Over the duration of their study, they observed a \SI{92}{\percent} reduction in abusable IPs, though the uncleanliness observation recurs as the reduction is smaller when considering /24 prefixes (\SI{72}{\percent}).
%Regardless, this drop is \emph{far} more significant than any seen in the availability of Open DNS resolvers.
%A large part of the remaining vulnerable machines are identified as end hosts, implying that quick fixes are unlikely.
%They make it clear that it is hard to reason about who the attackers are (bots, organisers or botmasters), and for what reasons they launch attacks (although ancillary evidence suggests that a remarkably common motivation me be rivalry through, e.g., games).

%\Textcite{DBLP:conf/imc/KuhrerHBRH15} investigate the landscape of \emph{open recursive DNS resolvers}, one of the major enabling factors for DNS amplification attacks.
%Specifically, a DNS server is said to be \emph{open} if it does not filter requests by source IP address.
%The existence of such servers is, they claim, paradoxical: rare is the need to publicly expose recursive DNS resolution when the servers should operate in a well-structured (hierarchical) manner.
%By scanning across all IPv4 addresses (according to the methodology of \textcite{DBLP:conf/uss/DurumericWH13}), they discover a downward trend in abusable servers (from \num{26.8e6} to \num{17.8e6} over the year) due to blocked requests/DNS filtering/shutdown/IP churn.
%As it turns out, many of these DNS servers run old and vulnerable software, and are very highly represented (\SI{67}{\percent}) by consumer routers linked to dynamic IPs.
%Curiously, cache snooping reveals that \SI{61}{\percent} of all open DNS resolvers see active use---many of these resolvers are legit (\SIrange{85}{92}{\percent}), with some even filtering out malicious domains.
%The illegitimate set corresponded to censorship in Iran and China, and to malicious redirection to snooping proxies or outright malware.

\Textcite{DBLP:conf/imc/KuhrerHBRH15} investigate the landscape of \emph{open recursive DNS resolvers}, one of the major enabling factors for \gls{acr:dns} amplification attacks.
Many of these \gls{acr:dns} servers run old and vulnerable software, and are very highly represented (\qty{67}{\percent}) by consumer routers linked to dynamic \glspl{acr:ip}.
%Curiously, cache snooping reveals that \SI{61}{\percent} of all open DNS resolvers see active use---many of these resolvers are legit (\SIrange{85}{92}{\percent}), with some even filtering out malicious domains.
%The illegitimate set corresponded to censorship in Iran and China, and to malicious redirection to snooping proxies or outright malware.

As of \citeyear{DBLP:conf/imc/JonkerKKRSD17}, the distribution of amplification attacks over \gls{acr:udp} protocols was observed to roughly have the pattern \gls{acr:ntp}$>$\gls{acr:dns}$>$CharGen.
This is in spite of the evidence put forth by \textcite{DBLP:conf/imc/CzyzKGPBK14}, which suggested a decline of \gls{acr:ntp}-based amplification attacks.
%Perhaps the effort to patch up many servers simply hit a (metaphorical) wall of operators who actually cared, thus leaving many viable NTP amplifiers in the wild?

%?? Any observations from \textcite{DBLP:conf/raid/KramerKMNKYR15}?

It must be reiterated that new amplification \gls{acr:ddos} vectors arise due to software bugs and misconfigurations even today.
\emph{TsuNAME}~\parencite{Moura21a} is a recent example, where the presence of recursive \gls{acr:dns} dependencies causes traffic amplification toward authoritative name servers.
While this cannot be directed to an arbitrary target \emph{per se}, this presents another vulnerability in critical infrastructure that administrators must be aware of.

%?? Mirai used as DDoS vector \cite{DBLP:conf/uss/AntonakakisABBB17}.

\paragraph{Link-flooding attacks}
\glsxtrfullpl{acr:lfa} or \emph{transit-link} attacks are another volumetric \gls{acr:ddos} vector which has come to light~\parencite{DBLP:conf/esorics/StuderP09,DBLP:conf/sp/KangLG13}.
In contrast with typical direct and reflection-based attacks, malicious actors here do not forward traffic directly to their intended victim.
Instead, they use their bandwidth to communicate with as many legitimate or dummy servers as they can such that the outbound traffic of all attacking clients aggregates at a common point in the Internet.
This exhausts the resources of a target \gls{acr:as} or set of bottleneck links needed to reach their intended victim, and traffic appears for all intents and purposes as a completely uncorrelated set of source and destination pairs.
Since the traffic only ever aggregates in, e.g., \gls{acr:isp} networks, target servers never see any attack traffic themselves.
The need for many source nodes means that attackers practically require botnets for \glspl{acr:lfa} to be feasible~\parencite{DBLP:conf/sp/SmithS18}; however, Internet-of-Things devices and other insecure machines are often recruited for this purpose via malware like \emph{Mirai}~\parencite{DBLP:conf/uss/AntonakakisABBB17}.

%do some stuff according to these guys \cite{DBLP:conf/sp/SmithS18} (who defend against it), and these guys (who predicted it)---Crossfire \cite{DBLP:conf/sp/KangLG13}, Coremelt \cite{DBLP:conf/esorics/StuderP09}.
%?? botnet traffic. ?? Mirai used as DDoS vector \cite{DBLP:conf/uss/AntonakakisABBB17}.

%\paragraph{Characteristics}
%
%?? Botnet C\&C communication \cite{DBLP:conf/sac/ZandVYK14}? NOT READ
%
%?? \SI{69}{\percent} of targets are web servers \cite{DBLP:conf/imc/JonkerKKRSD17}.
%
%DDoS attacks evolve over timescales of seconds to months.
%\Textcite{DBLP:conf/spw/KangGS16} investigate this, and consider the implications and considerations necessary to deal with such occurrences.
%Why might attackers desire this?
%The authors posit that a diverse attack portfolio makes for more effective attacks, so long as there is \emph{variation}---a single suite or pattern of evolution makes defence (and discovery of the attacker) far simpler.
%We see such evolution in:
%\begin{description}
%	\item[Volume and Capabilities] Peak \SI{300}{\gibi\bit\per\second}$\rightarrow$\SI{600}{\gibi\bit\per\second} over the last 4 years from date of publication.
%	\item[Attack targets] E.g., Spamhaus---attackers moved from targeting endpoint servers to targeting the routers in \emph{internet exchange points} (IXPs) once the former was detected.
%	\item[Strategy] E.g., semantic attacks $\rightarrow$ volumetric (TCP \texttt{SYN} flood) $\rightarrow$ volumetric (NTP amplification) $\rightarrow$ LFA.
%\end{description}
%They find that evolution in capabilities occurs over longer timescales, as these typically require resource acquisition (knowledge, bots, etc.).
%\emph{Strategies}, however, are easily poised to evolve over short time horizons, typically ``[adapting] to the target's (observed) defensive posture''.
%This behaviour was observed in both the cases of SpamHaus and ProtonMail.
%In light of this, they suggest a two-tier approach to defence.
%To thwart rational adversaries, they suggest the use of \emph{deterrents}---mechanisms located at e.g., a single network point, which can detect attacks and thus increase the cost of maintenance.
%Most defences fit this description.
%To handle cost-insensitive attackers, they suggest collaborative approaches (such as SENSS \cite{DBLP:conf/acsac/RamanathanMYZ18}).
%Unfortunately, the work makes little attempt to describe or study the patterns of short-term evolution which might be expected in a real-world attack.
%
%?? I think we need some other sources to reason about things from a game theoretic perspective---it seems to me that evolution is the name of the game \cite{DBLP:conf/atal/SinhaKT16} (not read this, but seemed relevant at the time).
%
%\paragraph{Amplification}
%\textcite{DBLP:conf/ndss/Rossow14}.
%?? Inbound/outbound traffic ratios at victim (above a certain bw thres).
%?? At the amp, scan activity in surrounding darknets can be an indicator.
%?? At the amp, similar ratio analysis (scaled to account for benign activity and real clients who require high bandwidth: lower amp, higher bw).
%
%\textcite{DBLP:conf/imc/CzyzKGPBK14} observe some further hallmarks specific to NTP attacks.
%The \texttt{monlist} command principally used as the basis for such attacks can often reveal the list of recent targets after the fact, offering external investigators a means to determine which open NTP servers see active use as amplifiers (and their unwilling victims).
%An interesting observation stemming from these records is that the sets of amplifiers and victims are both highly clustered across ASes---individually, that is.
%Furthermore, it is observed that attackers choose a selection of target ports on the victim machine (in order of popularity): HTTP, NTP, SSH, gaming services and DNS.
%Noting that many of thes services run on \emph{TCP}, it seems likely that attackers are hoping for firewalls to blindly allow through both TCP and UDP on these ports.
%Their randomised scanning techniques imply that the culprits behind similar scans could be detected via network telescopes (although it is unclear whether this would reveal the bots, the organiser or the botmaster).
%
%As an aside, the NTP \texttt{monlist} command fragments its $\sim$\SI{50}{\kibi\byte} payload into \SI{482}{\byte} chunks.


\subsection{Contributing factors in the detection problem}\label{sec:ddos-contributing-factors}
\paragraph{Variation in normality}
Benign traffic is in no way `normal', and is often composed of a variety of heterogeneous traffic classes acting in different ways.
Protocol families respond differently to both administrator actions \emph{and} the presence of attack traffic; mainly, this difference is seen between congestion-aware and -unaware flows.
At a high level, congestion-aware traffic tends to scale its rate up to its maximum fair share and scales down in response to congestion signals such as packet losses (e.g., \gls{acr:tcp}), while congestion unaware traffic ignores these requirements (e.g., most \gls{acr:udp} flows).\sidenote{This distinction is not quite as simple as `\gls{acr:tcp} \& \gls{acr:udp}'. Due to middlebox-driven Internet ossification, \emph{QUIC}~\parencite{DBLP:conf/sigcomm/LangleyRWVKZYKS17} and \emph{\gls{acr:sctp}}~\parencite{rfc4960} are carried over \gls{acr:udp}. Respectively, they are and can be congestion-aware. BitTorrent's \emph{\textmu{}TP}~\parencite{DBLP:conf/icccn/RossiTVM10} builds on \gls{acr:udp} to offer a lower-latency congestion-aware transport. Finally, adversarial replayed \gls{acr:tcp} traffic (e.g., \texttt{SYN} floods) is of course congestion-unaware.}
Consider probabilistic packet drop at a rate $p\in\left(0,1\right]$---pushback~\parencite{DBLP:journals/ccr/MahajanBFIPS02a}.
Loss-ignoring and \gls{acr:cbr} traffic's send rate will scale in proportion to $1 - p$.
\gls{acr:tcp} Reno and the like exhibit greater falloff proportional to $1/\sqrt{p}$ courtesy of the Mathis equation~\parencite{DBLP:journals/ccr/MathisSMO97}, with a kinder $1/p^{0.75}$ for \gls{acr:tcp} Cubic~\parencite{rfc8312}, inflicting greater collateral damage than expected on misclassified but legitimate flows.
Even then, congestion-aware traffic's precise response depends on \gls{acr:cca} design, protocol implementation details, and the nature of the carried traffic (e.g., bulk transfer vs.~\glspl{acr:rpc}).

Attack traffic may well share a feature with a non-dominant family of protocols, at which point basing a defence on that mechanism will result in harming or blocking that legitimate traffic---\emph{collateral damage}.
For instance, \gls{acr:cbr} traffic such as \gls{acr:voip} flows are unlikely to respond in a meaningful way to a change in their bandwidth allocation, short of recording and reporting packet loss statistics.
In contrast, most congestion-aware flows (including \gls{acr:lfa} sources) will respond to bandwidth expansion and contraction, with \glspl{acr:lfa} having little to no response compared to legitimate traffic~\parencite{DBLP:conf/ndss/KangGS16}.

Finally, the exact proportions of this heterogeneous mixture vary over time and between \gls{acr:as} classes.
%?? it depends on ?? environment
%?? Heterogeneirty of traffic classes
%?? Classes respond in different ways to different actions.
Consider a point estimate of sorts obtained by analysing the 2018 \gls{acr:caida} traces~\parencite{caida-2018-passive}, shown in \cref{adx:caida-traffic}.
On this tap of \gls{acr:isp} traffic, congestion-aware traffic makes up at least \qtyrange{73}{82}{\percent} of packets; varying over time and the link's direction.
The corollary is that congestion-unaware traffic makes up at most \qtyrange{27}{18}{\percent}---a significant fraction of collateral damage, if we are careless around our defence and detection model in the above example.
%?? Describe all my CAIDA analysis here
%?? analysis of CAIDA datasets~\parencite{caida-2018-passive}
%?? congestion-aware traffic makes up at least \qtyrange{73}{82}{\percent} of packets
%?? Also talk about QUIC's prevalence here
The first figure includes some peak \qty{3.26}{\percent} of QUIC traffic in the \emph{Sao Paulo to New York} direction.
Variability extends also to the behaviour of flows \emph{within} a protocol.
This presents in some cases as long-tailed distribution between more numerous, shorter \emph{mice} flows and longer \emph{elephant} flows~\parencite{DBLP:journals/ccr/PanBPS03}.
A consequence is that punishing actions can have a greater relative impact on some flow classes over others (in this case, packet losses would have the greatest impact on mice \glspl{acr:fct}).

%Our mode of action means that each agent is in control of pushback \cite{DBLP:journals/ccr/MahajanBFIPS02a}, and so carries a risk of introducing collateral damage into the network.
%This is particularly severe when handling TCP traffic: the Mathis equation \cite{DBLP:journals/ccr/MathisSMO97} states that TCP bandwidth is proportional to $1/\sqrt{p}$ (noting that $p$ is nonzero in any real network) while constant bitrate (CBR) UDP traffic is proportional to $1 - p$.
%%It's worth noting that there are various ways that this could be implemented, and that the application of \emph{programmable data planes} to this end are suggested as future work.
%This weakness is still present in modern TCP flavours, such as TCP Cubic which in turn has bandwidth proportional to $1/p^{0.75}$ \cite{rfc8312}.

\paragraph{Evolution}
Just as new attacks and attack vectors arise over time, so too does the rest of the network evolve.
New protocols such as \emph{QUIC}~\parencite{DBLP:conf/sigcomm/LangleyRWVKZYKS17} come into use in the Internet at large, and can achieve near-instantaneous widespread adoption via the backing of hypergiant network operators.
New \glspl{acr:cca} such as \emph{BBR}~\parencite{DBLP:journals/queue/CardwellCGYJ16} are deployed to improve flow bandwidth utilisation, but lead to observable changes in flow-level behaviour.
%?? Why? Protos, traffic kinds (e.g. web video)
%?? Discussion of evolution of traffic: what's come before, what's coming next.
At the aggregate level, heavy hitter flows have seen noteworthy increases in duration and rates over a 13-year time horizon, as has the mice-elephant balance~\parencite{DBLP:conf/anrw/BauerJHBC21}.
%?? Look for older in my old notes, but recent cite here~\parencite{DBLP:conf/anrw/BauerJHBC21}.
Detection and mitigation solutions must be aware of these eventualities to protect legitimate traffic over longer timescales.

%?? Evolution of protocols.
%?? QUIC~\parencite{DBLP:conf/sigcomm/LangleyRWVKZYKS17}
%
%?? QUIC carries \gls{acr:http} traffic, mostly...

%?? Can (and should probably) discuss different traffic classes here: congestion-aware, -unaware...

%?? Implications? Past works pushback collateral bad.
%
%?? Note, explain that this is NOT just TCP vs UDP due to existence of SCTP over UDP (See: DTLS in WebRTC), QUIC over UDP, ...

%?? Explain `UDP makes up a sizeable proportion of network traffic'

%?? Variability within protocols.

\subsection{Defences}
According to \textcite{DBLP:conf/imc/JonkerKKRSD17}, the most-used techniques in deployment are \emph{DDoS Protection Services}.
%\sidenote{?? Para before this. Mention one or two of the old-school ways brought up in the DDoS section like AIMD, Pushback...}
While typically proprietary in nature, we see a split between \emph{cloud services}, \emph{in-line systems} (i.e., middleboxes) and hybrids thereof.
Cloud services (traffic scrubbers) are known to be most appropriate for handling volumetric attacks and are externally hosted, analysing and filtering out malicious traffic by having services redirect all inbound communication for processing.
The act of redirection is often made cheaper and feasible by the use of selective \gls{acr:bgp} advertisement or \gls{acr:dns} modification, aided by reverse proxy or \emph{generic routing encapsulation}.
Amongst these, \gls{acr:bgp}-based diversion is most effective where many hosts must be protected, and \gls{acr:dns} works most reliably for single-host installations.
In-line systems, hosted within a service's domain of control, are most useful for handling semantic attacks as these often admit \emph{attack signatures} (since they must exploit a particular bug in the server).
Similarly, such attacks tend not to exhibit long-term characteristics that cloud scrubbers might use to aid detection, as many of these attacks present themselves as a single packet.
%These authors further find that being attacked does not necessarily increase the likelihood of moving to a DPS---what is an effective indicator is the \emph{strength} of attack targeting a particular service.
%To explain, around a fifth of targets already had a DPS prior to an attack, and only \SI{4}{\percent} of victims without a DPS migrate to one.

\gls{acr:ddn} solutions to \gls{acr:ddos} attacks have been examined through the literature, such as \Textcite{DBLP:conf/lcn/BragaMP10}, \emph{Marl}~\parencite{DBLP:conf/iaai/MalialisK13,DBLP:journals/eaai/MalialisK15}, and \emph{Athena}~\parencite{DBLP:conf/dsn/LeeKSPY17}.
\Cref{sec:ddn-uses-security} explains these, alongside their drawbacks and experimental shortcomings, in detail.
Marl in particular has design flaws which are placed under great scrutiny, motivating the improved \gls{acr:rl} work I develop in the remainder of this chapter.
%\Textcite{DBLP:conf/lcn/BragaMP10} have examined the detection of ongoing (flooding-based) DDoS attacks through \emph{self-organising maps}, making use of SDN to gather statistics effectively.
%Many of their features aren't overly relevant, as their focus is not active defence or discovering \emph{which} hosts are contributing to an attack.

\textcite{DBLP:conf/ndss/Rossow14}'s suggestions are mostly prophylactic.
At the \gls{acr:as}-level, \gls{acr:ip} spoofing by internal clients must be prevented.
Protocol designs should be hardened with session handling \emph{\`{a} la} QUIC or Datagram \gls{acr:tls} at the expense of latency, enforcing greater symmetry of request and response sizes, and rate limiting the frequency of per-client responses.
%?? ISP-level---packet filtering on port, len, substring.

Honeypots such as \emph{AmpPot}~\parencite{DBLP:conf/raid/KramerKMNKYR15} can play a key role in the detection and mitigation of volumetric attacks.
Fake amplifier services hosted by legitimate authorities, which appear to be useful amplifier nodes to malicious actors, may be included in the set of reflector nodes when attacks are launched.
As a result, infrastructure providers can receive early notification of attack targets and the protocols which must be blackholed.

\emph{SPIFFY}~\parencite{DBLP:conf/ndss/KangGS16} aims to remedy \glspl{acr:lfa} by observing how flows from each source respond to a sudden increase in available bandwidth.
\Citeauthor{DBLP:conf/ndss/KangGS16} realise that bots participating in an attack are often unable to match this bandwidth expansion due to having already saturated the capacity of their outbound links, while legitimate flows typically speed up to match the new fair-share rate.
%Attackers must either be detected or reduce the throughput of each bot, increasing the cost of launching an attack.
Due to the class of attacks it is designed to defend against, SPIFFY is intended to be deployed within \gls{acr:isp} networks.
However, they find that computing per-flow routes to offer this expansion is expensive on real networks (\qty{14}{\second} in the Cogent topology), and achieve only low expansion factors which require more rounds of filtering.
Finally, by definition their assumptions cannot extend to \gls{acr:cbr} traffic (e.g., \gls{acr:udp} \gls{acr:voip} traffic), which as we know from \cref{sec:ddos-contributing-factors,adx:caida-traffic} makes up a sizeable proportion of network traffic.
Only congestion-aware traffic will correctly alter its behaviour under this action and response model.

%\emph{Athena}~\parencite{DBLP:conf/dsn/LeeKSPY17} is a more generalised SDN framework for intrusion detection, but has shown the use of a \emph{k-nearest neighbours} classifier to detect individual attack flows.
%Although heavyweight (and proven to be effective compared with \textcite{DBLP:conf/lcn/BragaMP10}), their comparison against SPIFFY lacks the quantitative evidence required to understand how the system compares.

\Textcite{DBLP:conf/sp/SmithS18} present techniques based on \gls{acr:as}-level routing to tackle both transit-link and flooding-based attacks.
This view is taken due to the perceived cost of per-stream classification and inherent sensitivity to adversarial examples or crafted input.
The approach is creative, relying upon \gls{acr:bgp} \emph{fraudulent route reverse poisoning} to preserve traffic to a target \gls{acr:as}, but unlike SPIFFY the approach doesn't actually \emph{remove} the congestion.
Because of this, traditional flooding-based attacks aren't fully alleviated.

\emph{SENSS}~\parencite{DBLP:conf/acsac/RamanathanMYZ18} aims to help hosts and \emph{endpoint-servers} communicate upstream with \glspl{acr:isp}.
The rationale is that although \gls{acr:ddos} traffic can be filtered at any point along its path, it will impact less of the network if it is filtered \emph{close to its source}---this observation holds true in all attack classes (direct, reflection, \gls{acr:lfa}), which exhibit a tree-like pattern of traffic.
This information currently propagates through human channels, eventually leading to traffic blackholing being performed by key \glspl{acr:as}.
The core idea is that the \emph{victims} should be given responsibility for intelligence and decision-making, who pass on their requests to \glspl{acr:isp} (alongside ample payment).
They are able to show that this approach functions for multiple algorithms---including using \gls{acr:nat} for true outbound requests as a mechanism for reflector filtering close to the source, similar techniques to others to ``route around'' the congestion added by \glspl{acr:lfa}, and location-based filtering for signature-free attacks.
%The need for payment does seem odd at the outset, but it becomes clear that this is a necessary mechanism to enable ``remote networks to collaborate on demand, without prior trust''.
%The mitigation strategies they propose do hold water, and strike me as interesting---using NAT for true outbound requests as a mechanism for reflector filtering close-to-source, similar techniques to others to ``route around'' the congestion added by LFAs, and location-based filtering for signature-free attacks.
%What I'm concerned about is the degree of collaboration required; it seems likely to me that there may exist e.g., amplifiers who are difficult to block in such a manner due to non-cooperative ASes on their path, with geography and network uncleanliness as contributing factors...
%Their evaluation is convincing---a mixture of a testbed experiment over a small-scale environment (Iperf + ``custom UDP flood'') and an AS-level simulation recreating the DDoS attack on Dyn (gravity model \cite{DBLP:journals/ccr/Roughan05}, cogent topology from Topology Zoo \cite{topology-zoo}).

\Textcite{DBLP:journals/tnsm/SimpsonSMJPH18} propose the \emph{Antidose} collaborative solution.
\glspl{acr:as} ask one another to install allow- and block-list filters to represent the interests of their own transit traffic while disallowing known-bad sources and \glspl{acr:as}.
Hosts and agents must perform proof-of-work challenges attached to flow cookies to become eligible for allowlisting (which is verifiable by any other node)---however, this requires some degree of re-architecting the network stacks of all hosts.

Some collaborative solutions appear to hinge on the condition that \gls{acr:http} and \gls{acr:tcp} sessions can be reliably held over the saturation zone between high-priority endpoints.
Alternative channels may be possible through elected proxies or \gls{acr:udp}-based mechanisms like \gls{acr:dots}~\parencite{ietf-dots-use-cases-17}.
\gls{acr:dots} provides an architecture for network operators to enumerate, discover, and communicate with \gls{acr:ddos} mitigation services, with who they can exchange telemetry information and explicit mitigation requests.

%\section{Background and Threat Model}
%?? Introduce RL, related definitions etc.

%\subsection{Distributed Denial of Service}
%%?? DDoS attack variants (leading into characteristics, supporting features). Amplification (UDP \cite{DBLP:conf/ndss/Rossow14}, TCP \cite{DBLP:conf/uss/KuhrerHRH14}), Transit-link (Crossfire \cite{DBLP:conf/sp/KangLG13}, Coremelt \cite{DBLP:conf/esorics/StuderP09}). Mirai botnet's involvement \cite{DBLP:conf/uss/AntonakakisABBB17}.
%
%%?? Explain amplification attack, maybe transit-link?
%
%\emph{Distributed denial of service} (DDoS) attacks are concentrated efforts by many hosts to reduce the availability of a service, typically to inflict financial harm or as an act of vandalism.
%Attackers achieve this by either exploiting peculiarities of operating system or application behaviour in \emph{semantic attacks} (e.g., \emph{SYN flooding attacks}), or overwhelming their target through sheer volume of requests or inbound packets (\emph{volume-based attacks}) \cite{DBLP:conf/imc/JonkerKKRSD17}.
%Hosts often participate unwillingly, typically having been recruited into a \emph{botnet} by malware infection to be orchestrated from elsewhere \cite{DBLP:conf/uss/AntonakakisABBB17}.
%
%Although there are variations of each class of attack, flooding attacks are the most relevant to our work.
%\emph{Amplification attacks} exploit services who eagerly send large replies in response to small requests, where UDP-based services like DNS and NTP are most exploitable \cite{DBLP:conf/ndss/Rossow14, DBLP:conf/uss/KuhrerHRH14}.
%Malicious hosts send many small requests, spoofed to appear as though they originated from the victim, causing many large replies to be sent to the intended target---significantly increasing a botnet's throughput while masking the identity of each participant.
%\emph{Transit-link/link-flooding attacks} have been the subject of recent attention, wherein malicious traffic is forwarded across core links needed to reach a target (but not to the target itself) \cite{DBLP:conf/sp/KangLG13, DBLP:conf/esorics/StuderP09}.

% \subsection{Reinforcement Learning}\label{sec:reinforcement-learning}
% \emph{Reinforcement learning} (RL) is a variant of machine learning principally concerned with training an agent to choose an optimal sequence of actions in pursuit of a given task \cite{RL2E}.
% We assume the agent has a certain amount of knowledge whenever a decision must be made: at any point in time $t$, it knows which \emph{state} it is in ($S_t \in \mathcal{S}$), the set of \emph{actions} which are available to it ($\operatorname{A}(S_t) \subseteq \mathcal{A}$) and a numeric \emph{reward} obtained from the last action chosen ($R_t \in \mathbb{R}, A_{t-1} \in \operatorname{A}(S_{t-1})$).
% This model of system interaction is a \emph{Markov Decision Process} (MDP).
% RL methods combine this information with a current \emph{policy} $\pi$ to determine which action should be taken: such a choice need not be optimal if an agent needs to further explore some region of the state space.
% The policy is refined by updating value estimates for state-action pairs or via policy gradient methods, meaning that RL-based approaches learn adaptively and online if reward functions are available in the environment they are deployed in.
% In practice, this means that agents are able to automatically adapt to evolving problems without operator intervention or a new, custom-built training corpus.

% From any point in a sequence of decisions, we may describe the sum of rewards yet to come as the \emph{discounted return}, $G_t = R_{t+1} + \gamma R_{t+2} + \gamma^2 R_{t+3} + \ldots$, choosing the discount factor $\gamma \in [0,1)$ to determine how crucial future rewards are vis-\`{a}-vis the current state.
% Formally, an agent's goal is to choose actions which maximise the \emph{expected discounted return} $\operatorname{\mathbb{E}_{\pi}}[G_t]$.

% %?? Include some details of function approximation in the formalisation? I.e. tile coding, stability and convergence guarantees...

% There is immense variation in \emph{how} policies and/or values may be learned, reliant upon the learning environment, problem and required convergence guarantees.
% In particular, we focus on methods which choose actions according to their value estimates from the current state: let $\operatorname{q}(s, a) \in \mathbb{R}$ be the estimate of action $a$'s value if it were to be taken in state $s$.
% %?? Revisit the maths here, d, dim A, s...
% Exact (tabular) representations require that we store a value estimate for each action in every state---if state is real-valued or high-dimensional, then computation and storage quickly become infeasible.
% To cope with a continuous state and/or action space, one valuable technique is to employ linear function approximation backed by \emph{tile coding} \cite[pp.\ \numrange{217}{221}]{RL2E}.

% \begin{figure}
% 	\centering
% 	\begin{subfigure}{0.4\linewidth}
% 		\resizebox{\linewidth}{!}{
% 		\begin{tikzpicture}
% 		\node at (0,0.3){$s = \begin{pmatrix}
% 			0.7 \\
% 			0.3
% 			\end{pmatrix}$};
% 		\node at (0,-1) {$\bm{x}(s,\cdot) = \begin{Bmatrix}
% 			T_{1,9}, \\
% 			T_{2,5}, \\
% 			T_{\mathit{bias}}
% 			\end{Bmatrix}$};
		
% 		\node at (2.5,-0.5) {
% 			\begin{tikzpicture}
% 			\draw[step=0.5cm,color=uofgcobalt,opacity=0.7,shift={(0,0)},label=above:{Tiling 0}] (-0.5,-0.5) grid (1,1);
% 			\fill[uofgcobalt,opacity=0.5] (0.5,-0.5) rectangle (1,0);
% 			\node[color=uofgcobalt] at (0,1.1) {\footnotesize Tiling 1};
			
% 			\draw[step=0.5cm,color=uofgpumpkin,opacity=0.9,shift={(0.25,-0.25)},label=above:{Tiling 1}] (-0.5,-0.5) grid (1,1);
% 			\fill[uofgpumpkin,opacity=0.5,shift={(0.25,-0.25)}] (0,0) rectangle (0.5,0.5);
% 			\node[color=uofgpumpkin!50!uofgrust] at (0.25,-0.95) {\footnotesize Tiling 2};
			
% 			\node[circle, black, draw,
% 			fill, radius=0.5pt, inner sep=0pt,minimum size=1.5pt, label=above:{$s$}] at (0.625,-0.125) {};
% %			\filldraw (0.625,-0.125) circle[radius=1.5pt,label=above:{$s$}];

% 			\draw[->] (-0.25,-0.5)--(-0.25,0.85);
% 			\draw[->] (-0.25,-0.5)--(1.1,-0.5);
			
% 			\node at (1,-0.7) {\footnotesize 1};
% 			\node at (-0.4,0.75) {\footnotesize 1};
% 			\node at (-0.35,-0.6) {\footnotesize 0};
% 			\end{tikzpicture}
% 		};
		
% 		\end{tikzpicture}
% 		}
% 		\caption{Tile Coding\label{fig:tilecode-code}}
% 	\end{subfigure}%
% 	\begin{subfigure}{0.4\linewidth}
% 		\resizebox{\linewidth}{!}{
% 		\begin{tikzpicture}
% 		% Top half
% 		\def\topacs{
% 			{-0.3,-0.5,-0.1,0.8},
% 			{0.1,0.1,-0.2,0.3},
% 			{0.3,0.4,0.2,-0.4}%
% 		}
	
% 		\foreach \line [count=\y] in \topacs {
% 			\foreach \pix [count=\x] in \line {
% 				\ifthenelse{\lengthtest{\pix pt < 0 pt}}{
% 					\pgfmathtruncatemacro\lambda{(\pix+1)*100}
% 					\draw[shift={(-0.7,0)},fill=midac!\lambda!lowac] (0.7*\x,-0.35*\y) rectangle +(0.7,0.35);
% 				}{
% 					\pgfmathtruncatemacro\lambda{\pix*100}
% 					\draw[shift={(-0.7,0)},fill=highac!\lambda!midac] (0.7*\x,-0.35*\y) rectangle +(0.7,0.35);
% 				}
% 				\node[shift={(-0.35,0.175)}] at (0.7*\x,-0.35*\y) {\footnotesize \pix};
% 			}
% 		}
	
% 		\draw[xstep=0.7cm,ystep=0.35cm,shift={(0,0)}] (0,-1.06) grid (2.8,0);
% 		\node[label=left:{$T_{1,9}$},shift={(0,-0.125)}] at (0,0) {}; 
% 		\node[label=left:{$T_{2,5}$},shift={(0,-0.125)}] at (0,-0.375) {}; 
% 		\node[label=left:{$T_{\mathit{bias}}$},shift={(0,-0.125)}] at (0,-0.75) {};
		
% 		% bottom half
% 		\def\botacs{
% 			{0.1,0.0,-0.1,0.7}%
% 		}
	
% 		\foreach \line [count=\y] in \botacs {
% 			\foreach \pix [count=\x] in \line {
% 				\ifthenelse{\lengthtest{\pix pt < 0 pt}}{
% 					\pgfmathtruncatemacro\lambda{(\pix+1)*100}
% 					\draw[shift={(-0.7,-1.275)},fill=midac!\lambda!lowac] (0.7*\x,-0.35*\y) rectangle +(0.7,0.35);
% 				}{
% 					\pgfmathtruncatemacro\lambda{\pix*100}
% 					\draw[shift={(-0.7,-1.275)},fill=highac!\lambda!midac] (0.7*\x,-0.35*\y) rectangle +(0.7,0.35);
% 				}
% 				\node[shift={(-0.35,-1.1)}] at (0.7*\x,-0.35*\y) {\footnotesize \pix};
% 			}
% 		}
	
% 		\draw[xstep=0.7cm,ystep=0.35cm,shift={(0,-1.275)}] (0,-0.35) grid (2.8,0);
% 		\node[label=left:{$\mathit{Total}$},shift={(0,-0.175)}] at (0,-1.275) {};
		
% 		\draw [->] (2.45, -1.9) -- (2.45, -1.65);
% 		\end{tikzpicture}
% 		}
% 		\caption{\centering Value Estimation and Action Selection\label{fig:tilecode-select}}
% 	\end{subfigure}
	
% 	\caption{
% 		An example of tile coding for 2-dimensional state and 4 actions, using 2 tilings, 3 tiles per dimension, and a bias tile.
% 		All components of $s$ are clamped to $[0,1]$.
% 		For simplicity, we denote $\bm{x}(s,\cdot)$ as a list of indices and represent the values of all actions at each tile with a vector.
% 		(a) The state $s$ activates the bias tile and exactly one tile in each tiling.
% 		(b) The action values of active tiles are summed to produce the current value estimate for each action available in $s$---for this state, local knowledge ensures that action 4 is chosen by the greedy policy despite typically being a poor choice elsewhere.
% 		\label{fig:tilecode}
% 	}
% 	\vspace{-0.6cm}
% \end{figure}

% Tile coding is a form of feature representation which converts a state-action pair into a sparse boolean feature vector $\operatorname{\mathbf{x}}(s, a)$ by subdividing a $d$-dimensional subset of the space into a number of overlapping grids with an optional bias component.
% Each tile corresponds to an entry of $\operatorname{\mathbf{x}}(s, a)$ which is set to 1 if the state-action pair lies within it.
% \Cref{fig:tilecode-code} demonstrates the process for a 2-dimensional state space, and that the numbers of tilings and tiles per dimension control feature resolution and generalisation.
% Moreover, to capture combinatorial effects or create multi-scale representation we may combine codings by concatenating individual feature vectors.
% We may then approximate an action's value with respect to a policy parameter vector $\wvec{}$, defining some $\acval{s}{a}{\wvec{}} \approx \operatorname{q}(s, a)$:
% \begin{equation}
% %	\begin{gather}
% \acval{s}{a}{\wvec{}} = \wvec{}^{\top} \operatorname{\mathbf{x}}(s, a)
% %	\end{gather}
% \label{eqn:lin-approx}
% \end{equation}
% As each component of $\wvec{}$ is the value estimate of the corresponding tile, learning an effective policy is equivalent to learning $\wvec{}$.
% Given a learning rate $\alpha \in \mathbb{R}$ and initialising $\wvec{0}=\bm{0}$, we may continually update $\wvec{t}$ using the \emph{1-step semi-gradient Sarsa} algorithm \cite[pp.\ \numrange{243}{244}]{RL2E}:
% \begin{subequations}
% 	\begin{gather}
% 	\delta_t = R_{t+1} + \gamma \acval{S_{t+1}}{A_{t+1}}{\wvec{t}} - \acval{S_t}{A_t}{\wvec{t}},\\
% 	\bm{w}_{t+1} = \bm{w}_{t} + \alpha \delta_t \nabla{\acval{S_t}{A_t}{\wvec{t}}},
% 	\end{gather}%
% 	\label{eqn:sg-sarsa}%
% 	where $\delta_t$ is known as the \emph{temporal-difference} (TD) error, and the vector gradient $\nabla$ is taken with respect to $\wvec{}$.
% \end{subequations}

% Computing the approximate value of every available action forms the basis of a policy.
% Actions with maximal value can be chosen each time (the \emph{greedy} policy), we might modify this by taking random actions with probability $\epsilon$ to encourage early exploration (the \emph{$\epsilon$-greedy policy}), or we might use some other mechanism.
% \Cref{fig:tilecode-select} extends the prior working example to show how the value of each action is computed (and which action would be chosen by a greedy policy), combining a global estimate ($T_{\mathit{bias}}$) with knowledge particular to each state.

% \begin{figure}
% 	\centering
% 	\resizebox{0.45\linewidth}{!}{
% 		\begin{tikzpicture}		
% 		% Top half
% 		\def\startacs{
% 			{0.8},
% 			{0.3},
% 			{-0.4}%
% 		}
		
% 		\foreach \line [count=\y] in \startacs {
% 			\foreach \pix [count=\x] in \line {
% 				\ifthenelse{\lengthtest{\pix pt < 0 pt}}{
% 					\pgfmathtruncatemacro\lambda{(\pix+1)*100}
% 					\draw[shift={(-0.7,0)},fill=midac!\lambda!lowac] (0.7*\x,-0.35*\y) rectangle +(0.7,0.35);
% 				}{
% 					\pgfmathtruncatemacro\lambda{\pix*100}
% 					\draw[shift={(-0.7,0)},fill=highac!\lambda!midac] (0.7*\x,-0.35*\y) rectangle +(0.7,0.35);
% 				}
% 				\node[shift={(-0.35,0.175)}] at (0.7*\x,-0.35*\y) {\footnotesize \pix};
% 			}
% 		}
		
% 		\draw[xstep=0.7cm,ystep=0.35cm,shift={(0,0)}] (0,-1.06) grid (0.7,0);
% 		\node[label=left:{$T_{1,9}$},shift={(0,-0.125)}] at (0,0) {}; 
% 		\node[label=left:{$T_{2,5}$},shift={(0,-0.125)}] at (0,-0.375) {}; 
% 		\node[label=left:{$T_{\mathit{bias}}$},shift={(0,-0.125)}] at (0,-0.75) {};
% 		\node[label=below:{\footnotesize Action 4},shift={(0.35,-0.125)}] at (0,-0.75) {};
% 		\node[label=above:{\footnotesize $\wvec{t}$},shift={(0.35,-0.125)}] at (0,0) {};
		
% 		\def\finalacs{
% 			{0.7},
% 			{0.2},
% 			{-0.5}%
% 		}
	
% 		\foreach \line [count=\y] in \finalacs {
% 			\foreach \pix [count=\x] in \line {
% 				\ifthenelse{\lengthtest{\pix pt < 0 pt}}{
% 					\pgfmathtruncatemacro\lambda{(\pix+1)*100}
% 					\draw[shift={(2-0.7,0)},fill=midac!\lambda!lowac] (0.7*\x,-0.35*\y) rectangle +(0.7,0.35);
% 				}{
% 					\pgfmathtruncatemacro\lambda{\pix*100}
% 					\draw[shift={(2-0.7,0)},fill=highac!\lambda!midac] (0.7*\x,-0.35*\y) rectangle +(0.7,0.35);
% 				}
% 				\node[shift={(2-0.35,0.175)}] at (0.7*\x,-0.35*\y) {\footnotesize \pix};
% 			}
% 		}
		
% 		\draw[xstep=0.7cm,ystep=0.35cm,shift={(2,0)}] (0,-1.06) grid +(0.7,1.06);
% 		\node[label=below:{\footnotesize Action 4},shift={(2.35,-0.125)}] at (0,-0.75) {};
% 		\node[label=above:{\footnotesize $\wvec{t+1}$},shift={(0.35,-0.125)}] at (2,0) {};
		
% 		\draw[->] (0.9,-0.5) -- node[above] {$+ \alpha \delta_t$} (1.8,-0.5);
% 		\end{tikzpicture}
% 	}
% \vspace{-0.25cm}
% \caption{
% 	The update step for \cref{fig:tilecode}, given an observed TD error $\delta_t=-0.2$ (indicating a lower observed reward than the expected long-term value of \num{0.7}) and $\alpha=0.5$.
% 	Action 4's value is thus reduced in the tiles associated with state $s$, but remains the most likely choice; the negative $\delta_t$ may have arisen from noise in the reward signal.
% 	For illustrative purposes, we choose an unrealistically high $\alpha$ (typically, $\alpha\le0.05$ is a more reasonable choice).
% 	\label{fig:tilecode-update}
% }
% \vspace{-0.6cm}
% \end{figure}

% This combination of algorithm and coding strategy is well-optimised, if actions are discrete; this allows a particularly efficient (vectorised) implementation of the policy and update rules by storing a vector of action values for each tile.
% Action values for any state are then obtained by summing the weight vectors from all activated tiles---taking $|\mathcal{A}|(n_{\mathit{tilings}}-1)$ floating point additions per decision.
% Observing that $\nabla{\acval{s}{a}{\wvec{}}} = \operatorname{\mathbf{x}}(s, a)$, further optimisations arise by considering that a tile-coded feature vector is a binary vector of constant Hamming weight (and so is amenable to representation as an array of indices, $s_{\mathit{list}}$).
% This means that we need only perform $n_{\mathit{tilings}} + 2$ additions and \num{2} multiplications per model update:
% \begin{equation}
% \bm{w}_{t+1}[i][\operatorname{index}(A_t)] = \bm{w}_{t}[i][\operatorname{index}(A_t)] + \alpha \delta_t, \forall i \in s_{\mathit{list}}.
% \label{eqn:sg-sarsa-opt}
% \end{equation}
% \Cref{fig:tilecode-update} shows how this applies to our prior example.
% If desired we may define a state space with an arbitrary number of tiles per dimension (higher-resolution, lower generalisation), yet having constant-size state vectors and constant action computation cost ($\mathcal{O}(n_{\mathit{tilings}})$).
% Beyond this, we need not store action values for tiles which have not yet been visited, conserving memory.
% A caveat of tile coding remains, in that the value of $\alpha$ must be reduced according to the number of tilings to prevent divergence at the expense of slower learning ($\alpha \leftarrow \alpha / n_{\mathit{tilings}}$).

%?? Is this \emph{actually} just sarsa? We're using fn approx (of course), but this is fraught with its own difficulties. Is it strictly speaking correct to describe it as Sarsa at this point? It's, at the very least, 1-step semi-gradient Sarsa given that it is clearly on-policy w/ fn approx...

%\subsection{Intrusion Detection}
%Probably want to talk about NIDS/IPS,
%?? Discuss mininet? Networking terms? SDN stuff?

\section{Motivation}\label{sec:ddos-motivation}
%?? What makes RL a suitable method for network anomaly detection, what features are most relevant?
%?? Point I was thinking of: feedback-loop-like model allows monitoring \emph{after} an action is taken to (in theory) allow forgiveness of mistakenly punished flows. This does hinge on taking a flow-by-flow look at the state space, but if we can combine knowledge of current state (duh!), the last action taken (i.e. an indicator of our previous assessment [such as high pdrop $\implies$ bad]) then perhaps a flow which falls off identically to a legit flow can be rescued.
Moving beyond the overt benefits of choosing \gls{acr:rl}-based defences for coping with evolving or ongoing control problems, we believe that there are concrete reasons for their use here.
We have seen that for other domains in particular, misclassification is a serious problem, which can introduce \emph{collateral damage} in the context of \gls{acr:ddos} prevention.
In theory, the feedback-loop-like model allows us to monitor flows \emph{after} an action is taken to allow forgiveness of mistakenly punished flows.
This does rely upon the ability to take a flow-by-flow view of the state space, but if we can combine knowledge of current state with the last applied action, then perhaps a flow which falls off identically to a legitimate flow can be rescued.

%Which features might be best suited to this problem?
%?? Relevant features: aggregate network state (load at various points [this has been done, of course]), flow-specific measurements (upload/download ratio when bandwidth above threshold \cite{DBLP:conf/ndss/Rossow14}, packet inter-arrival times, etc.)
Other studies suggest that there are particularly useful features which make the task of online \gls{acr:ddos} flow identification feasible.
Aggregate network load measures observed at various locations suggest the overall health of a network~\parencite{DBLP:journals/eaai/MalialisK15}, for instance high link occupations but few successful requests reported by a target server might be an indicative feature.
Similarly, the ratio of correspondence between pair flows can suggest asymmetry, and in many cases illegitimacy~\parencite{DBLP:conf/ndss/Rossow14}.
Generic volume-based statistics (counts, counts per duration, average packet sizes) have seen effectiveness in such as \gls{acr:knn} classifiers trained to detect \gls{acr:ddos} attacks in progress~\parencite{DBLP:conf/dsn/LeeKSPY17}.
Most importantly, there is evidence showing behavioural changes in response to bandwidth expansion~\parencite{DBLP:conf/ndss/KangGS16}, suggesting similar artefacts might arise after throttling, packet drop, or other interference.
%?? If we assume amplification attacks, we know it won't be `random' source IPs (since it's mostly-legit servers who think that they're doing a good job by replying)
%?? If we assume amplification attacks, we know it won't be `random' source IPs (since it's mostly-legit servers who think that they're doing a good job by replying)

%\section{A Plan, of Sorts}
%
%\begin{enumerate}
%	\item The main case for contribution in what I have so far:
%	\begin{itemize}
%		\item Past work reliant on unrealistic network models: tcp-like behaviour (and its effects on collateral damage) not captured, disjoint ranges of traffic distribution (no benign heavy-hitters), ISP-like topology.
%		\item I offer more realistic network emulation environment, better treatment of protocol/traffic characteristics.
%	\end{itemize}
%	\item Forthcoming: rethinking state/action spaces to operate at a finer level of granularity. New network model (live tcp back-and-forth), allows us to test collateral damage assumptions in a more realistic manner (and show clear case for moving beyond work of malialis and kudenko)
%\end{enumerate}

\section{Threat model}\label{sec:ddos-threat}
An attacker's goal is to minimise the fair-share bandwidth allocation that a server can give to hosts, and they are expected to act rationally in its pursuit.
Threat actors are external and act intentionally, aren't expected to be 
\glspl{acr:apt}, and likely range from hacktivists to moderately funded adversaries.
We assume that attacks will be volumetric \gls{acr:ddos} attacks with the structure of an \emph{amplification attack}, and that traffic aggregates at the target (unlike in a transit-link attack or \gls{acr:lfa}).
The addresses of the set of unwitting reflector nodes are visible to the target, though any bots taking part in an attack or the machines those bots control are not revealed to the target without communication with \nth{3} party organisations such as upstream \glspl{acr:isp}.
The discovery of any reflector by some defence system does have a cost to the attacker---there is a particularly large (yet finite) supply of viable reflector nodes~\parencite{DBLP:conf/ndss/Rossow14}, but the constraints that each has a large upstream bandwidth and support for high-amplification protocols narrow this pool.

We do not assume that an attacker has white-box access to an agent's policy, or that they will attempt to intelligently modify flow/system state to indirectly control an agent~\parencite{DBLP:conf/eurosp/PapernotMJFCS16, DBLP:conf/eurosp/PapernotMSW18, DBLP:journals/corr/HuangPGDA17, DBLP:conf/sp/Carlini017}---the kinds of evasion attack considered throughout \cref{sec:ddn-security}.
While they may be able to perform some degree of reverse engineering by observing the health of their own legitimate canary flows, ``stealing'' the policy through observation~\parencite{DBLP:conf/uss/TramerZJRR16}, this would require an attacker to indirectly observe the effects of (probabilistic) actions applied to their traffic---in addition to effects imposed by other flows competing for resources.
Moreover, gaining feedback on the fate of attack packets is less feasible with connectionless traffic, and doubly so when it is generated by an amplifier not under the adversary's control.
Investigating whether perturbations applied to flow state would persist in volatile network traffic statistics falls also outside of the scope of this work.
The same observation extends to the possibility of poisoning attacks~\parencite{DBLP:journals/corr/abs-1902-09062}.
These are \gls{acr:apt}-level capabilities, whose exploration presents a rich source for future work.

\section{Per-flow RL agent designs}\label{sec:ddos-mitigation-with-per-flow-reinforcement-learning}
Our main hypothesis is that the best method for advancing past the current shortcomings of RL-based DDoS mitigation is to design agents such that filtering decisions are computed per flow.
However, these alterations must account for computational constraints imposed by the deployment environment---the amount of flows passing over an agent is unbounded.
We describe and justify our approach, our algorithmic improvements, and present two action models, one of which draws on domain knowledge introduced by SPIFFY \cite{DBLP:conf/ndss/KangGS16}.

\subsection{System Design and Assumptions}
A deployment environment is a network with a set of \emph{ingress/egress points} from its domain of control, through which traffic can flow, and a set of protected \emph{destination nodes}.
These nodes may be services, servers, or in the case of Autonomous Systems (ASes) and transit networks, egress points leading to other networks.
{\color{revisiontext}\cbstart Agents are co-located with each egress switch (i.e., $k$ ingress points from other ASes require $k$ agents), all employ the same action model/design}, and control the proportion of upstream packets from each external host to discard.
Each destination node $s$ has a maximum capacity, \cbend $U_s$.

We assume that the deployment environment is a moderately complex software-defined network, because the paradigm offers features which can directly benefit RL agents acting within.
The OpenFlow protocol allows a controller (or other authorised hosts) to install complex actions, forwarding rules and logic into a switch at runtime.
Furthermore, networks of this kind more naturally enable the future use of \emph{network function virtualisation}, a technology which could allow relocation and easy installation of learners (e.g., as examined by \textcite{DBLP:journals/tnsm/JakariaRF19}).
Agents communicate with their co-hosted OpenFlow-enabled switches---running a modified version of \emph{Open vSwitch} (OVS) \cite{open-vswitch}---to install probabilistic packet-drop rules.

Our system design applies to both software-defined and traditional networks of arbitrary shape and size.
Only the ingress/egress nodes from a network need to be OpenFlow-enabled, as it is advantageous to perform filtering as close to a flow's source as possible.
In a traditional network, each agent has exclusive control over its switch's tables.
In a fully software-defined network, these agents require exclusive control over the first table, forwarding legitimate packets to subsequent tables managed by the network's controller.
The main difference is that a traditional network needs this additional hardware, and does not allow an operator to dynamically determine where the ``edge'' of their network lies through vNF relocation.

\subsection{Algorithm}
To make decisions cheaply and at low latency, we use \emph{semi-gradient Sarsa with tile coding} as described in \cref{eqn:sg-sarsa} and \cref{sec:reinforcement-learning}, rather than using neural networks or more complicated function approximators.
Exploration is introduced via $\epsilon$-greedy action selection, linearly annealing $\epsilon$ to 0 over time.
Each agent has its own internal parameter vector $\wvec{}$, and agents do not share their weight vector updates with one another (but may share experience and traces with one another).

Although the choice of a classical RL method likely brings lower theoretical performance, there are significant reasons to favour such methods; these include lower latency decision-making, lower energy usage, reduced model complexity (and training time), the availability of necessary hardware, and simpler decision boundaries.
This aligns with our goal of quick online learning, and faster adaptation to aggregate changes in traffic without introducing dedicated tensor processing hardware to networks.
Simpler decision boundaries reduce the risk of overfitting and unexpected behaviour, and we expect that the simplicity of tile-based policy computation will also greatly aid interpretability of anomalous action choices.

When choosing a learning algorithm, we compared against Q-learning as well as methods based on \emph{eligibility traces} such as Watkins's $\operatorname{Q}(\lambda)$ \cite[pp. 312--314]{RL2E} and $\operatorname{Sarsa}(\lambda)$ \cite[p. 305]{RL2E}.
Preliminary experiments found that Sarsa offered the best performance and behaviour.

\subsubsection{Action rate}
We adapt the algorithm to prioritise rapid response to changes in network state and to visit as many state-to-state transitions as possible for effective learning.
To this end, we allow agents to make many decisions per timestep.
We maintain the last state-action pair associated with each source IP and destination, and calculate any actions for the flows which still exist.
Finally, we update $\wvec{}$ using each available trace and the reward signal from the relevant destination.
As exploration still occurs for each action, this approach reduces $\epsilon$ multiple notches every timestep.
In turn, we increase the annealing window for $\epsilon$ by a factor of \num{2.67} so as to preserve exploration over time, by accounting for the greater volume of decisions being made.

\subsubsection{Per-tile updates}
While the standard formulation of \cref{eqn:sg-sarsa} updates the value of all tiles identically (by a scalar $\alpha \delta_t$), we found it more effective to compute a different temporal difference value \emph{for each tiling}.
While we make use of the sum of all tiles' action value estimates when making decisions, each tiling is updated using only its own contribution, allowing us to set $\alpha$ to a higher value without divergence.
A crucial observation is that value updates to each tile can move by different values in different directions, converging on effective estimates sooner.

\subsubsection{Decision narrowings}
When learning control on the basis of a high-dimensional, tile-coded state space, assignment of credit for each decision is difficult (because all tiles have identical gradient).
To combat this, with probability $\epsilon$ an agent will mark a flow as being governed by a subset of the state space for the next 5 decisions.
Each agent chooses actions on that source/destination pair using one element of local state, the global state, and the bias tile---we include the latter two to strike a balance between and accuracy and correct credit assignment.

\subsection{Feature Space}\label{sec:feature-space}

\begin{figure}
	\centering
	\resizebox{0.65\linewidth}{!}{
		\begin{tikzpicture}[
		texts/.style = {text=black},
		labeltexts/.style = {text=gray},
		treeline/.style = {draw=uofgburgundy},
		treenode/.style = {texts, circle, centered, fill=white, treeline},
		load/.style = {fill=uofgcobalt},
		loadline/.style = {draw=uofgcobalt, line width=0.75mm},
		loadhide/.style = {fill=uofgcobalt!40!white},
		external/.style = {fill=uofgrust},
		externalhide/.style = {fill=uofgrust!40!white},
		hideline/.style = {draw=uofgsandstone!40!white},
		hidenode/.style = {treenode, hideline},
		grow'=right
		]
		\node [treenode, loadhide, label={[texts]above:Agent 1}] (mainagent) {};
		\node [treenode, loadhide, right = of mainagent] (inner0) {};
		\node [treenode, loadhide, above right = 0.2cm and 1cm of inner0] (inner1) {};
		\node [hidenode, below right = 0.2cm and 1cm of inner0] (inner2) {};
		\node [treenode, loadhide, below right = 0.2cm and 1cm of inner1] (inner3) {};
		\node [treenode, loadhide, right = of inner3] (inner4) {};
		\node [treenode, loadhide, above right = 0.2cm and 1cm of inner4, label={[texts]above:$s_0$}] (s0) {};
		\node [hidenode, below right = 0.2cm and 1cm of inner4, label={[labeltexts]above:$s_1$}] (s1) {};
		
		\node [hidenode, above left = 0.2cm and 0.5cm of inner1, label={[labeltexts]above:Agent 2}] (agent2) {};
		\node [hidenode, below left = 0.2cm and 0.5cm of inner2, label={[labeltexts]below:Agent 3}] (agent3) {};
		
		\draw[hideline, -] ($ (mainagent) + (-1,0) $) -- (mainagent);
		\draw[hideline, -] ($ (agent2) + (-1,0) $) -- (agent2);
		\draw[hideline, -] ($ (agent3) + (-1,0) $) -- (agent3);
		\draw[hideline, -] (inner1) -- (agent2);
		\draw[hideline, -] (inner2) -- (agent3);
		
		\draw[loadline, -] (mainagent) -- (inner0);
		\draw[loadline, -] (inner0) -- (inner1);
		\draw[hideline, -] (inner0) -- (inner2);
		\draw[-] (inner1) -- (inner3);
		\draw[hideline, -] (inner2) -- (inner3);
		\draw[loadline, -] (inner3) -- (inner4);
		\draw[loadline, -] (inner4) -- node [texts, above] {$U_{s_0}$} (s0) ;
		\draw[hideline, -] (inner4) -- node [labeltexts, below] {$U_{s_1}$} (s1);
		
		\end{tikzpicture}
	}
\vspace{-0.25cm}
	\caption{
		Global state selection for a flow between an external host and server $s_0$ which passes over Agent 1.
		All nodes in the path taken through the defended network are filled in blue, and all link load measurements which are chosen for action computation are indicated with a thick blue line.
		\label{fig:global-state-path}
	}
\vspace{-0.5cm}
\end{figure}

Our state space combines elements of global state (network link load observations) with per-flow measurements.
Each is tile-coded with 8 tilings and 6 tiles per dimension, using the windows described in \cref{tab:codings}.

%?? How is global state acquired? See \cref{fig:global-state-path}. Why take paths in the way that we do? Mention deterministic ECMP routing etc...
Global state is a vector of load values in $\mathbb{R}^4$ (\si{\mega\bit\per\second}) depending upon the bandwidth measurements regularly received from monitors in the environment.
For any flow, an agent then computes the path it would take through the network.
The incoming load recorded along the first hop, last hop, and tertiles of the path may then be tile-coded together.
In the event that the path from an agent to its destination is shorter than 4 hops, we duplicate (in order of preference) the load measurement of a middle hop or the last hop.
\Cref{fig:global-state-path} illustrates the process.

We build global state in this way to offer compatibility with multipath, multi-destination networks, offering support for diverse deployment environments from endpoint servers to transit ASes.
Computing the path from agent to destination is not computationally expensive.
Multipath routing is often fast since typical \emph{equal-cost multipath} (ECMP) routing algorithms simply hash a packet's flow key, and are deterministic to provide consistent quality-of-service to hosts.

%?? Path computation fast and efficient due to deterministic routing based on hashes (ECMP)
%?? Works for any arbitrary topology, even 

We describe and analyse each of the per-flow features included in the state vector throughout \cref{sec:rethinking-the-state-space}.
Each feature is tiled separately, with the exception of packet in/out count (per-window and total), mean in/out packet size, and $\Delta$ in/out rate, which are combined with the last action taken.
Rather than having the network push the data to an agent, the agent requests this information about active flows periodically to isolate it from non-control-plane traffic and to eliminate the risk of resource exhaustion by excessive requests.

\subsection{Reward Function}

%?? Say ``we take $R_G$ from M and K''.
%
%?? Reward at each destination.

\newcommand{\arrload}[2]{\operatorname{load}^{#2}_{t}(#1)}
\newcommand{\uload}[1]{\arrload{#1}{\uparrow}}
\newcommand{\dload}[1]{\arrload{#1}{\downarrow}}
\newcommand{\bload}[1]{\arrload{#1}{\updownarrow}}
\newcommand{\cond}[2]{\operatorname{c}_{#1,t}#2}
%\fakepara{Reward function directionality}
%The reward functions, as defined, do not take traffic direction into account.
%We redefine these to identify overload states using both upstream and downstream loads, while allowing customisation of which direction is chosen for protection.
Each destination node $s$ generates a reward signal, $R_{s,t}$, at every timestep $t$.
Assume, for now, that each destination has access to some classification function $g(\cdot)$ which estimates the volume of legitimate traffic received, and expects to receive $\mathit{traffic}_s$.
Denoting the upstream, downstream and combined loads $\uload{s}, \dload{s}, \bload{s}$ at this node:

\begin{subequations}
	\vspace{-0.5cm}
	\newcommand{\load}[1]{\operatorname{load}_{t}(#1)}
	\begin{gather}
	\cond{s} = [\max(\uload{s}, \dload{s}) > U_s],\\
	R_{s,t} = (1 - \cond{s}) \frac{g(\arrload{s}{-})}{\mathit{traffic}_s} - \cond{s},\label{eqn:reward-rtx}
	\end{gather}
	replacing $\arrload{s}{-}$ in \cref{eqn:reward} with whichever directional load is prioritised according to the traffic characteristics of the deployment environment, where $\cond{s}$ represents the ``overloaded'' condition at destination $s$.\label{eqn:reward}
\end{subequations}
We choose $\uload{\cdot}$ for our UDP-based models and $\dload{\cdot}$ for HTTP, though we expect that $\bload{\cdot}$ would be the most suitable for general deployment or heterogeneous traffic patterns.
These choices reflect where the bulk of transmitted bytes in each traffic model are observed (and the lack of this knowledge in the general case).

While our use and definition of $g(\cdot)$ appears nebulous, there are many ways to infer this quantity in practice.
End-host servers may use canary flows or other active measurements, or employ existing quality-of-experience metrics in the case of VoIP services such as lost packets, reorderings, and jitter.
ASes and transit networks may make use of reports received from downstream networks, i.e. over the \emph{DDoS Open Threat Signalling} (DOTS) protocol \cite{ietf-dots-use-cases-17}.
Even if such heuristics or perfect knowledge aren't available in deployment, a sufficiently well-trained agent needs only to greedily follow the policy it has learned from training, allowing pre-training by a simulated environment (with perfect knowledge) to transfer to reality.

If a network is believed to be vulnerable to indirect attacks, such as link-flooding attacks, we may use the following reward:
\begin{equation}
	R_{s,t}^{\mathit{Cross}}(\beta) = \beta R_{s,t} + (1 - \beta) \min{\{R_{s',t} | s' \ne s\}} \label{eqn:lfa-reward}
\end{equation}
where the collaboration parameter $\beta \in [0,1]$ models the expected degree of interference between flows, and $s, s'$ are protected destination nodes in the network.
The key insight underpinning LFAs is that flows can affect a target \emph{without communicating with that target}.
$\beta$ then acts as a tunable parameter which can incentivise agents to remove flows which harm overall system health, by including the performance of the worst-performing destination.
However, such attacks (and the effectiveness of $R_{s,t}^{\mathit{Cross}}$) are not examined by our work.

\subsection{Action Space}
When monitoring a source-destination pair, an agent uses its state vector to decide which proportion of that flow's \emph{inbound} traffic should be dropped.
%?? Why pdrop? Allows for discrete action space, don't have to account for buckets/fairness/burstiness.
This is implemented by installing an action via OpenFlow, instructing its host switch to drop each relevant packet with probability $p$.
We choose to drop packets rather than impose traffic limits as it offers us a discrete action space without prior knowledge of traffic characteristics or measurement.
Furthermore, we need not consider burstiness, fairness or tuning (such as per-flow bucket sizes) which could limit scalability.
We offer two models on how to choose $p$:

\subsubsection{Instant control}
Each agent directly chooses $p \in \left\{ 0.0, 0.1, \ldots, 0.9 \right\}$, giving a discrete, static action set which cannot completely filter traffic.
These choices ensure that the rate reduction imposed on a source IP may never be permanent or irreversible.
Since this model needs no forward planning, we found it best to set the discount factor $\gamma=0$ (making agents purely myopic).

\subsubsection{Guarded control}
The measurements of \textcite{DBLP:conf/ndss/KangGS16} suggest that bot attack flows cannot scale up to match an increase in available bandwidth.
We apply their observations within the RL paradigm by constraining how an agent treats each flow using a simple finite state automaton: we restrict $p \in \left\{ 0.00, 0.05, 0.25, 0.50, 1.0 \right\}$.
The action set is then simply to \emph{maintain}, \emph{increase}, or \emph{decrease} $p$ for a flow in single steps.
We choose these potential values for $p$ to add complete filtering to a steady progression of rate-limiters (\SI{25}{\percent} increments for UDP traffic).
The outlier, $p=0.05$, corresponds to roughly a \SI{50}{\percent} rate reduction for TCP flows in our test topology.
This uneven spread of choices for $p$ allows light and heavy rate reduction to be applied to both congestion-aware and congestion-unaware traffic as required.

To enable temporary bandwidth expansion in all deployments, every flow is initially placed under light packet drop ($p=0.05$); this is chosen above the equivalent for UDP due to TCP's higher prevalence.
Most importantly, an agent must now choose to punish a flow multiple times in succession to cause rapid degradation, reducing variance while allowing an agent to see how a host reacts to structured changes in the environment.

As each agent now requires the capability to plan ahead, we require a discount factor $\gamma \ne 0$, allowing the value of future states to influence state-action value updates.
We found the setting $\gamma = 0.8$ to be the most effective choice for this hyperparameter during exploratory testing.

\subsubsection{Risks}
Our mode of action means that each agent is in control of pushback \cite{DBLP:journals/ccr/MahajanBFIPS02a}, and so carries a risk of introducing collateral damage into the network.
This is particularly severe when handling TCP traffic: the Mathis equation \cite{DBLP:journals/ccr/MathisSMO97} states that TCP bandwidth is proportional to $1/\sqrt{p}$ (noting that $p$ is nonzero in any real network) while constant bitrate (CBR) UDP traffic is proportional to $1 - p$.
%It's worth noting that there are various ways that this could be implemented, and that the application of \emph{programmable data planes} to this end are suggested as future work.
This weakness is still present in modern TCP flavours, such as TCP Cubic which in turn has bandwidth proportional to $1/p^{0.75}$ \cite{rfc8312}.
This is of particular importance due to the prevalence of TCP and other congestion-aware protocols within the Internet.
Our own analysis of CAIDA datasets \cite{caida-2018-passive} shows that congestion-aware traffic makes up at least \SIrange{73}{82}{\percent} of packets, corresponding to \SIrange{77}{84}{\percent} of data volume\footnote{\url{https://github.com/FelixMcFelix/caida-stats}}.
QUIC, a future congestion-aware protocol, comprises \SIrange{2.6}{9.1}{\percent} of traffic observed on backbone links, depending on location and typical workload \cite{DBLP:conf/pam/RuthPDH18}.
%As far as future network protocols are concerned, QUIC \cite{DBLP:conf/sigcomm/LangleyRWVKZYKS17}, a congestion-aware stream transmission protocol, will behave much like TCP, showing the importance of further development to properly handle traffic with such characteristics.

%?? Make more sane. This is about ``Why go per-flow?''
This further justifies our focus on per-flow decisions---real-world deployments see many flows pass over any egress point, making global actions (such as those chosen by \textcite{DBLP:journals/eaai/MalialisK15}) more likely to inflict collateral damage.
%This manifests in two ways: the best-achievable performance drops, and so too does the learning rate.
Given the probability that a host is legitimate, $P_G \in [0,1]$, it follows that a host will be malicious with probability $P_B = 1 - P_G$.
Defining \emph{imperfect service} to mean any case where all $n$ hosts connecting over a switch do not share the same classification (i.e., a mixture), then the probability that a switch is delivering imperfect service is $P_{M,n} = 1 - (P_G^n + P_B^n)$.
\begin{thm}
	As the host/learner ratio $n$ increases, it is more likely that a throttling switch will exhibit imperfect service: $\forall n \in \mathbb{Z}^{+}, P_{M,n} \le P_{M,n+1}$.
\end{thm}
\begin{proof}
	\emph{Base case:} $P_{M,1}=0, P_{M,2} = 1 - P_G^2 - P_B^2 > 0$.
	\emph{Inductive step:} Assume that the theorem holds for $n$. Observe that $P_G^n \ge P_G^{n+1}$ (resp.\ $P_B$). It then follows that:
	\begin{align*}
	P_G^n + P_B^n &\ge P_G^{n+1} + P_B^{n+1}\\
	1 - (P_G^n + P_B^n) &\le 1 - (P_G^{n+1} + P_B^{n+1})\\
	P_{M,n} &\le P_{M,n+1} \qedhere
	\end{align*}
\end{proof}
\begin{corr}
	Restricting $P_G \in (0,1)$ so that both $P_G$ and $P_B$ are non-zero ensures strict inequality: $P_{M,n} < P_{M,n+1}$.
\end{corr}
When considering that many hosts have an especially adverse reaction to our main means of control, flow-level granularity becomes an obvious choice.

\subsection{Systems Considerations}\label{sec:systems-considerations}
Taking many actions per timestep means that any agents are assigned a larger, and potentially unbounded, set of tasks to perform every time they receive load and flow statistics from the network and their parent switch.
This introduces some potential issues: the inability to respond to unexpected changes in flow state, delayed service of new flows, and risks that flow states become outdated.
At their worst, these risks present additional attack surface to an adversary.
To adapt to these problems, we make use of \emph{timed random sequential} updates.

Each agent begins with an empty work list.
For the set of flows active in any timestep, we shuffle the list and perform as many action calculations and updates as possible, within a set time limit.
Uncompleted work is passed on to the next timestep, until the list is emptied, at which point it is repopulated using the set of available measurements.
To ensure that flow control actions are made with recent information, we combine state vectors for unvisited flows in the current work set, and replace the stored vector for all others.
State vector combination is done by summing deltas and packet counts, updating means via weighted sums, and replacing all other fields.
Following \citeauthor{DBLP:conf/sigcomm/ChenL0L18}'s observations concerning short flows \cite{DBLP:conf/sigcomm/ChenL0L18}, we maintain a deadline of \SI{1}{\milli\second}---in tests, an agent is typically able to process around 3 flows in this time.
We expect this should be tuned based on the frequency at which statistics arrive.
Naturally, this implies that an agent must carry work forward (and coalesce state updates) when \emph{host density} is $n>3$ (\cref{sec:evaluation}); this behaviour is not explicitly a property of network size.

\section{System architecture}\label{sec:ddos-architecture}
%?? Here is where I would put my ramblings about system architecture...
We present our design of a system which supports the effective real-world deployment of RL-based DDoS mitigation.
\Cref{fig:sys-arch} displays this, separating system elements which are local to each agent from those which reside elsewhere in the network.
We operate each agent as a \emph{virtualised Network Function} (vNF) adjacent to a software-defined switch.
SDN allows a controller (or authorised hosts) to install actions, forwarding rules and logic into a switch at runtime.
Moreover, networks of this kind more naturally enable the future relocation and easy installation of learners.
Agent vNFs communicate with these co-hosted switches to install probabilistic packet-drop rules.
We describe the main purpose and operation of each module within an agent's vNF, and discuss techniques to make deployment more efficient using existing technologies.

\subsection{Core and RL Executor}\label{sec:core-and-rl-executor}
The core module is the main loop in an agent's architecture.
At each timestep, the core receives information about which flows have arrived and should be acted upon from the \emph{TRS Scheduler}.
The core then retrieves the current and previous state vector associated with each flow from the \emph{Flowstate Database}, passing them into the RL algorithm alongside the last action chosen for that flow (if available).

The RL algorithm then returns an action.
Each action is passed to the database, which computes and returns a packet drop rate according to the agent model (\emph{Instant/Guarded}) while updating flow state.
This is then converted into an OpenFlow message carrying packet drop rules; these are batched to the agent's switch using the same groupings produced by the scheduler.
Finally, timing information is passed back into the scheduler to refine its estimates about how much work should be scheduled in the next timestep.

State space sizes guarantee that an \emph{Instant} policy remains under \SI{520}{\kibi\byte}, though our sparse representation typically leads to far smaller policies: $\sim$\SI{17.8}{\kibi\byte} from our experiments.
\emph{Guarded} policies are \SI{30}{\percent} of this size.
As described earlier, action updates require a constant number of floating point operations---\num{160} floating point additions and \num{32} multiplications per update of $\wvec{}$ with per-tile updates, with \num{160} additions required to choose an action.
The vast majority of these operations can be vectorised.
Action computation for \emph{Guarded} agents is cheaper still, requiring only \num{48} additions per action.

\subsection{Stats API and Collectors}
Agents require information from the network and one another to be effective.
Our agents can act either independently, having no agent-to-agent communication, or cooperatively.
In the latter case agents transfer, when possible, \emph{experience} to one another---lists of state-action-state transitions with associated rewards.
It's noted that a transition may be high-value or surprising to one agent, while well-known to another, causing each to produce different policy updates from the same unit of experience.
For this reason we do not transfer policy deltas between agents, causing each to learn its own policy.
Which scheme achieves better performance is left to future work.

%?? How do stats get from switcehs to the agent?
Load collectors and estimators periodically push observations to each active agent vNF.
In our current implementation, load statistics are gathered via vNFs active at each network switch, though we expect that OpenFlow stats requests, NetFlow or SNMP data may be used to derive these cheaply.
Transferring state to agents and experience sharing can both be made more efficient through effective use of broadcast addressing in a target network.
Depending on the capabilities of switches in the network, the estimator can either send benign traffic estimates or parameters for use in a reward function.

Gathering and transmission of load/flow statistics would be difficult to perform quite as often as an emulated environment allows.
However, the measurements acquired in such a scenario are likely to be less noisy (by being collected over longer periods of time), which could aid effective training.

\subsection{Flowstate Database}
For each flow 5-tuple, we hold two state vectors containing the features described in \cref{sec:feature-space}---the current state, and the state which induced the last action.
To ensure that flow control actions are made with recent information, we combine state vectors for unvisited flows in the current work set.
State vector combination is done by summing deltas and packet counts, updating means via weighted sums, and replacing all other fields with the newest measurements.
For flows outside of the current work list, we simply replace the stored vector.

\subsection{TRS Scheduler}
Acting on an unbounded set of flows in each timestep introduces potential issues: the inability to respond to unexpected changes in flow state, delayed service of new flows, and the risk that flow states become outdated.
At their worst, these risks present additional attack surface to an adversary.
To tackle these problems, we make use of \emph{Timed Random Sequential} updates.

The scheduler begins with a shuffled work list of active flows.
When requested, the scheduler estimates the cost of an action computation using the timing information received from the core, proceeding down the list to send a set of 5-tuples to the core which can be handled in a set time limit.
The scheduler continues until the list is empty, at which point it is repopulated and reshuffled with active flows.

Following \citeauthor{DBLP:conf/sigcomm/ChenL0L18}'s observations on handling short flows \cite{DBLP:conf/sigcomm/ChenL0L18}, we maintain a deadline of \SI{1}{\milli\second}---an agent is typically able to process around 3 flows in this time.
We expect deadlines should be tuned based on the frequency at which statistics arrive.
The amount of processed flows per deadline depends on the agent design (FLOP count, policy size), but also on the amount of flow telemetry data needing processed---our current implementation is written in python, restricting this handling to a single thread.
An implementation in a systems language such as Rust or C++ would allow faster concurrent processing.

There is a risk that so much work can be queued up that an agent is never able to act on some attack flows.
A solution is to impose an upper bound on the amount of action computations/policy updates that can be performed before the work list is regenerated.
This removes the guarantee that all flows will be visited often, but if updates occur regularly then this sampling may be sufficient to achieve good performance.

\subsection{Agent Switches}
%?? How do we do stat collection?
Our agent switches operate a modified version of Open vSwitch, implementing an action which requests that each matched packet be dropped with a certain probability.
To get around the lack of floating-point support in many environments, we represent this probability using a 32-bit unsigned integer (scaling \num{1.0} to $2^{32}-1$).
On commodity hardware, we believe that a similar effect can be achieved using OpenFlow meters (at the expense of these being stateful measures).

We use OpenFlow groups to simplify control messages: premade tables with permitted levels of packet drop.
This saves some overhead compared to using experimenter/extension headers.
Flows are automatically given a group with the default level of packet drop (according to the chosen agent design), meaning that switches don't need to refer to a controller or the agent vNF.

%?? What do our modifications to openflow look like? How do we get around the lack of floating point math?

\begin{figure}
	\centering
	\resizebox{0.8\linewidth}{!}{\begin{tikzpicture}
		\node(remote){Remote};
		
		%%%
		
		\node[below=0.05 of remote](swpos){};
		\node[fill=white!80!uofgcobalt, draw=black, minimum height=1.5cm, minimum width=2cm, below right= 0.1 of swpos.north west](sw1){};
		\node[fill=white!80!uofgcobalt, draw=black, minimum height=1.5cm, minimum width=2cm, below right= 0.1 of sw1.north west](sw2){};
		\node[fill=white!80!uofgcobalt, draw=black, minimum height=1.5cm, minimum width=2cm, below right= 0.1 of sw2.north west](switch){};
		\node[below right, inner sep=2pt] at (switch.north west) {\small Switch};
		\node[fill=white!90!uofgcobalt, draw, rectangle, rounded corners=0.05cm, above=0.1] (oswlc) at (switch.south) {\begin{varwidth}{1.5 cm}\small \centering Load\\Collector\end{varwidth}};
		
		%
		
		\node[right=2 of swpos](epos){};
		\node[fill=white!80!uofgthistle, draw=black, minimum height=1.5cm, minimum width=3.5cm, below right= 0.1 of epos.north west](e1){};
		\node[fill=white!80!uofgthistle, draw=black, minimum height=1.5cm, minimum width=3.5cm, below right= 0.1 of e1.north west](e2){};
		\node[fill=white!80!uofgthistle, draw=black, minimum height=1.5cm, minimum width=3.5cm, below right= 0.1 of e2.north west](egress){};
		\node[below right, inner sep=2pt] at (egress.north west) {\small Egress Switch};
		\node[fill=white!90!uofgthistle, draw, rectangle, rounded corners=0.05cm, above=0.1] (eglc) at ($(egress.south) + (-0.85,0)$) {\begin{varwidth}{1.5 cm}\small \centering Load\\Collector\end{varwidth}};
		\node[fill=white!90!uofgthistle, draw, rectangle, rounded corners=0.05cm, right=0.1] (egest) at (eglc.east) {\begin{varwidth}{1.5 cm}\small \centering Estimator\\$g(\cdot)$\end{varwidth}};
		
		%
		
		\node[right=3.5 of epos](apos){};
		\node[fill=white!60!uofgpumpkin, draw=black, minimum height=1.5cm, minimum width=2cm, below right= 0.1 of apos.north west](a1){};
		\node[fill=white!60!uofgpumpkin, draw=black, minimum height=1.5cm, minimum width=2cm, below right= 0.1 of a1.north west](a2){};
		\node[fill=white!60!uofgpumpkin, draw=black, minimum height=1.5cm, minimum width=2cm, below right= 0.1 of a2.north west](otheragent){};
		\node[below right, inner sep=2pt] at (otheragent.north west) {\small Agent vNF};
		\node[fill=white!90!uofgpumpkin, draw, rectangle, rounded corners=0.05cm, above=0.3] (oasa) at (otheragent.south) {\begin{varwidth}{1.5 cm}\small \centering Stats API\end{varwidth}};
		
		%
		
		\node[below=2.3 of remote.west](linestart){};
		\path let \p1 = (linestart) in node (lineend) at (9,\y1){};
		\draw [dashed] (linestart) -- (lineend);
		
		%%%
		
		\node[below=2.4 of remote](local){Local};
		
		%%%
		
		\node[below=0.05 of local](aswpos){};
		\node[fill=white!80!uofgthistle, draw=black, minimum height=3.5cm, minimum width=2.2cm, below right= 0.1 of aswpos.north west](aswitch){};
		\node[below right, inner sep=2pt] at (aswitch.north west) {\small Agent Switch};
		\node[fill=white!90!uofgthistle, draw, rectangle, rounded corners=0.05cm, above=0.1] (aswoft) at (aswitch.south) {\begin{varwidth}{1.5 cm}\small \centering OpenFlow\\Tables\end{varwidth}};
		\node[fill=white!90!uofgthistle, draw, rectangle, rounded corners=0.05cm, above=0.1] (aswsc) at (aswoft.north) {\begin{varwidth}{1.5 cm}\small \centering Stats\\Collector\end{varwidth}};
		\node[fill=white!90!uofgthistle, draw, rectangle, rounded corners=0.05cm, above=0.1] (aswlc) at (aswsc.north) {\begin{varwidth}{1.5 cm}\small \centering Load\\Collector\end{varwidth}};
		
		%
		
		\node[right=3 of aswpos](avfpos){};
		\node[fill=white!60!uofgpumpkin, draw=black, minimum height=3.5cm, minimum width=4.5cm, below right= 0.1 of avfpos.north west](avf){};
		\node[below right, inner sep=2pt] at (avf.north west) {\small Agent vNF};
		\node[fill=white!90!uofgpumpkin, draw, rectangle, rounded corners=0.05cm, below=0.15] (avfsa) at ($(avf.north) + (0.2,0)$) {\begin{varwidth}{1.5 cm}\small \centering Stats API\end{varwidth}};
		\node[fill=white!90!uofgpumpkin, draw, rectangle, rounded corners=0.05cm, below=0.9] (avfdb) at (avfsa.south west) {\begin{varwidth}{1.5 cm}\small \centering Flowstate\\Database\end{varwidth}};
		\node[fill=white!90!uofgpumpkin, draw, rectangle, rounded corners=0.05cm, right=0.1] (avfsched) at (avfdb.east) {\begin{varwidth}{1.5 cm}\small \centering TRS\\Scheduler\end{varwidth}};
		\node[fill=white!90!uofgpumpkin, draw, rectangle, rounded corners=0.05cm, below=2.3] (avfcore) at (avfsa.south) {\begin{varwidth}{1.5 cm}\small \centering Core\end{varwidth}};
		\node[fill=white!90!uofgpumpkin, draw, rectangle, rounded corners=0.05cm, right=0.8] (avfrl) at (avfcore.east) {\begin{varwidth}{1.5 cm}\small \centering RL\end{varwidth}};
		
		%%%
		
		\tikzset{>=stealth}
		
		\draw[thick, ->] (aswlc) -- (avfsa.west) node[midway,above] {\tiny Current load};
		\draw[thick, ->] (aswsc) -- (avfsa.west) node[midway,sloped, above] {\tiny Flow stats};
		
		\draw[thick, ->] (oswlc) -- (avfsa) node[midway,above, sloped] {\tiny Current load};
		
		\draw[thick, ->] (eglc) -- (avfsa) node[midway,above, sloped] {\tiny Current load};
		\draw[thick, ->] (egest) -- (avfsa) node[midway,above, sloped] {\tiny Estimation data};
		
		\draw[thick, <->] (oasa) -- (avfsa) node[midway,above, sloped] {\tiny Experience};
		
		\draw[thick, ->] (avfsa) -- (avfdb) node[midway,above, sloped] {\tiny State};
		\draw[thick, ->] (avfsa) -- (avfsched) node[midway,above, sloped] {\tiny Live flows};
		
		\draw[thick, ->] (avfcore) -- (aswoft) node[midway,above, sloped] {\tiny Packet drop rules};
		\draw[thick, <->] (avfcore) -- (avfdb) node[midway,below, sloped] {\tiny State};
		\draw[thick, <->] (avfcore) -- (avfsched) node[midway,below, sloped] {\tiny Work};
		\draw[thick, <-] ($(avfcore.east) + (0,0.1)$) -- ($(avfrl.west) + (0,0.1)$) node[midway,above, sloped] {\tiny Actions};
		\draw[thick, ->] (avfcore) -- (avfrl) node[midway,below, sloped] {\tiny State};
		
		\draw[thick, ->] (avfsa.east) to [out=0, in=45] (avfrl.north);
		\node[right=0.3] at (avfsa.east) {\tiny Experience};
	\end{tikzpicture}}
	\caption{
		System architecture  for our RL-driven DDoS defence system.
		\label{fig:sys-arch}
	}
\vspace{-1em}
\end{figure}

\section{Rethinking the state space}\label{sec:rethinking-the-state-space}
\begin{figure}
	\centering
	\includegraphics[width=0.65\linewidth]{plots/marl/ftprep-cap-box}
	\caption{
		Learned performance of Instant Control agents when benign traffic is UDP-like, using only a single feature as a basis for decisions.
		Mean IAT, inbound packet sizes, and global state offer the best predictive performance, while most features offer marginal advantage over the unprotected baseline.
		\label{fig:udp-feature-plots}
	}
\end{figure}

%\begin{figure}
%	\centering
%	\includegraphics[width=\linewidth]{../plots/ftprep-laf-cap-box}
%	\vspace{-1.2cm}
%	\caption{
%		?? UDP, combined with last action.
%		\label{fig:udp-laf-feature-plots}
%	}
%\end{figure}

\begin{figure}
	\centering
	\includegraphics[width=0.65\linewidth]{plots/marl/ftprep-tcp-cap-box}
	\caption{
		Learned performance of Instant Control agents when benign traffic is TCP-like, using only a single feature as a basis for decisions.
		All of the chosen features can offer a marked improvement over no protection at all.
		Global state and Mean IAT still offer the greatest improvement above baseline, but packet-level statistics are considerably less effective for this class of traffic.
		\label{fig:tcp-cap-feature-plots}
	}
\end{figure}

\begin{figure}
	\centering
%	\vspace{-0.25cm}
	\includegraphics[width=0.65\linewidth]{plots/marl/ftprep-tcp-laf-cap-box}
	\caption{
		Learned performance of Instant Control agents when benign traffic is TCP-like, combining each feature with the last action taken as a basis for decisions.
		This combination causes a significant improvement in the effectiveness of packet-level and per-window statistics.
		\label{fig:tcp-laf-feature-plots}
	}
\end{figure}

The main element required by a per-source model is a feature set with high predictive power, so that behavioural differences are apparent to an agent.
Elaborating on the statistics discussed in \cref{sec:motivation} which others have shown to be effective, we believe the following features to be useful (and humanly justifiable), and investigate their use alongside different traffic types:
%?? We use these features, and why...

\fakepara{Global state}
This is the vector of load measurements along a flow's path introduced in \cref{sec:feature-space}.
These values indicate the overall health of the network, and crucially are all measurements which an agent directly controls.

\fakepara{Source IP address}
While trivial to spoof (and thus of limited use for many classes of attack), reflectors are themselves legitimate services being abused by spoofing attackers.
As a result, they communicate with attack victims using their own IP address.
In real-world scenarios the addresses of reflector nodes might exhibit similarity due to network uncleanliness~\parencite{DBLP:conf/imc/CollinsSFJWSK07}, e.g., unhardened services exposed by a single organisation.

\fakepara{Last action taken}
This encodes an agent's current belief in the maliciousness of a flow.
This feature also potentially allows forgiveness, serving as a reference point for determining whether a source mistakenly marked as malicious exhibits different falloff behaviour after punishment.
It's important to note that this feature only makes sense once combined with another flow feature, and never appears individually tile-coded.

\fakepara{Flow duration and size}
Features which describe the length of time a connection has been active, and the amount of data transferred within that time.
An extraordinarily long flow, having sent a lot of data, could be more likely to be an amplifier: though most (\SI{62}{\percent}) waves of amplifier traffic last shorter than \SI{15}{\minute} \cite{DBLP:conf/raid/KramerKMNKYR15}, this is considerably longer than the typical length of an HTTP request/response.

\fakepara{Correspondence ratio}
The ratio between upstream and downstream traffic for a source IP.
We define this to be $C_X = \min(\uload{\cdot}, \dload{\cdot})/\max(\uload{\cdot}, \dload{\cdot})$, where a value close to 0 indicates strong asymmetry.

\fakepara{$\mathbf{\Delta}$ Send/receive rate}
The change in traffic rates caused by the last action.
Behavioural changes induced by bandwidth expansion/reduction are expected to be most visible here.

\fakepara{Mean inter-arrival time (IAT)}
A measure of how often packets arrive at the agent's parent switch; low IATs indicate a high number of packets per second, and can be a possible marker of malicious behaviour.
We only make use of the mean IAT of \emph{inbound} traffic.

\fakepara{(Per-window) packet count}
The amount of packets sent to/from a source over a flow's lifetime (or the current window of measurement), similar in use to flow size and mean IAT.

\fakepara{Mean packet size per window}
Legitimate flows, both TCP- and UDP-based, often transmit packets with a distribution of sizes.
Attack traffic is not likely to be so diverse: we might expect solely max-size packets in the case of amplification attacks, or minimum-size packets in other flooding attacks.

The exclusion of features such as source/destination ports or protocol numbers is a deliberate choice.
If \emph{QUIC} (or a similar protocol) were to become ubiquitous, then these fields would have little to no correlation with the class of traffic a flow might contain.
Our aim was to design around this constraint as a form of future-proofing.

\begin{table}
	\centering
	\caption{Tile coding windows for each feature.\label{tab:codings}}
	
	\resizebox{0.45\linewidth}{!}{
	\begin{tabular}{@{}ll@{}}
		\toprule
		New Feature (unit) & Range \\
		\midrule
		Load (\si{\mega\bit\per\second}) & $[0, U_s]$ \\
		IP & $[0, 2^{32}-1]$ \\
		Last Action (\si{\percent}) & $[0, 1]$ \\
		Duration (\si{\milli\second}) & $[0, \num{2000}]$ \\
		Size (\si{\mebi\byte}) & $[0,10]$ \\
		Correspondence Ratio & $[0,1]$ \\
		Mean IAT (\si{\milli\second}) & $[0, \num{10000}]$ \\
		$\Delta$In/Out Rate (\si{\mega\bit\per\second}) & $[-50, 50]$ \\
		Packets In/Out & $[0, 7000]$ \\
		Packets In/Out Window & $[0, 2000]$ \\
		Mean In/Out Packet Size (\si{\byte}) & $[0, 1560]$ \\
		\bottomrule
	\end{tabular}
	}
\vspace{-0.5cm}
\end{table}

All of the above features, save for global state, are 1-dimensional.
\Cref{fig:udp-feature-plots} shows the effectiveness of each feature for UDP (resp.\ \cref{fig:tcp-cap-feature-plots} for TCP), on a single-destination topology (\cref{sec:single-dest}) with $n=2$ hosts per egress point averaged over 10 runs.
\Cref{fig:tcp-laf-feature-plots} demonstrates how feature accuracy varies when tiled alongside \emph{last action}, with similar trends observed when applied to UDP traffic (omitted).
%?? Core findings---different protocols need different features, so everything we proposed above has a use!
The plots show that different protocols and traffic classes are best defended by different features---as such, every feature presented has value in a complete model.
All features converge to their highest-observed performance within around \num{4000} timesteps.
In general, some of the most effective features are the global state, mean IAT, mean inbound packet size and $\Delta$ rates.
%?? How do they do when combined after individual training? Pretty well, especially for TCP.
%Additional testing shows that the learned per-feature policies may be easily combined (by summing action values), and that this technique is particularly effective for TCP; these results are omitted to preserve space.
%In no cases, however, do we manage to completely block attack traffic---at convergence, we observe that system load remains consistently at $U_s$.

\section{Traffic modelling}\label{sec:a-new-normal}

%In establishing...
%
%?? How will I structure this?
%?? Motivation -> Model -> Results?
%?? OR Use the results of the last section to springboard into here?

%From what we have seen, it is difficult (or impossible) for trace-based or numerical simulations to correctly capture certain dynamics without an extraordinary amount of care or consideration.
%As it turns out, 
%Our goal is to briefly describe an environment which tests \emph{specific} behaviours to examine the \emph{specific} problems which have arisen during our testing of past approaches.
Here, I describe network models built around live testing of reactive \gls{acr:tcp} and \gls{acr:udp} traffic in an \gls{acr:sdn}-enabled environment, which is adaptable to arbitrary topologies, with an explicit focus on preserving their real-time dynamics in a way that trace-based evaluation cannot.
First and foremost, the goal is to replicate representative load and packet inter-arrival characteristics, and to capture how these characteristics evolve in response to actions.
I introduce these models because we are interested in capturing interactive, correlated back-and-forth exchanges associated with live \gls{acr:http} traffic; mainly because of the particular interactions between the application-level dynamics, congestion awareness at the transport level and the nature of control signal used.
%Naturally, this model is not perfect or representative for all traffic, yet it captures some of the behaviour which we expect will plague most legitimate TCP flows.
%If need be, we expect the frequency or distribution of requests could be conditioned to match observations of real-world access patterns.

%?? ANGLE: set up an environment to test \emph{specific} behaviours to examine \emph{specific} problems in past work. I make no claims that it is perfect or representative for all traffic, just for this (likely common) behaviour which I expect to plague almost all legit TCP flows.

%?? Existing sims used for testing such applications reliant on traces, or not sophisticated enough to capture interactive, back-and-forth (correlated) behaviours---possibly discarded as second-hand effects by past work when these are so crucial given user traffic patterns (and the nature of the control signal we choose to enact).

%?? Remember, the motivation is clear. We don't care so much that it is "representative" wrt a specific deployment location or network type. The whole purpose of this is that we aim to test specific behaviour which traces cannot replicate (i.e., correlated back-and-forth, dynamics introduced to congestion-aware protocols, ...)
%?? If we need to, we can condition the distribution of requests according to statistics mined from an existing trace if reviewer number 2 needs that extra push to be convinced.

\subsection{Network design}
We make use of a fully software-defined network, built using OpenFlow-aware switches in \emph{mininet}~\parencite{mininet} alongside a controller based on \emph{Ryu}~\parencite{ryu}.
All internal routers are primed with knowledge of the shortest path to each internal host, while new inbound flows register the `way back' for each hop used, to ensure consistent bidirectional traffic conditions for each flow.
If several ports offer different (equal-length) paths to a destination, a consistent random port is chosen from the flow-hash by an OpenFlow \emph{Group action} (in \emph{select} mode).
If such information is lost, perhaps expiring due to inactivity, it suffices to forward an outbound packet on a random outbound port, as we assume that any external \gls{acr:ip} address is reachable through any of the test network's egress ports (i.e., that it is not connected to any stub \glspl{acr:as}).
The controller is also responsible for computing how switches respond to ARP requests: this need arises due to the reliance upon Linux's networking stack for live applications, and wouldn't need to be considered for trace-based evaluation.
%We make further use of the topology presented earlier (\cref{sec:topology}), noting that our architecture allows us to trivially extend and modify this if required.

\subsection{TCP (HTTP) traffic model}\label{sec:tcp-http-traffic-model}
%?? Legitimate traffic: TCP traffic (HTTP clients downloading web pages, dependent resources and files) with a mixture of lifetimes for each request.
To model legitimate \gls{acr:tcp} traffic, server nodes run an nginx v1.10.3 \gls{acr:http} daemon, serving statically generated web pages alongside various large files and binaries.
Benign hosts run a simple libcurl-based application written in Rust, repeatedly requesting resources from the server.
Hosts and clients both use \gls{acr:tcp} Cubic~\parencite{rfc8312}.
Each host's download rate is limited to match the maximum bandwidth assigned to it, and requests several random files known to exist within a website, followed by any dependent resources for each (stylesheets, images, etc.) as a browser might.
On completion, a host changes its \gls{acr:ip} address to generate separate statistics per-flow, while minimising downtime.
This presents a balanced distribution of flow duration and size, with large files included to model elephant flows.

\subsection{UDP (Opus/VoIP) traffic model}\label{sec:udp-opus-traffic-model}
\gls{acr:voip} traffic exhibits very different characteristics to the above model; packet arrivals are highly periodic due to real-time requirements, flows have a constant bitrate, and do not react substantially to lost packets.
Interestingly, \gls{acr:ddos} attack traffic is known to share many of these characteristics, offering an interesting detection problem.
I present a \gls{acr:voip} traffic model based on Discord~\parencite{discord}, a freely-available messaging and \gls{acr:voip} platform geared toward gaming communities.
Discord is a good model for this prototype due to its publicly documented \gls{acr:api}, many open source bot frameworks, large user base, and due to the lack of models for Opus-encoded traffic.
Further details on trace measurement and generation are provided through \cref{adx:opus-traffic}.
%?? Highly periodic, CBR

Hosts send \gls{acr:rtp} traffic with Salsa20 encrypted payloads---\qty{20}{\milli\second} audio frames at \qty{96}{\kilo\bit\per\second}.
We generate similar traffic at hosts by replaying anonymised traces gathered in general use and tabletop role-playing servers; each trace contains only the size of each audio payload, entries denoting missed packets, and the duration of silent periods.
We trim these silent periods to a maximum \qty{5}{\second} due to the lengthy talk/silence bursts introduced by users in RPG servers, and estimate the size of missed packets by taking an exponentially-weighted moving average over known sizes.
Hosts punctuate audio frames with a 4-byte keepalive every \qty{5}{\second}.
All traffic passes over a central server which groups hosts into rooms, and is forwarded to other participants; we do not replicate pre-call Websocket traffic which would be used for authentication.
There is no peer-to-peer traffic---the server acts as a \gls{acr:turn} relay for all hosts.
%?? Reflective factor among \emph{authenticated hosts}.
Each flow occupies an expected \qty{52.4}{\kilo\bit\per\second} upstream bandwidth.
To match the target upload rate assigned to a host, each runs enough individual sessions to meet the target data rate.

%?? Malicious traffic: UDP flood traffic (hping3, MTU-size packets, ). Why not min-size packets? Because the traffic generator gets in a horrible rut if I do so...
\subsection{Attack traffic model}\label{sec:attack-traffic-model}
Malicious traffic is generated by use of the \emph{hping3} program, generating \gls{acr:udp}-flood traffic targeting random ports.
Each instance of \emph{hping3} was configured to generate Ethernet \gls{acr:mtu}-sized packets (\qty{1500}{\byte}) with a random source and destination port towards a target server, and configure the output rate $r$ (in \unit{\mega\bit\per\second}) by setting the inter-arrival time $t_{\mathit{attack}}=\frac{1500 \cdot 8}{r\cdot10^6}$.
This fulfils certain characteristics of many types of amplification \gls{acr:ddos} traffic: it is congestion-unaware~\parencite{DBLP:conf/ndss/Rossow14}, and packets are larger than the minimum frame size and identically-sized.
This latter behaviour is seen in the wild: \gls{acr:ntp} amplification traffic is fragmented at the application layer into \qty{482}{\byte} chunks~\parencite{cisco-ntp-amp}.
This model differs from \gls{acr:ntp} amplification in frame size so that inter-arrival times are larger, to keep emulation of the network feasible at high traffic rates.

\section{Evaluation}\label{sec:ddos-evaluation}
%Traffic is played back from hosts via Tcpreplay at a bandwidth assigned uniformly from a `good' or `bad' distribution, each using the same pcap file with source and destination IP addresses rewritten.

This work is most naturally compared against Marl, introduced by \textcite{DBLP:journals/eaai/MalialisK15}, the state-of-the-art in \gls{acr:rl}-based \gls{acr:ddos} prevention.
We are most interested in seeing how their approach contrasts with the new agent designs across different topologies and workloads.
Different network environments will also impose different levels of host density, where popular web servers may have orders of magnitude more clients than egress points from their network---I aim to show how these characteristics affect performance and learning rate.
Marl is known to outperform the AIMD~\parencite{DBLP:journals/ton/YauLLY05} strategy, yet the state of the art has long since moved on.
To paint a more current picture, I compare this work against an effective modern approach, \emph{SPIFFY}~\parencite{DBLP:conf/ndss/KangGS16}.
SPIFFY tests a proportion of flows by routing them through an alternate path with higher bandwidth, observing how their speed changes some time later.
This comparison lets us position our new agent designs against the state of the art, observing that SPIFFY has a similar mode of interaction to \gls{acr:rl}-based systems (taking action, observing an effect, and acting once again) and does not rely on protocol characteristics or signatures.
In reimplementing SPIFFY, I make the simplifying assumption that a suitable unused path exists (with identical bandwidth to the server's link).
\qty{10}{\percent} of active flows were tested at a time (according to the authors' observation that there is a factor of \qty{10}{\times} difference between the ideal and achieved bandwidth expansion), excluding flows below \qty{50}{\kilo\bit\per\second} and requiring a \qty{3}{\times} expansion from legitimate flows, making a judgement after \qty{5}{\second}.

To test this, I made use of both traffic models introduced in \cref{sec:a-new-normal} (Opus and \gls{acr:tcp}), both topologies discussed below (1-dest vs.\ Fat-Tree), and vary the amount of hosts typically communicating over each agent's ingress/egress node.
Additionally, these new models were evaluated in multi-agent mode (\emph{separate}, no model sharing), and in single-agent mode (\emph{single}, zero-cost perfect information sharing).
In each case, the algorithm's performance was averaged over \num{10} episodes of length \num{10000} timesteps (setting each agent's $\wvec{}=\mathbf{0}$ between episodes).
Host allocations at the beginning of each episode were generated pseudorandomly to ensure fairness between episodes---a host is malicious with probability $\operatorname{P}\left(\mathit{malicious}\right)$, and is benign otherwise.
Benign hosts generate traffic according to either \cref{sec:tcp-http-traffic-model,sec:udp-opus-traffic-model} depending on the experiment, while malicious hosts generate traffic as described in \cref{sec:attack-traffic-model} (both at experiment-dependent rates).

All experiments were executed on Ubuntu 18.04.2 LTS (GNU/Linux 4.4.3-040403-generic x86\_64), using an Intel Core i7-6700K (\qtyproduct[product-units=single]{4 x 4.2}{\giga\hertz}) which had \SI{32}{\gibi\byte} of \gls{acr:ram}.
%All code underpinning these findings is available on a public repository\footnote{\url{https://github.com/FelixMcFelix/rln-dc-ddos-paper}}.
%All code underpinning these findings is available on a public repository.\footnote{Private until publication.}

\subsection{Single destination}\label{sec:single-dest}
%?? Move description of tree topol to here.
The network is tree-structured, where one server $s$ connects through a dedicated switch to $k$ team leader switches, each connected to $\ell$ intermediate switches, which in turn each connect to $m$ egress switches.
We then have $N_{\mathit{hosts}} = k \ell m n$.
\Cref{fig:marl-topol} demonstrates this.
%Although \citeauthor{DBLP:journals/ccr/MahajanBFIPS02a}, the originators of this topology, make it clear that it exists as a fairly unrepresentative example \cite{DBLP:journals/ccr/MahajanBFIPS02a}, it remains the case that such a network topology allows for functional testing, and indeed is illustrative of one way in which attack traffic might aggregate in the network.
%It is hard, however, to argue its relevance to specific classes of victim or to reason about the interactions it might have with dependent applications.
%We aim to address this through \cref{sec:performance-in-an-emulated-environment}.
The network topology was configured using $k=2$ teams, $\ell=3$ intermediate nodes per team, $m=2$ agents per intermediate node, and $n \in \{2, 4, 8, 16\}$ hosts per learner.
This is a slight simplification of \Textcite{DBLP:journals/eaai/MalialisK15}'s \textquote{online} experiment, choosing fewer teams but remaining as a single server with a fan-out network.
%The algorithm parameters were set at $\gamma=0$ (leading to opportunistic behaviour), $\alpha=0.05$, having linearly annealed $\epsilon=0.2 \rightarrow 0$ by $t=3000$.
%Benign and malicious hosts uploaded between \SIrange{0}{1}{\mega\bit\per\second} and \SIrange{2.5}{6}{\mega\bit\per\second} respectively, and hosts were redrawn at each episode's start with $\operatorname{P}(\mathit{malicious})=0.4$.
%$U_s$  $k \ell mn+2$ \si{\mega\bit\per\second}.
%The performance of each choice of $n$ was averaged over \num{10} episodes of length \num{10000} timesteps (setting each agent's $\wvec{}=\bm{0}$ between episodes).
%Host allocations were generated pseudorandomly to ensure fairness between choices of $n$.
%These parameter choices match those of the original study to enable direct comparison, and are (to the best of our knowledge) arbitrary, but we justify our range of $n$ as capturing increasing scales of host activity.

\begin{figure}
	\centering
	\resizebox{0.9\linewidth}{!}{\begin{tikzpicture}[
	texts/.style = {text=black},
	labeltexts/.style = {text=uofgsandstone},
	treeline/.style = {draw=uofgburgundy},
	treenode/.style = {texts, circle, centered, fill=white, treeline},
	load/.style = {fill=uofgcobalt},
	loadhide/.style = {fill=uofgcobalt!40!white},
	external/.style = {fill=uofgrust},
	externalhide/.style = {fill=uofgrust!40!white},
	hideline/.style = {draw=uofgsandstone!40!white},
	hidenode/.style = {treenode, hideline},
	grow'=right
]
	\node[treenode, label={[texts]above:Server}] (root) {}
	child [treeline] { node [treenode, label={[texts]above:Core}] (sswitch) {}
		child [treeline] { node [treenode, label={[texts]above:Leader}] (teaml) {} 
			child [treeline] { node [treenode, label={[texts]above:Intermediate}] (inter) {}
				child [treeline] { node [treenode, load, label={[texts]above:Agent/Egress}] (agent) {}
					child [treeline] { node [treenode, external] (extern) {}
						child [treeline] { node [treenode, external, label={[texts]above:Host}] (host) {} }
						child [hideline] { node [hidenode, externalhide] (endhost) {} }
					}
				}
				child [hideline] { node [hidenode, loadhide] (endagent) {} }
			}
			child [hideline] { node [hidenode] (endinter) {} }
		}
		child [hideline] { node [hidenode] (endteaml) {} }
		edge from parent
		node[below, labeltexts] {$U_s$}
	};
	
	%\draw[-] (teaml) -- (endteaml);
	\node [labeltexts] (kdots) at ($(teaml)!0.5!(endteaml)$) {$\rvdots$};
	\node [labeltexts, right = -0.1cm of kdots] {$k$};
	\node [labeltexts] (ldots) at ($(inter)!0.5!(endinter)$) {$\rvdots$};
	\node [labeltexts, right = -0.1cm of ldots] {$\ell$};
	\node [labeltexts] (mdots) at ($(agent)!0.5!(endagent)$) {$\rvdots$};
	\node [labeltexts, right = -0.1cm of mdots] {$m$};
	\node [labeltexts] (ndots) at ($(host)!0.5!(endhost)$) {$\rvdots$};
	\node [labeltexts, right = -0.1cm of ndots] {$n$};
\end{tikzpicture}}
	\caption[Tree-structured network topology diagram for evaluating a single-destination network.]{
		Network topology diagram, showing how the server and its core switch's $k$ teams are structured, with $\ell$ intermediate routers per team, connected to $m$ agents which each moderate $n$ hosts beyond a single external switch.
		%	Empty nodes are considered to be internal.
		Red nodes are external, and each blue node hosts an agent.
		\label{fig:marl-topol}
	}
\end{figure}

\subsection{Multiple destinations}
The previous topology allows for direct comparison against the state-of-the-art, and indeed is illustrative of one way in which attack traffic might aggregate in the network.
It is hard, however, to argue its relevance to specific classes of victim or to reason about the interactions it might have with dependent applications.
In contrast, the fat-tree topology~\parencite{DBLP:conf/sigcomm/Al-FaresLV08} sees regular use in real-world data centres and scales well horizontally.
%?? Come up with description of fat-tree (multi-dest) topol.
%?? Why fat tree? regularly appears in modern datacentres.
%?? $k=4$ fat-tree , with one pod hosting two servers $s_0,s_1$.
We use a $k=4$ fat-tree, with one pod hosting two servers $s_0$ and $s_1$.
$n$ external hosts connect through each core switch (where agents are hosted), and communicate with $s_0, s_1$ uniformly randomly.
Both servers host identical services.
We set $n \in \{6, 12, 24, 48\}$ hosts per learner (keeping $N_{\mathit{hosts}}$ identical to each tier of the single-host topology), and restrict $U_{s_0} = U_{s_1} = U_s / 2$.

\subsection{Parameters}
The algorithm parameters were set at $\alpha=0.05$, linearly annealing $\epsilon=$ \num{0.2} $\rightarrow$ 0 by $t=$~\num{3000} in the case of Marl (\num{8000} actions per agent in the \emph{Instant/Guarded} models).

Benign hosts each occupied \qtyrange{0}{1}{\mega\bit\per\second}, and hosts were redrawn at each episode's start with $\operatorname{P}(\mathit{malicious})=$~\num{0.4}.
%The original introduction of this approach to direct-control reinforcement learning as introduced by \textcite{DBLP:journals/eaai/MalialisK15} fails to consider key cases: the absence of a suitable heuristic classifier $g(\cdot)$, disjoint ranges of traffic distribution (i.e., the presence of benign heavy-hitters), the accurate simulation of TCP-like behaviour (and its effects on collateral damage), and high densities of hosts at egress points.
%?? Why? ...
%Of these, the latter two are most deserving of a closer investigation, as they have stronger implications for wide-scale deployment.
%These are important issues, particularly when we consider real-world deployment.
%Heuristic estimates of traffic legitimacy come with computational cost and couple the reward function to the accuracy of the estimator, hosts often show diversity in their own traffic patterns (perhaps being multi-modal), and it is known that TCP is the most used transport protocol for Internet traffic \cite{DBLP:conf/saint/ZhangDJC09}.
%?? NEED TO VERIFY VOLUME OF CONGESTION-AWARE PROTOCOLS
Malicious hosts each sent \qtyrange{2.5}{6}{\mega\bit\per\second} when attacking \gls{acr:udp} traffic, though this was increased to \qtyrange{4}{7}{\mega\bit\per\second} when using \gls{acr:tcp}-like traffic (to meaningfully impact benign flows).
Given $n$ and $\operatorname{P}(\mathit{malicious})$, we see an expected malicious bandwidth \numrange{1.27}{1.87} and \qtyrange{2.03}{2.18}{\times} $U_s$ respectively.
%The expected fraction of $U_s$ consumed by each host is \SI{21.5}{\percent} for $n=2$, and \SI{2.84}{\percent} for $n=16$.
For these choices of $n$ in both topologies, we observe $N_{\mathit{hosts}} \in \left\{24, 48, 96, 192\right\}$, and an expected number of malicious hosts $\mathbb{E}\left[N_{\mathit{attackers}}\right] \in \left\{9.6, 19.2, 38.4, 76.8\right\}$.
For the largest choice of $n$, we see an expected total attack traffic $\mathbb{E}\left[V_{\mathit{attack}}\right] =$ \qtylist{334.05;422.4}{\mega\bit\per\second} for Opus and \gls{acr:http} traffic respectively.

$U_s$ was fixed at $N_{\mathit{hosts}}+2$ \unit{\mega\bit\per\second} (to account for burstiness), and each link had a delay of \qty{10}{\milli\second}.
All links had unbounded capacity, save for each server-switch.
These parameters match those of the original study to enable direct comparison, and many are (to the best of our knowledge) arbitrary, but I justify the range of $n$ as capturing increasing scales of host activity.

\section{Results}\label{sec:the-results-of-doing-so}
We now examine the performance of our two new models (\emph{Instant}, \emph{Guarded}) as compared against existing RL work (\emph{Marl}) and {\color{revisiontext} \cbstart \emph{SPIFFY} under different traffic behaviour and topologies, varying the host-to-learner ratio $n$ and environment.
%?? Probably best just to look at top level stuff, and THEN a simplified comparison for each intended improvement (like banded rewards).
%Additionally, we examine the performance effects of environmental characteristics and potential improvements: negative reinforcement, the case where one agent makes all decisions, and pre-training on individual features.
We present the average rewards for all combinations of these factors in \crefrange{tab:av-vals}{tab:av-ecmp-vals}---providing a rough idea of expected performance, with the highest-performing model in bold and the best RL-based model \cbend underlined.}
Average rewards take into account any portions of time that an agent allows illegal system states.
Several plots augment this, illustrating peak performance or the amount of time which an agent requires to learn.

\begin{table}
	\centering
	\caption{Average reward for combinations of model, host density and traffic class with a single destination.\label{tab:av-vals}}
	
	\cbstart
	\resizebox{0.8\linewidth}{!}{
		\expandableinput tables/marl/tnsm-tree-avg-reward-spiffy.tex
	}
	\cbend
\vspace{-0.25cm}
\end{table}
\begin{table}
	\centering
	\caption{Average reward for combinations of model, host density and traffic class with multiple destinations.\label{tab:av-ecmp-vals}}
	
	\cbstart
	\resizebox{0.8\linewidth}{!}{
		\expandableinput tables/marl/tnsm-ecmp-avg-reward-spiffy.tex
	}
	\cbend
\vspace{-0.5cm}
\end{table}

\subsection{Congestion-unaware traffic}
%\begin{figure}
%	\centering
%	\includegraphics[width=0.9\linewidth]{../plots/udp-2}
%	
%	\caption{
%		Online performance for $n=2$ hosts per egress point when benign traffic is UDP-like.
%		Although Marl++ offers a marked improvement (a peak $\sim$\SI{30}{\percent} more benign traffic arrives unimpeded), SPF significantly underperforms for this relatively simple topology.
%		Non-SPF agents start off reasonably well, slowly learning better policies.
%		\label{fig:udp-2}
%	}
%\end{figure}
%\begin{figure}
%	\centering
%	\includegraphics[width=0.9\linewidth]{../plots/udp-16}
%	
%	\caption{
%		Online performance for $n=16$ hosts per egress point when benign traffic is UDP-like.
%		Marl++ remains marginally ahead of its predecessor, though both have undergone a significant drop in effectiveness.
%		SPF, remarkably, displays performance on par with Marl++ for this more difficult topology.
%		Both of the new models take longer to train, but achieve better peak and average performance than Marl.
%		\label{fig:udp-16}
%	}
%\end{figure}
\begin{figure}
	\centering
	\includegraphics[width=0.75\linewidth]{plots/marl/tnsm-udp-box-separate}
	\vspace{-0.3cm}
	\caption{
		Online performance for Opus benign traffic in a single-destination network, multi-agent mode.
%		Marl++ offers a marked improvement (a peak $\sim$\SI{30}{\percent} more benign traffic arrives unimpeded) at small $n$, and remains marginally ahead of its predecessor by $n=16$, though both have undergone a significant drop in effectiveness.
		\emph{Instant} outperforms Marl for $n \in \{4, 8\}$ (with higher variance), but performs similarly to Marl at $n\in \{2, 16\}$.
		\emph{Guarded} underperforms compared to the other agent designs in this problem variant.
%		SPF, remarkably, slightly outperforms Marl++ (with lower variance) for this more difficult topology despite being worse for smaller $n$.
%		Both of the new models take longer to train, but achieve better peak and average performance than Marl.
%		?? REWORK/MAKE ACCURATE
		\label{fig:udp-tree-box}
	}
\vspace{-0.5cm}
\end{figure}

%\begin{figure}
%	\centering
%	\includegraphics[width=0.95\linewidth]{../plots/tnsm-udp-box-single}
%	
%	\caption{
%		Online performance for Opus benign traffic in a single-destination network, single-agent mode.
%		?? DO I need this?
%		\label{fig:udp-tree-box-single}
%	}
%\end{figure}
%
%\begin{figure}
%	\centering
%	\includegraphics[width=0.95\linewidth]{../plots/tnsm-ecmp-udp-box-separate}
%	
%	\caption{
%		Online performance for Opus benign traffic in a multi-destination network, multi-agent mode.
%		?? DO I need this?
%		\label{fig:udp-ecmp-box}
%	}
%\end{figure}
%
%\begin{figure}
%	\centering
%	\includegraphics[width=0.95\linewidth]{../plots/tnsm-ecmp-udp-box-single}
%	
%	\caption{
%		Online performance for Opus benign traffic in a multi-destination network, single-agent mode.
%		?? DO I need this?
%		\label{fig:udp-ecmp-box-single}
%	}
%\end{figure}

In a single-destination network, we observe that Marl's performance degrades as $n$ increases.
Typically, our \emph{Instant} agent design achieves the best performance in multi-agent mode, having lower collateral damage than the current state-of-the-art, but sharply degrades at low $n$ when agents share experience.
This trend reverses for the \emph{Guarded} model, which improves as $n$ increases and in single-agent mode---when $n\ge4$, the single-agent variant offers consistent improvement.
%Across all choices of $n$, we see that Marl++ exhibits reduced collateral damage compared to Marl, with SPF starting poorly yet becoming more effective for larger $n$ (\cref{tab:av-vals}, \emph{Capped}, UDP).
\Cref{fig:udp-tree-box} shows the preserved traffic in multi-agent mode.
%?? Discuss multi-dest topol once all results available.
When defending multiple destinations, we see a sharp decrease in the effectiveness of all agent designs.
Our new agent designs become more effective as $n$ increases, while Marl's effectiveness is roughly constant (aside from the outlier at $n=12$).
Interestingly, SPIFFY is unable to effectively protect constant bitrate traffic.

\subsection{Congestion-aware traffic}
%\begin{figure}
%	\centering
%	\includegraphics[width=0.9\linewidth]{../plots/tcp-2}
%	
%	\caption{
%		Online performance for $n=2$ hosts per egress point when benign traffic is TCP-like.
%		Marl++ and Marl achieve very similar performance, starting off similarly well without notable improvement over an episode.
%		SPF's performance is disappointingly close to baseline, indicating that it is as useful as having no defence system.
%		\label{fig:tcp-2}
%	}
%\end{figure}
\begin{figure}
%	\vspace{-0.5cm}
	\centering
	\includegraphics[width=0.75\linewidth]{plots/marl/tnsm-tcp-box-single}
	\vspace{-0.3cm}
	\caption{
		Online performance for HTTP benign traffic in a single-destination network, single-agent mode.
		\emph{Instant} and \emph{Guarded} exhibit similar efficacy at $n=2$, protecting less traffic than Marl.
		Only \emph{Guarded}'s performance rapidly increases with $n$, achieving a considerably better median and lower variance than the other models.
		The longer tails of outliers typically indicate the longer training time the new models require---we observe that \emph{Guarded} typically has considerably lower variance once it has converged on a stable policy.
		\label{fig:tcp-tree-box}
	}
\vspace{-0.6cm}
\end{figure}
%\begin{figure}
%	\centering
%	\includegraphics[width=0.95\linewidth]{../plots/tnsm-tcp-box-single}
%	
%	\caption{
%		Online performance for HTTP benign traffic in a single-destination network, single-agent mode.
%		?? DO I need this?
%		\label{fig-tcp-tree-box-single}
%	}
%\end{figure}
\begin{figure}
	\centering
	\vspace{-0.8cm}
	\includegraphics[width=0.75\linewidth]{plots/marl/tnsm-tcp-16-single}
	\vspace{-0.35cm}
	\caption{
		Online performance of standard and single-agent models in a single-destination network with $n=16$ hosts per egress point, HTTP traffic.
		At this level of host density, \emph{Guarded} reaches higher peak performance sooner and is considerably more consistent throughout the episode.
		\emph{Guarded} benefits greatly from information sharing, converging to protect around \SI{75}{\percent} of TCP traffic within \SI{100}{\second}.
		The \emph{Instant} model converges to Marl's level of performance.
%		With a single agent, Marl++ shows worse performance, while SPF improves significantly and continues to learn well past annealing $\epsilon \rightarrow 0$.
		\label{fig:tcp-tree-16}
	}
%\vspace{-0.5cm}
\end{figure}

%\begin{figure}
%	\centering
%	\includegraphics[width=0.95\linewidth]{../plots/tnsm-ecmp-tcp-box-separate}
%	
%	\caption{
%		Online performance for HTTP benign traffic in a multi-destination network, multi-agent mode.
%		?? DO I need this?
%		\label{fig:tcp-ecmp-box}
%	}
%\end{figure}

%\begin{figure}
%	\centering
%	\includegraphics[width=0.95\linewidth]{../plots/tnsm-ecmp-tcp-box-single}
%	
%	\caption{
%		Online performance for HTTP benign traffic in a multi-destination network, single-agent mode.
%		?? DO I need this?
%		\label{fig:tcp-ecmp-box-single}
%	}
%\end{figure}

%\begin{figure}
%	\centering
%	\includegraphics[width=0.95\linewidth]{../plots/tnsm-ecmp-tcp-16-single}
%	
%	\caption{
%		?? Eh
%		\label{fig:tcp-ecmp-16}
%	}
%\end{figure}

\Cref{tab:av-vals} shows that Marl offers a low (though fairly consistent) level of protection for TCP traffic, which the \emph{Instant} agent offers no substantial improvement over.
However, \emph{Guarded} agents offer a remarkable improvement for this class of traffic, particularly when experience can be shared---offering a \SI{2.21}{$\!\times$} improvement over the state-of-the art during training, which is made clearer in \cref{fig:tcp-tree-box}.
\Cref{fig:tcp-tree-16} shows that this model can protect a peak \SI{80}{\percent} of TCP traffic (\SI{2.5}{$\!\times$} improvement) after just \SI{100}{\second}, but also that all of the new models require considerably longer than Marl to learn their best-achieving policy.
%SPF reaches a stronger plateau before \emph{both}, remaining more consistent and appearing to continue learning.
%?? Discuss multi-dest topol once results available.
We observe that the same trends present themselves in the multi-destination topology: \emph{Guarded} remains the best fit for TCP, in both training modes.
Crucially, the rigid tree of learners and teams which define Marl, along with its lack of action granularity, seem to be a poor fit in this environment.
{\color{revisiontext}In both cases, SPIFFY greatly outperforms the RL-based methods.}

%\subsection{Single-agent performance}
%Making all decisions with a single agent is roughly equivalent to having a zero-cost communication channel between each pair of agents, theoretically allowing faster training by giving each agent more experience.
%Curiously, we observe that this often leads to drastically worse policies when used as part of Marl++ for small $n$, but makes SPF a considerably more competitive model---especially for TCP, and as $n$ grows larger (\cref{fig:tcp-16}).
%Single-agent SPF almost consistently outperforms Marl.
%We discuss our conjectures for why this reversal occurs in \cref{sec:discussion}.

\subsection{Increased Attack Volume}\label{sec:results-attack-volume}
To assess the effect of larger volumes of attack traffic, we increase an attacker's output by various factors, supposing $n=16$ with HTTP traffic (\emph{Guarded}, Single); \cref{tab:atk-vol} records the expected rate of attack and average performance.
The initial increase in traffic volume causes the steepest reduction in performance (due to the increased cost of incorrect action), though performance levels out as attack traffic increases.

\begin{table}
	\centering
%	\vspace{-0.1cm}
	\caption{Average reward versus attack volume.\label{tab:atk-vol}}
\resizebox{0.45\linewidth}{!}{
\begin{tabular}{@{}SSS@{}}
	\toprule\multicolumn{1}{c}{Factor} & \multicolumn{1}{c}{$\mathbb{E}\left[V_{\mathit{attack}}\right]$ (\si{\mega\bit\per\second})} & \multicolumn{1}{c}{Reward} \\ \midrule
	1.5 & 633.6 & 0.671 \\
	2.0 & 844.8 & 0.625 \\
	2.5 & 1056.0 & 0.620 \\
	3.0 & 1267.2 & 0.619 \\
	3.5 & 1478.4 & 0.600 \\
	\bottomrule
\end{tabular}
}
\vspace{-0.75cm}
\end{table}

\subsection{Computational Cost}
%?? Consider talking about the execution times of the old MARL approach here? They're real nice (as expected), so we have lots of room to play around with while (hopefully) remaining under the 1ms target time given by \textcite{DBLP:conf/sigcomm/ChenL0L18}.

%Overall, each episode takes around \SI{10}{\minute} to run, while each set of \num{10} requires around \SI{2}{\hour} due to additional set-up/tear-down costs associated with mininet.
Measurements from each of these experiments indicated that the cost of computing any action is typically within \SIrange{80}{100}{\micro\second} per flow.
This is reassuring when measured alongside the insights from other work.
\Textcite{DBLP:conf/sigcomm/ChenL0L18} observe that, ideally, actions must be computed and taken within \SI{1}{\milli\second} to have a meaningful affect on short flows.
%Most flows are short, and flow-size follows a heavy-tailed distribution.
That our starting point falls significantly below this threshold allows us to safely consider more costly actions or larger state spaces, which would typically increase the computational cost.
This cost is constant and independent of network size.
As discussed in \cref{sec:systems-considerations}, we are able to judge 3 flows before this deadline: the difference is primarily accounted for by serialisation/communication delays and single-threaded processing in the Python language.
%?? TODO: update with modern numbers...

\section{Discussion}\label{sec:ddos-discussion}
%?? New angle -- \emph{Guarded} offers substantal improvements for \emph{most} internet traffic according to the stats we know. Is accurate protection of beningn UDP traffic actually more difficult?

%Talk about flaws here, what could go wrong...

%?? Why does SPF only do well sometimes? Model is actually more difficult to learn, so it seems to do best when it has a larger set of decisions to learn from. But, it does worse for TCP?
\paragraph{Model performance}
Of the results presented, \emph{Guarded}'s unpredictable (often worse) starting performance is unexpected, given its far smaller action space.
It's natural to expect that this would make the model easier to learn, but the additional state required appears to make the task \emph{harder}, beyond even the value of choosing a non-zero discount factor---which added forward-planning to explicitly mitigate this effect.
Accordingly, we see that this design performs best (and exhibits considerably lower variance) when agents learn from as much knowledge as possible: high $n$ and single-agent training.
To filter incoming traffic from a source, it must decide to degrade inbound traffic multiple times in a row, reducing the likelihood that a legitimate flow is punished by accident.
Our belief is that \emph{Guarded} is a considerably stronger model for these reasons, and its successes offer strong rationale to consider the best schemes for efficient information sharing.
Paradoxically, \emph{Instant} generally achieves the best performance for \gls{acr:udp} traffic yet actively suffers when trained as a single learner---this may occur due to a roughly even spread of values between disparate actions, due to shared characteristics between legitimate and malicious flows.

%?? May be hard to learn multiple features at once while controlling multiple flows while contending with many more agents, with harder dynamics like TCP. Does this hinder learning in the long run?
Although we have improved upon Marl in both identified problem cases, the improvements are not quite on the order we'd expect for \gls{acr:udp} traffic.
%?? I don’t know if you want to also mention the problem of things getting “stuck” in locally optimal (but globally sub-optimal) policies. I think it’s fairly uncontroversial that an increase in state space would make this more likely. Also, the lack of discounting could play a large role (and more so as the state space becomes more complex…
The most likely explanation is that agents are converging to, and becoming stuck in, locally optimal (but globally sub-optimal) policies.
The increased state space size makes this a more likely occurrence, as does the unclear effect of hyperparameters ($\alpha$, $\gamma$) as we scale up the state space.
I suspect that these difficulties may be exacerbated by the competitive nature of learning that these models embody: agents are learning action values for multiple features simultaneously, taking many actions at once (making it harder to observe the true value of each action), and controlling shared global state.
Although our design does take steps to counteract such effects, these mitigations may not be enough.
Moreover, benign \gls{acr:udp} traffic shares many characteristics with attack traffic, suggesting that more training samples or some unknown feature might aid control, or that it may be worthwhile to extensively pre-train agents non-competitively on each feature using individual flows.

%It is likely that the design of Marl++ 
%?? Need to mention that Marl performs lower than their paper's numbers for TCP...

%Finally, it is crucial to note that the models and techniques presented here are an improvement , this work still trails behind existing (exact) DDoS flow detection mechanisms.
Most importantly, I must state that while the models and techniques presented here are a significant improvement over past \gls{acr:rl}-based work, this strand still trails behind existing (exact) \gls{acr:ddos} flow detection mechanisms where \gls{acr:tcp} traffic is concerned.
The ability to better protect \gls{acr:voip} traffic when compared against one of these approaches is a curious observation, which suggests that other (exact) protocol-agnostic approaches may carry hidden assumptions and is a promising direction for future investigation.
Similar traffic makes up a significant fraction of network load today (\qtyrange{18}{27}{\percent}).
Although this work maps the territory to some extent, there are still more advancements to be made before \gls{acr:rl}-based \gls{acr:ddos} defence is truly competitive.
The benefits at present are, however, substantial.
What this offers above the approaches we discussed earlier are potentially more flexible deployments, low-overhead and fixed-cost decision-making, without requiring active measurement or the network resources and capabilities that the most effective techniques rely upon.
Moreover, our decision making processes are entirely agnostic of the protocol or content of traffic, offering future-proofing against the introduction of new transports.

\paragraph{Security concerns and vulnerability}
Can an agent be flooded with new flows to reduce their ability to make decisions?
One of the risks introduced by our policy update strategy is that so much work can be queued up that an agent is never able to act on some attack flows.
The natural solution is to impose an upper bound on the amount of action computations/policy updates that can be performed before a work list is discarded completely.
This removes the guarantee that all flows will be visited fairly often, but if updates occur regularly then this random sampling may be sufficient to achieve good performance.

Can an attack on the controller can impact our approach?
This question hinges upon whether the deployment environment is a traditional network or is fully \gls{acr:sdn}-enabled---each agent is, in a sense, \emph{a} controller alongside the network's controller.
In a traditional network, only the agents act as controllers, but since they periodically request per-flow data (rather than continuously receiving it) no amount of flows generates more requests or messages to the agent.
More work is generated, but we discuss how to handle this safely above.
Accordingly, agents can never be stalled by request volume: their only remote communication (load measurements) comes from trusted nodes, is highly periodic, and has constant size.
The same logic holds for a fully software-defined network.
Recalling that we do not employ the network's controller to install filtering rules on edge switches, an agent's ability to act is unimpeded.
Thus, the controller is made no more vulnerable than in any other \gls{acr:sdn}.
The only necessary change for such a scenario is that a load measurement which has not been updated (due to a timeout or missed deadline) should be set at $R_t=-1$.
%\begin{figure}
%	\centering
%	\includegraphics[width=0.9\linewidth]{../plots/policy-16-tcp-f5-mean-log}
%	
%	\caption{
%		Mean Marl action values from pre-training on mean IATs ($n=16$).
%		As these are mean values, lighter cells in each row indicate actions \emph{likely} to be taken by an agent's policy.
%		Repeated values originate from the bias tile, which is always active, and indicate regions of the state-space which have yet to be visited---here, this is most of the state space.
%		Agents visibly learn action values for tiles which override the default bias tile's preferences.
%		The measured effectiveness of this feature then suggests that a low-resolution coding over the standard region may, in fact, be a better choice.
%		We see that agents prefer to choose $p=0.0$ in most states, but higher $p$ when IATs are particularly small.
%		\label{fig:intern-16-tcp-iat}
%	}
%\end{figure}
%
%\begin{figure}
%	\centering
%	\includegraphics[width=0.9\linewidth]{../plots/policy-4-tcp-f7-mean}
%	
%	\caption{
%		Mean Marl action values from pre-training on $\Delta$ out rates ($n=4$).
%		A sudden ramp-up of server-to-host traffic is used as a strong indicator of flow legitimacy, while more punishing actions have comparatively higher value as $\Delta$ out rate decreases.
%		Furthermore, we see again that a region of the state space has gone somewhat unexplored---we observed after plotting this visualisation that many decreases in this feature are too small to hit the adjacent tile, which implies a mixed-resolution coding may improve an agent's policy further.
%		\label{fig:intern-16-tcp-something}
%	}
%\end{figure}

%?? Is the state space interpretable? Yes!
\gls{acr:ml} algorithms have earned a reputation for eluding human interpretation, while being vulnerable to evasion and poisoning (\cref{sec:ddn-security,sec:ddn-challenges}).
Given the security risks associated with introducing such techniques, it is natural to be concerned with the interpretability of the models we have proposed.
With the exception of global state, the tile coding parameters we make use of ensure that the set of outputs for each feature we add is relatively enumerable: for $n$ tilings and $c$ tiles per dimension there are $nc^{\dim{f}}$ individual action value vectors per feature $f$ (\num{48} for the new features we introduce, \num{10368} for global state), though considerably more combinations thereof ($c^{n \cdot \dim{f}}$).
%\Cref{fig:intern-16-tcp-iat,fig:intern-16-tcp-something} show how we may visualise the portion of a policy for each feature, and describe what information can be gained from doing so.
Furthermore, system state which is dependent on many signals drawn from across a wide network (such as our global state) is difficult to exert precise control over.
These signals' topological separation, in concert with their burstiness and unpredictability, may have substantial effects on an attacker's capabilities.

\paragraph{Real-world deployment}
Currently, we assume that switches support an extension to OpenFlow to enable remotely installable packet-drop rules, either by running a modified version of \gls{acr:ovs} on commodity hardware at these locations or through custom firmware for egress switches.
Similar functionality could be employed by making use of OpenFlow's meter rules.

Where overheads are concerned, the state space sizes guarantee that an \emph{Instant} agent's policy remains under \qty{520}{\kibi\byte}, although in practice our sparse representation typically leads to far smaller policies: $\sim$\qty{17.8}{\kibi\byte} from our experiments.
\emph{Guarded} policies are \SI{30}{\percent} of this size.
As we have described earlier, action updates require a constant number of floating point operations.
%---\num{160} floating point additions and \num{32} multiplications per update of $\wvec{}$ with per-tile updates, above the \num{160} additions required to choose an action.
The vast majority of these operations can be vectorised trivially, if such hardware is present.
%Action computation for \emph{Guarded} agents is cheaper still, requiring only \num{48} additions per action.
Beyond this, we require that egress switches are capable of co-hosting an agent (i.e., through \gls{acr:nfv}), with the necessary hardware to support this.
%We believe that it may be possible to implement similar behaviour on standard commodity switches through application of \emph{programmable data planes} \cite{DBLP:conf/ancs/JouetP17}.

%?? Incorrect, pickling...
%?? deployment/system guidelines, architecture, and overhead/footprint of doing this
%?? Mention: probably fewer state measurements than in our emulations, but longer measurements means less noisy (so probably more accurate)
Gathering and transmission of load/flow statistics would be difficult to perform as often as an emulated environment allows, without inadvertently affecting host traffic.
However, the measurements acquired in such a scenario are likely to be less noisy (by being collected over longer periods of time), which could aid training.
The main bottlenecks are likely in forwarding the load measurements from various aggregation points (which can be made more efficient through multicast) and in running some estimator $\operatorname{g}(\cdot)$ to condition the reward function.
%?? Potential for dividing up pre-feature training across different agents, or train like this locally (sets of flows trained by 1 sub-model).
%?? Different global state per-agent? Seems this must be locally trained, just make it (everything up the chain to key destinations).
We expect that agents will be able to share policies for all features, which may help to offset the reduced rate of incoming experience.
Regardless, it will take longer to achieve enough state-state transitions to converge on a good policy.

One limit of \gls{acr:sdn}-capable hardware is that OpenFlow rules occupy \qty{6}{\times} the space of standard rules---commercial switches only have \gls{acr:tcam} space for \numrange{2000}{20000} rules~\parencite{DBLP:journals/comsur/NguyenSBT16}.
This approach consumes a rule for each active flow (the host density), and by the end of an experiment a switch can accrue around \num{900} rules.
While we use a default fallback action to maintain connectivity, eviction of high-value decisions which filter high-bandwidth attackers poses a significant risk.
Given that most flows are small, with the majority of bytes coming from a few ``heavy-hitters''~\parencite{DBLP:journals/ccr/PanBPS03}, it may suffice to only apply \gls{acr:rl}-based analysis to larger flows.
%?? What could we do in the meantime? I propose manually assigning elephant flows a high importance rule, while increasing the importance of "block" decisions even higher.
OpenFlow rules have an \emph{importance}, controlling which rules may be evicted by a new entry (preventing entries from evicting those with higher importance).
If an agent is to act on all flows, a solution is to assign an importance of 0 to mice flows, 1 to elephant flows, and 2 to total filtering (leaving agents to time out and remove elephant flow rules to prevent bloat).
Given the high churn and prevalence of mice flows, eviction here is most likely to affect flows which are complete.
In both cases, extra rules can be made available by upgrading rules which completely filter a flow into upstream blackholing as in collaborative approaches~\parencite{DBLP:conf/acsac/RamanathanMYZ18}, having the agent remove this rule once blackholing is active.

%\section{Related Work}\label{sec:related-work}
%%?? Try and compare my work here when possible?

\fakepara{DDoS Prevention}
\Textcite{DBLP:conf/lcn/BragaMP10} examine the detection of flooding DDoS attacks through \emph{self-organising maps}, using SDN to gather statistics effectively.
Many of their features aren't overly relevant, as their focus is not active defence or discovering \emph{which} hosts contribute to an attack.
%?? Actually talk about Marl (???) to appease reviewer \#1.
The closest available approach within this field is that of \textcite{DBLP:journals/eaai/MalialisK15} (whom we have positioned our work against), and their contribution in applying RL to the task of intrusion prevention is significant: their work helps to show the viability of live, adaptive, feedback-loop-like control of the network to detect and prevent DDoS attacks.
They create a tree overlay topology (subdivided into teams), where each agent applies packet drop to \emph{all} flows inbound to a protected server.
%?? Recap their flaws, since they've been cut form every other aspect.
Our results show that their technique underperforms at high host density and when congestion-aware traffic dominates---that their results do not demonstrate this suggests an evaluation driven purely by traces (rather than live application dynamics).

\emph{SPIFFY} \cite{DBLP:conf/ndss/KangGS16} aims to remedy transit-link attacks by observing how flows from each source respond to a sudden increase in available bandwidth.
\Citeauthor{DBLP:conf/ndss/KangGS16} realise that bots participating in an attack are often unable to match this bandwidth expansion (having already saturated the capacity of their outbound links), while legitimate flows typically speed up to match the new fair-share rate.
%Attackers must either be detected or reduce the throughput of each bot, increasing the cost of launching an attack.
%Unlike our approach (and due to the class of attacks it is designed to defend against), SPIFFY is intended to be deployed within ASes, although .
A weakness of their approach is that computing a route to measure bandwidth expansion on real networks can be costly (up to \SI{14}{\second} for the Cogent topology), and that the low expansion factors in real network can require more ``rounds'' of filtering.
By contrast, our approach takes a constant time to compute an action for a flow regardless of topology size.
Their assumptions about traffic response to such bandwidth expansion do not hold for constant bitrate flows (e.g., VoIP) and may not extend to HTTP DASH flows, both of which make up a sizeable proportion of network traffic.

\emph{Athena} \cite{DBLP:conf/dsn/LeeKSPY17} is a generalised SDN framework for intrusion detection, but has shown the use of a \emph{k-nearest neighbours} classifier to detect individual attack flows.
Although heavyweight (and proven to be effective compared with \textcite{DBLP:conf/lcn/BragaMP10}), their comparison against SPIFFY lacks the quantitative evidence required to understand how the system compares.
\Textcite{DBLP:conf/sp/SmithS18} use AS-level routing to tackle both transit-link and flooding-based attacks.
This view is taken due to the perceived cost of per-stream classification and inherent sensitivity to adversarial examples.
The approach is creative, relying upon BGP \emph{fraudulent route reverse poisoning} to preserve traffic to a target AS, but unlike SPIFFY the approach doesn't actually \emph{remove} the congestion.
Because of this, flooding-based attacks aren't fully alleviated.

%?? Abuses of RL 
\fakepara{RL in Networks}
Earnest, well-considered application of RL towards the challenge of intrusion prevention has seen comparatively little examination.
Past work treats the paradigm as a traditional classifier for anomaly detection \cite{shamshirband2014anomaly} and DDoS prevention \cite{DBLP:conf/mates/ServinK08}.
Given that the main strengths of RL techniques are the ability to control ongoing interaction and adapt by observing the concrete effects of actions, such works don't apply the rich literature on the subject to its fullest potential.

For categorising how RL fits into solving problems, we label works as direct- or indirect-control RL.
A \emph{direct-control} RL problem is one where the RL agent(s) learn optimal control over a set of actions as the \emph{primary} defence or decision-maker---requiring measurements, reward functions and action sets tailored for this purpose.
%We feel there is a shortage of work in this category at present, at least in the field of networks.
To date, the best-fitting example we have encountered is that of \textcite{DBLP:journals/eaai/MalialisK15}.
An \emph{indirect-control} RL problem is one where agents act in service to \emph{another technique} responsible for decision-making, optimising or generalising aspects of its operation beyond that of hand-coded heuristics.
A past example includes learning when best to share knowledge between \emph{hidden Markov model} anomaly detectors \cite{DBLP:conf/paisi/XuSH07}.
%The position of this work is weakened by its reliance on the problematic `DARPA99' dataset \cite{DARPA-IDD, DBLP:conf/cisda/TavallaeeBLG09, DBLP:conf/sp/SommerP10}, but the idea itself is well-treated and this acts as a driver for improvements in this direction.
This work is weakened by its reliance on the problematic `DARPA99' dataset \cite{DBLP:conf/sp/SommerP10}, but the idea itself is well-treated.
Outside of intrusion detection, there has been growing interest in the use of RL in data-driven networking, such as for intra-AS route optimisation \cite{DBLP:conf/hotnets/ValadarskySST17} and resource-constrained process allocation \cite{DBLP:conf/hotnets/MaoAMK16}.
\textcite{DBLP:conf/sigcomm/MaoNA17} employ client-side observations of network state and video performance with RL to optimise bitrate selection for multimedia streaming.
\emph{AuTO} \cite{DBLP:conf/sigcomm/ChenL0L18} employs deep RL to perform traffic optimisation.
Crucially, they find that the vast majority of flows are short-lived, requiring effective decisions in less than a millisecond.
To overcome the high latency of action computation via a neural network, two agents are trained, handling aspects of short and long flows respectively.
The first learns to optimise the flow size thresholds to demarcate long and short flows; these short flows are routed by ECMP.
The second agent makes bespoke decisions about routing, prioritisation etc.\ for each of the remaining long flows.


%\section{Conclusions and Future Work}
\section{Summary}\label{sec:ddos-summary}
Through this chapter, we have discussed reinforcement learning and how it can be used to approach the task of \gls{acr:ddos} prevention, lending credence to one of the claims in the initial thesis statement: \superrecallthesis{1}.
The key to doing so was to study the dynamics of the network itself---its behaviours, and realistic recreations thereof---to detect operational flaws in existing \gls{acr:rl} and algorithmic works.
In turn, I designed different action models built upon a shared (and justified) model: making decisions on a per-flow or per-source basis, and relying upon learnt policies to differentiate congestion-aware and -unaware flows that methods like SPIFFY ignore.

First, we covered a large problem in modern networks: the ever-present threat of \gls{acr:ddos} attacks---and how Internet traffic characteristics make its solution more difficult.
We identified weaknesses in past remedies offered by the community, recommending instead an \gls{acr:rl} agent design which acts per flow, and have outlined the algorithmic and engineering choices needed to make its deployment feasible.
Supporting this, we've examined the presented feature space in depth, offering quantitative and qualitative justification for each choice, while also expanding the global state in past Marl approaches to support arbitrary network topologies.
Using simpler tile-coded policies, we have also covered how decision narrowings and per-tile updates allowed faster convergence---and independently developed methods for coalescing state which have become more common in tasks with long inference times as the field has bloomed.
We have examined the \emph{Instant} and \emph{Guarded} action models, integrating various degrees of domain expertise with \gls{acr:rl} agents.
To make real-world deployment possible in the face of obviously adversarial inputs, we have seen how it is essential to consider rate limiting (and probabilistic) strategies like \gls{acr:trs} scheduling, and have presented a \gls{acr:vnf}- and \gls{acr:sdn}-backed system architecture.
By empirical evaluation, we've seen that these new agent designs advance the state of the art in \gls{acr:rl}-based \gls{acr:ddos} prevention, with \emph{Guarded} agents showing the most promise for future evaluation.

%We hope it is clear that reinforcement learning holds promise and can inspire further innovation.
%It allows us to offer distinct advantages above existing works, such as protocol-agnostic DDoS flow detection, flexible deployment, and automatically learnt low-overhead decision-making---without requiring many of the network resources or capabilities that other techniques rely upon.
%It's hoped that more research in this direction will open the door to works which \emph{respect the complexity of the network}; evolving topologies, natural change in traffic and protocol distributions, and the mutation of attacks.

While this adds another positive note to the score of \gls{acr:ddn} use cases seen throughout \cref{sec:use-cases}, what I must stress is that this chapter emphasises the value of \emph{co-design} and true subject-matter expertise.
Networks in particular are complex, and controlled elements respond to an agent's action in ways which trace-based evaluation cannot capture---hence my disdain in the frontmatter of \cref{sec:use-cases}.
This builds on the general advice of \cref{sec:ddn-use-takeaways}: better modelling, simulation, and understanding of the environment \emph{led to better designs for their control}.
%?? Justifies trace-hatred \cref{sec:use-cases,sec:ddn-use-takeaways}, need for deep expertise
%?? actually really emph this wrt co-design: better modelling of the environment led to better designs!
This foundation is crucial, as it falls to us to derive the \emph{mechanisms} of control: state and action spaces, reward functions, interaction and measurement models, and similar aspects of agent or classifier design.
Learnt \gls{acr:ddn} policies work well at optimising within these constraints that we set.
The final takeaway is that \gls{acr:ddn} solutions, and how we evaluate them, \emph{must respect the complexity of the network}; evolving topologies, natural change and diversity in traffic and protocol distributions, as well as the mutation of attacks and the wider problem space.

%While we can train successful policies, \gls{acr:ddn} cannot itself derive the \emph{mechanisms} of control: action models, reward functions, and the state which they should operate on.
%Learnt policies and parameters operate as well as they can \emph{within the framework we give them}, and generally succeed at so doing.

%The work presented in this chapter considers how online \gls{acr:rl} can be used to defend networks from volumetric \gls{acr:ddos} attacks, agnostic of the protocols of carried traffic, and is based upon \citetitle{DBLP:journals/tnsm/SimpsonRP20}~\parencite{DBLP:journals/tnsm/SimpsonRP20}.
%I first reiterate the motivation for the task and \gls{acr:rl} as a solution (\cref{sec:ddos-motivation}), and define the threat model for attackers with respect to known \gls{acr:ddos} attack methods and the security context of \gls{acr:ml} (\cref{sec:ddos-threat}).
%\Cref{sec:ddos-mitigation-with-per-flow-reinforcement-learning} then outlines the design and rationale behind new agent designs built to improve on the failings of past \gls{acr:rl} works, by making decisions on a per-flow or per-source basis.
%This includes algorithmic modifications to learn from and control many traces simultaneously, achieve faster convergence by per-tile updates, and to better learn form individual features.
%I describe a feature space built on a mixture of global and local state, reward functions tailored to different attack classes, and contribute two action models and their risks (\emph{Instant} and \emph{Guarded}).
%The \emph{Guarded} model is inspired by past work on algorithmic \gls{acr:ddos} prevention, as an example of how the integration of domain-specific knowledge can lead to more effective \gls{acr:rl} agents in shorter timescales.
%\Cref{sec:ddos-architecture} then describes how state measurement and installation of actions could be managed in an \gls{acr:sdn} deployment.
%To determine \emph{which} per-flow features are worth using for \gls{acr:ddos} control and mitigation, I then present qualitative and quantitative analysis of a large selection of these metrics for different agent designs on varied protected traffic types (\cref{sec:rethinking-the-state-space}).
%\Cref{sec:a-new-normal} then details my implementation of reactive simulations of \gls{acr:http} and \gls{acr:voip} web-server traffic, designed to test system characteristics that packet trace playback fails to capture.
%\Cref{sec:ddos-evaluation,sec:the-results-of-doing-so} then describe and show empirical performance measurements of the two new agent designs against existing \gls{acr:rl} \gls{acr:ddos} techniques, and algorithmic works via \emph{SPIFFY}~\parencite{DBLP:conf/ndss/KangGS16}, reuniting two divergent strands of research and grounding the study of \gls{acr:rl}-based \gls{acr:ddos} defences.
%I conclude with some discussion on the significance of the results and wider security implications of this solution in particular (\cref{sec:ddos-discussion}), and summarise in \cref{sec:ddos-summary}.

