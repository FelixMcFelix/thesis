
%In establishing...
%
%?? How will I structure this?
%?? Motivation -> Model -> Results?
%?? OR Use the results of the last section to springboard into here?

%From what we have seen, it is difficult (or impossible) for trace-based or numerical simulations to correctly capture certain dynamics without an extraordinary amount of care or consideration.
%As it turns out, 
%Our goal is to briefly describe an environment which tests \emph{specific} behaviours to examine the \emph{specific} problems which have arisen during our testing of past approaches.
We contribute network models built around live testing of reactive TCP and UDP traffic in an SDN-enabled environment, which is adaptable to arbitrary topologies, with an explicit focus on preserving their real-time dynamics in a way that trace-based evaluation cannot.
First and foremost, we are interested in representative load and packet inter-arrival characteristics and in how these characteristics evolve in response to actions.
We introduce these models because we are interested in capturing interactive, correlated back-and-forth exchanges associated with live HTTP traffic; mainly because of the particular interactions between the application-level dynamics, congestion awareness at the transport level and the nature of control signal used.
%Naturally, this model is not perfect or representative for all traffic, yet it captures some of the behaviour which we expect will plague most legitimate TCP flows.
%If need be, we expect the frequency or distribution of requests could be conditioned to match observations of real-world access patterns.

%?? ANGLE: set up an environment to test \emph{specific} behaviours to examine \emph{specific} problems in past work. I make no claims that it is perfect or representative for all traffic, just for this (likely common) behaviour which I expect to plague almost all legit TCP flows.

%?? Existing sims used for testing such applications reliant on traces, or not sophisticated enough to capture interactive, back-and-forth (correlated) behaviours---possibly discarded as second-hand effects by past work when these are so crucial given user traffic patterns (and the nature of the control signal we choose to enact).

%?? Remember, the motivation is clear. We don't care so much that it is "representative" wrt a specific deployment location or network type. The whole purpose of this is that we aim to test specific behaviour which traces cannot replicate (i.e., correlated back-and-forth, dynamics introduced to congestion-aware protocols, ...)
%?? If we need to, we can condition the distribution of requests according to statistics mined from an existing trace if reviewer number 2 needs that extra push to be convinced.

\subsection{Network Design}
We make use of a fully software-defined network, built using OpenFlow-aware switches in mininet alongside a controller based on \emph{Ryu} \cite{ryu}.
All internal routers are primed with knowledge of the shortest path to each internal host, while new inbound flows register the ``way back'' for each hop used, to ensure consistent traffic conditions for each flow.
If several ports offer different (equal-length) paths to a destination, a consistent random port is chosen from the flow-hash by an OpenFlow \emph{Group action} (in \emph{select} mode).
If such information is lost, perhaps expiring due to inactivity, it suffices to forward an outbound packet on a random outbound port, as we assume that any external IP is reachable through any of the test network's egress ports (i.e., that it is not connected to any stub ASes).
The controller is also responsible for computing how switches respond to ARP requests: this need arises due to the reliance upon Linux's networking stack for live applications, and wouldn't need to be considered for trace-based evaluation.
%We make further use of the topology presented earlier (\cref{sec:topology}), noting that our architecture allows us to trivially extend and modify this if required.

\subsection{TCP (HTTP) Traffic Model}\label{sec:tcp-http-traffic-model}
%?? Legitimate traffic: TCP traffic (HTTP clients downloading web pages, dependent resources and files) with a mixture of lifetimes for each request.
To model legitimate TCP traffic, server nodes run an nginx v1.10.3 HTTP daemon, serving statically generated web pages alongside various large files and binaries.
Benign hosts run a simple libcurl-based application written in Rust, repeatedly requesting resources from the server.
Hosts and clients both use TCP Cubic \cite{rfc8312}.
Each host's download rate is limited to match the maximum bandwidth assigned to it, and requests several random files known to exist within a website, followed by any dependent resources for each (stylesheets, images, etc.) as a browser might.
On completion, a host changes its IP to generate separate statistics per-flow, while minimising downtime.
This presents a balanced distribution of flow duration and size, with large files included to model elephant flows.

\subsection{UDP (Opus/VoIP) Traffic Model}\label{sec:udp-opus-traffic-model}
VoIP traffic exhibits very different characteristics to the above model; packet arrivals are highly periodic due to real-time requirements, flows have a constant bitrate, and do not react substantially to lost packets.
Interestingly, DDoS attack traffic is known to share many of these characteristics, offering an interesting detection problem.
We present a VoIP traffic model\footnote{\url{https://github.com/FelixMcFelix/opus-voip-traffic}} based on Discord\footnote{\url{https://discord.gg}}, a freely-available messaging and VoIP platform geared toward gaming communities.
We chose Discord as our prototype due to its publicly documented API, many open source bot frameworks, large user base, and due to the lack of models for Opus-encoded traffic.
%?? Highly periodic, CBR

Hosts send RTP traffic with Salsa20 encrypted payloads---\SI{20}{\milli\second} audio frames at \SI{96}{\kilo\bit\per\second}.
We generate similar traffic at hosts by replaying anonymised traces gathered in general use and tabletop RPG servers; each trace contains only the size of each audio payload, entries denoting missed packets, and the duration of silent periods.
We trim these silent periods to a maximum \SI{5}{\second} due to the lengthy talk/silence bursts introduced by users in RPG servers, and estimate the size of missed packets by taking an exponentially-weighted moving average over known sizes.
Hosts punctuate audio frames with a 4-byte keepalive every \SI{5}{\second}.
All traffic passes over a central server which groups hosts into rooms, and is forwarded to other participants; we do not replicate pre-call Websocket traffic which would be used for authentication.
There is no peer-to-peer traffic---the server acts as a TURN relay for all hosts.
%?? Reflective factor among \emph{authenticated hosts}.
We find that each flow occupies an expected \SI{52.4}{\kilo\bit\per\second} upstream bandwidth.
To match the target upload rate assigned to each host, it runs enough individual sessions to meet the target data rate.

%?? Malicious traffic: UDP flood traffic (hping3, MTU-size packets, ). Why not min-size packets? Because the traffic generator gets in a horrible rut if I do so...
\subsection{Attack Traffic Model}\label{sec:attack-traffic-model}
Malicious traffic is generated by use of the \emph{hping3} program, generating UDP-flood traffic targeting random ports.
We configure each instance of \emph{hping3} to generate ethernet MTU-sized packets (\SI{1500}{\byte}) with a random source and destination port towards a target server, and configure the output rate $r$ (in \si{\mega\bit\per\second}) by setting the inter-arrival time $t_{\mathit{attack}}=\frac{1500 \cdot 8}{r\cdot10^6}$.
This fulfils certain characteristics of many types of amplification DDoS traffic: it is congestion-unaware \cite{DBLP:conf/ndss/Rossow14}, and packets are larger than the minimum frame size and identically-sized (e.g., NTP amplification traffic is fragmented at the application layer into \SI{482}{\byte} chunks \cite{cisco-ntp-amp}).
We differ from NTP amplification in frame size so that inter-arrival times are larger, to keep emulation of the network feasible at high traffic rates.