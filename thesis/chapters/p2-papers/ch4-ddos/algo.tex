Our main hypothesis is that the best method for advancing past the current shortcomings of RL-based DDoS mitigation is to design agents such that filtering decisions are computed per flow.
However, these alterations must account for computational constraints imposed by the deployment environment---the amount of flows passing over an agent is unbounded.
We describe and justify our approach, our algorithmic improvements, and present two action models, one of which draws on domain knowledge introduced by SPIFFY \cite{DBLP:conf/ndss/KangGS16}.

\subsection{System Design and Assumptions}
A deployment environment is a network with a set of \emph{ingress/egress points} from its domain of control, through which traffic can flow, and a set of protected \emph{destination nodes}.
These nodes may be services, servers, or in the case of Autonomous Systems (ASes) and transit networks, egress points leading to other networks.
{\color{revisiontext}\cbstart Agents are co-located with each egress switch (i.e., $k$ ingress points from other ASes require $k$ agents), all employ the same action model/design}, and control the proportion of upstream packets from each external host to discard.
Each destination node $s$ has a maximum capacity, \cbend $U_s$.

We assume that the deployment environment is a moderately complex software-defined network, because the paradigm offers features which can directly benefit RL agents acting within.
The OpenFlow protocol allows a controller (or other authorised hosts) to install complex actions, forwarding rules and logic into a switch at runtime.
Furthermore, networks of this kind more naturally enable the future use of \emph{network function virtualisation}, a technology which could allow relocation and easy installation of learners (e.g., as examined by \textcite{TNSM-that-guys-paper}).
Agents communicate with their co-hosted OpenFlow-enabled switches---running a modified version of \emph{Open vSwitch} (OVS) \cite{open-vswitch}---to install probabilistic packet-drop rules.

Our system design applies to both software-defined and traditional networks of arbitrary shape and size.
Only the ingress/egress nodes from a network need to be OpenFlow-enabled, as it is advantageous to perform filtering as close to a flow's source as possible.
In a traditional network, each agent has exclusive control over its switch's tables.
In a fully software-defined network, these agents require exclusive control over the first table, forwarding legitimate packets to subsequent tables managed by the network's controller.
The main difference is that a traditional network needs this additional hardware, and does not allow an operator to dynamically determine where the ``edge'' of their network lies through vNF relocation.

\subsection{Algorithm}
To make decisions cheaply and at low latency, we use \emph{semi-gradient Sarsa with tile coding} as described in \cref{eqn:sg-sarsa} and \cref{sec:reinforcement-learning}, rather than using neural networks or more complicated function approximators.
Exploration is introduced via $\epsilon$-greedy action selection, linearly annealing $\epsilon$ to 0 over time.
Each agent has its own internal parameter vector $\wvec{}$, and agents do not share their weight vector updates with one another (but may share experience and traces with one another).

Although the choice of a classical RL method likely brings lower theoretical performance, there are significant reasons to favour such methods; these include lower latency decision-making, lower energy usage, reduced model complexity (and training time), the availability of necessary hardware, and simpler decision boundaries.
This aligns with our goal of quick online learning, and faster adaptation to aggregate changes in traffic without introducing dedicated tensor processing hardware to networks.
Simpler decision boundaries reduce the risk of overfitting and unexpected behaviour, and we expect that the simplicity of tile-based policy computation will also greatly aid interpretability of anomalous action choices.

When choosing a learning algorithm, we compared against Q-learning as well as methods based on \emph{eligibility traces} such as Watkins's $\operatorname{Q}(\lambda)$ \cite[pp. 312--314]{RL2E} and $\operatorname{Sarsa}(\lambda)$ \cite[pp. 305]{RL2E}.
Preliminary experiments found that Sarsa offered the best performance and behaviour.

\subsubsection{Action rate}
We adapt the algorithm to prioritise rapid response to changes in network state and to visit as many state-to-state transitions as possible for effective learning.
To this end, we allow agents to make many decisions per timestep.
We maintain the last state-action pair associated with each source IP and destination, and calculate any actions for the flows which still exist.
Finally, we update $\wvec{}$ using each available trace and the reward signal from the relevant destination.
As exploration still occurs for each action, this approach reduces $\epsilon$ multiple notches every timestep.
In turn, we increase the annealing window for $\epsilon$ by a factor of \num{2.67} so as to preserve exploration over time, by accounting for the greater volume of decisions being made.

\subsubsection{Per-tile updates}
While the standard formulation of \cref{eqn:sg-sarsa} updates the value of all tiles identically (by a scalar $\alpha \delta_t$), we found it more effective to compute a different temporal difference value \emph{for each tiling}.
While we make use of the sum of all tiles' action value estimates when making decisions, each tiling is updated using only its own contribution, allowing us to set $\alpha$ to a higher value without divergence.
A crucial observation is that value updates to each tile can move by different values in different directions, converging on effective estimates sooner.

\subsubsection{Decision narrowings}
When learning control on the basis of a high-dimensional, tile-coded state space, assignment of credit for each decision is difficult (because all tiles have identical gradient).
To combat this, with probability $\epsilon$ an agent will mark a flow as being governed by a subset of the state space for the next 5 decisions.
Each agent chooses actions on that source/destination pair using one element of local state, the global state, and the bias tile---we include the latter two to strike a balance between and accuracy and correct credit assignment.

\subsection{Feature Space}\label{sec:feature-space}

\begin{figure}
	\centering
	\resizebox{0.65\linewidth}{!}{
		\begin{tikzpicture}[
		texts/.style = {text=black},
		labeltexts/.style = {text=gray},
		treeline/.style = {draw=uofgburgundy},
		treenode/.style = {texts, circle, centered, fill=white, treeline},
		load/.style = {fill=uofgcobalt},
		loadline/.style = {draw=uofgcobalt, line width=0.75mm},
		loadhide/.style = {fill=uofgcobalt!40!white},
		external/.style = {fill=uofgrust},
		externalhide/.style = {fill=uofgrust!40!white},
		hideline/.style = {draw=uofgsandstone!40!white},
		hidenode/.style = {treenode, hideline},
		grow'=right
		]
		\node [treenode, loadhide, label={[texts]above:Agent 1}] (mainagent) {};
		\node [treenode, loadhide, right = of mainagent] (inner0) {};
		\node [treenode, loadhide, above right = 0.2cm and 1cm of inner0] (inner1) {};
		\node [hidenode, below right = 0.2cm and 1cm of inner0] (inner2) {};
		\node [treenode, loadhide, below right = 0.2cm and 1cm of inner1] (inner3) {};
		\node [treenode, loadhide, right = of inner3] (inner4) {};
		\node [treenode, loadhide, above right = 0.2cm and 1cm of inner4, label={[texts]above:$s_0$}] (s0) {};
		\node [hidenode, below right = 0.2cm and 1cm of inner4, label={[labeltexts]above:$s_1$}] (s1) {};
		
		\node [hidenode, above left = 0.2cm and 0.5cm of inner1, label={[labeltexts]above:Agent 2}] (agent2) {};
		\node [hidenode, below left = 0.2cm and 0.5cm of inner2, label={[labeltexts]below:Agent 3}] (agent3) {};
		
		\draw[hideline, -] ($ (mainagent) + (-1,0) $) -- (mainagent);
		\draw[hideline, -] ($ (agent2) + (-1,0) $) -- (agent2);
		\draw[hideline, -] ($ (agent3) + (-1,0) $) -- (agent3);
		\draw[hideline, -] (inner1) -- (agent2);
		\draw[hideline, -] (inner2) -- (agent3);
		
		\draw[loadline, -] (mainagent) -- (inner0);
		\draw[loadline, -] (inner0) -- (inner1);
		\draw[hideline, -] (inner0) -- (inner2);
		\draw[-] (inner1) -- (inner3);
		\draw[hideline, -] (inner2) -- (inner3);
		\draw[loadline, -] (inner3) -- (inner4);
		\draw[loadline, -] (inner4) -- node [texts, above] {$U_{s_0}$} (s0) ;
		\draw[hideline, -] (inner4) -- node [labeltexts, below] {$U_{s_1}$} (s1);
		
		\end{tikzpicture}
	}
\vspace{-0.25cm}
	\caption{
		Global state selection for a flow between an external host and server $s_0$ which passes over Agent 1.
		All nodes in the path taken through the defended network are filled in blue, and all link load measurements which are chosen for action computation are indicated with a thick blue line.
		\label{fig:global-state-path}
	}
\vspace{-0.5cm}
\end{figure}

Our state space combines elements of global state (network link load observations) with per-flow measurements.
Each is tile-coded with 8 tilings and 6 tiles per dimension, using the windows described in \cref{tab:codings}.

%?? How is global state acquired? See \cref{fig:global-state-path}. Why take paths in the way that we do? Mention deterministic ECMP routing etc...
Global state is a vector of load values in $\mathbb{R}^4$ (\si{\mega\bit\per\second}) depending upon the bandwidth measurements regularly received from monitors in the environment.
For any flow, an agent then computes the path it would take through the network.
The incoming load recorded along the first hop, last hop, and tertiles of the path may then be tile-coded together.
In the event that the path from an agent to its destination is shorter than 4 hops, we duplicate (in order of preference) the load measurement of a middle hop or the last hop.
\Cref{fig:global-state-path} illustrates the process.

We build global state in this way to offer compatibility with multipath, multi-destination networks, offering support for diverse deployment environments from endpoint servers to transit ASes.
Computing the path from agent to destination is not computationally expensive.
Multipath routing is often fast since typical \emph{equal-cost multipath} (ECMP) routing algorithms simply hash a packet's flow key, and are deterministic to provide consistent quality-of-service to hosts.

%?? Path computation fast and efficient due to deterministic routing based on hashes (ECMP)
%?? Works for any arbitrary topology, even 

We describe and analyse each of the per-flow features included in the state vector throughout \cref{sec:rethinking-the-state-space}.
Each feature is tiled separately, with the exception of packet in/out count (per-window and total), mean in/out packet size, and $\Delta$ in/out rate, which are combined with the last action taken.
Rather than having the network push the data to an agent, the agent requests this information about active flows periodically to isolate it from non-control-plane traffic and to eliminate the risk of resource exhaustion by excessive requests.

\subsection{Reward Function}

%?? Say ``we take $R_G$ from M and K''.
%
%?? Reward at each destination.

\newcommand{\arrload}[2]{\operatorname{load}^{#2}_{t}(#1)}
\newcommand{\uload}[1]{\arrload{#1}{\uparrow}}
\newcommand{\dload}[1]{\arrload{#1}{\downarrow}}
\newcommand{\bload}[1]{\arrload{#1}{\updownarrow}}
\newcommand{\cond}[2]{\operatorname{c}_{#1,t}#2}
%\fakepara{Reward function directionality}
%The reward functions, as defined, do not take traffic direction into account.
%We redefine these to identify overload states using both upstream and downstream loads, while allowing customisation of which direction is chosen for protection.
Each destination node $s$ generates a reward signal, $R_{s,t}$, at every timestep $t$.
Assume, for now, that each destination has access to some classification function $g(\cdot)$ which estimates the volume of legitimate traffic received, and expects to receive $\mathit{traffic}_s$.
Denoting the upstream, downstream and combined loads $\uload{s}, \dload{s}, \bload{s}$ at this node:

\begin{subequations}
	\vspace{-0.5cm}
	\newcommand{\load}[1]{\operatorname{load}_{t}(#1)}
	\begin{gather}
	\cond{s} = [\max(\uload{s}, \dload{s}) > U_s],\\
	R_{s,t} = (1 - \cond{s}) \frac{g(\arrload{s}{-})}{\mathit{traffic}_s} - \cond{s},\label{eqn:reward-rtx}
	\end{gather}
	replacing $\arrload{s}{-}$ in \cref{eqn:reward} with whichever directional load is prioritised according to the traffic characteristics of the deployment environment, where $\cond{s}$ represents the ``overloaded'' condition at destination $s$.\label{eqn:reward}
\end{subequations}
We choose $\uload{\cdot}$ for our UDP-based models and $\dload{\cdot}$ for HTTP, though we expect that $\bload{\cdot}$ would be the most suitable for general deployment or heterogeneous traffic patterns.
These choices reflect where the bulk of transmitted bytes in each traffic model are observed (and the lack of this knowledge in the general case).

While our use and definition of $g(\cdot)$ appears nebulous, there are many ways to infer this quantity in practice.
End-host servers may use canary flows or other active measurements, or employ existing quality-of-experience metrics in the case of VoIP services such as lost packets, reorderings, and jitter.
ASes and transit networks may make use of reports received from downstream networks, i.e. over the \emph{DDoS Open Threat Signalling} (DOTS) protocol \cite{ietf-dots-use-cases-17}.
Even if such heuristics or perfect knowledge aren't available in deployment, a sufficiently well-trained agent needs only to greedily follow the policy it has learned from training, allowing pre-training by a simulated environment (with perfect knowledge) to transfer to reality.

If a network is believed to be vulnerable to indirect attacks, such as link-flooding attacks, we may use the following reward:
\begin{equation}
	R_{s,t}^{\mathit{Cross}}(\beta) = \beta R_{s,t} + (1 - \beta) \min{\{R_{s',t} | s' \ne s\}} \label{eqn:lfa-reward}
\end{equation}
where the collaboration parameter $\beta \in [0,1]$ models the expected degree of interference between flows, and $s, s'$ are protected destination nodes in the network.
The key insight underpinning LFAs is that flows can affect a target \emph{without communicating with that target}.
$\beta$ then acts as a tunable parameter which can incentivise agents to remove flows which harm overall system health, by including the performance of the worst-performing destination.
However, such attacks (and the effectiveness of $R_{s,t}^{\mathit{Cross}}$) are not examined by our work.

\subsection{Action Space}
When monitoring a source-destination pair, an agent uses its state vector to decide which proportion of that flow's \emph{inbound} traffic should be dropped.
%?? Why pdrop? Allows for discrete action space, don't have to account for buckets/fairness/burstiness.
This is implemented by installing an action via OpenFlow, instructing its host switch to drop each relevant packet with probability $p$.
We choose to drop packets rather than impose traffic limits as it offers us a discrete action space without prior knowledge of traffic characteristics or measurement.
Furthermore, we need not consider burstiness, fairness or tuning (such as per-flow bucket sizes) which could limit scalability.
We offer two models on how to choose $p$:

\subsubsection{Instant control}
Each agent directly chooses $p \in \left\{ 0.0, 0.1, \ldots, 0.9 \right\}$, giving a discrete, static action set which cannot completely filter traffic.
These choices ensure that the rate reduction imposed on a source IP may never be permanent or irreversible.
Since this model needs no forward planning, we found it best to set the discount factor $\gamma=0$ (making agents purely myopic).

\subsubsection{Guarded control}
The measurements of \textcite{DBLP:conf/ndss/KangGS16} suggest that bot attack flows cannot scale up to match an increase in available bandwidth.
We apply their observations within the RL paradigm by constraining how an agent treats each flow using a simple finite state automaton: we restrict $p \in \left\{ 0.00, 0.05, 0.25, 0.50, 1.0 \right\}$.
The action set is then simply to \emph{maintain}, \emph{increase}, or \emph{decrease} $p$ for a flow in single steps.
We choose these potential values for $p$ to add complete filtering to a steady progression of rate-limiters (\SI{25}{\percent} increments for UDP traffic).
The outlier, $p=0.05$, corresponds to roughly a \SI{50}{\percent} rate reduction for TCP flows in our test topology.
This uneven spread of choices for $p$ allows light and heavy rate reduction to be applied to both congestion-aware and congestion-unaware traffic as required.

To enable temporary bandwidth expansion in all deployments, every flow is initially placed under light packet drop ($p=0.05$); this is chosen above the equivalent for UDP due to TCP's higher prevalence.
Most importantly, an agent must now choose to punish a flow multiple times in succession to cause rapid degradation, reducing variance while allowing an agent to see how a host reacts to structured changes in the environment.

As each agent now requires the capability to plan ahead, we require a discount factor $\gamma \ne 0$, allowing the value of future states to influence state-action value updates.
We found the setting $\gamma = 0.8$ to be the most effective choice for this hyperparameter during exploratory testing.

\subsubsection{Risks}
Our mode of action means that each agent is in control of pushback \cite{DBLP:journals/ccr/MahajanBFIPS02a}, and so carries a risk of introducing collateral damage into the network.
This is particularly severe when handling TCP traffic: the Mathis equation \cite{DBLP:journals/ccr/MathisSMO97} states that TCP bandwidth is proportional to $1/\sqrt{p}$ (noting that $p$ is nonzero in any real network) while constant bitrate (CBR) UDP traffic is proportional to $1 - p$.
%It's worth noting that there are various ways that this could be implemented, and that the application of \emph{programmable data planes} to this end are suggested as future work.
This weakness is still present in modern TCP flavours, such as TCP Cubic which in turn has bandwidth proportional to $1/p^{0.75}$ \cite{rfc8312}.
This is of particular importance due to the prevalence of TCP and other congestion-aware protocols within the Internet.
Our own analysis of CAIDA datasets \cite{caida-2018-passive} shows that congestion-aware traffic makes up at least \SIrange{73}{82}{\percent} of packets, corresponding to \SIrange{77}{84}{\percent} of data volume\footnote{\url{https://github.com/FelixMcFelix/caida-stats}}.
QUIC, a future congestion-aware protocol, comprises \SIrange{2.6}{9.1}{\percent} of traffic observed on backbone links, depending on location and typical workload \cite{DBLP:conf/pam/RuthPDH18}.
%As far as future network protocols are concerned, QUIC \cite{DBLP:conf/sigcomm/LangleyRWVKZYKS17}, a congestion-aware stream transmission protocol, will behave much like TCP, showing the importance of further development to properly handle traffic with such characteristics.

%?? Make more sane. This is about ``Why go per-flow?''
This further justifies our focus on per-flow decisions---real-world deployments see many flows pass over any egress point, making global actions (such as those chosen by \textcite{DBLP:journals/eaai/MalialisK15}) more likely to inflict collateral damage.
%This manifests in two ways: the best-achievable performance drops, and so too does the learning rate.
Given the probability that a host is legitimate, $P_G \in [0,1]$, it follows that a host will be malicious with probability $P_B = 1 - P_G$.
Defining \emph{imperfect service} to mean any case where all $n$ hosts connecting over a switch do not share the same classification (i.e., a mixture), then the probability that a switch is delivering imperfect service is $P_{M,n} = 1 - (P_G^n + P_B^n)$.
\begin{thm}
	As the host/learner ratio $n$ increases, it is more likely that a throttling switch will exhibit imperfect service: $\forall n \in \mathbb{Z}^{+}, P_{M,n} \le P_{M,n+1}$.
\end{thm}
\begin{proof}
	\emph{Base case:} $P_{M,1}=0, P_{M,2} = 1 - P_G^2 - P_B^2 > 0$.
	\emph{Inductive step:} Assume that the theorem holds for $n$. Observe that $P_G^n \ge P_G^{n+1}$ (resp.\ $P_B$). It then follows that:
	\begin{align*}
	P_G^n + P_B^n &\ge P_G^{n+1} + P_B^{n+1}\\
	1 - (P_G^n + P_B^n) &\le 1 - (P_G^{n+1} + P_B^{n+1})\\
	P_{M,n} &\le P_{M,n+1} \qedhere
	\end{align*}
\end{proof}
\begin{corr}
	Restricting $P_G \in (0,1)$ so that both $P_G$ and $P_B$ are non-zero ensures strict inequality: $P_{M,n} < P_{M,n+1}$.
\end{corr}
When considering that many hosts have an especially adverse reaction to our main means of control, flow-level granularity becomes an obvious choice.

\subsection{Systems Considerations}\label{sec:systems-considerations}
Taking many actions per timestep means that any agents are assigned a larger, and potentially unbounded, set of tasks to perform every time they receive load and flow statistics from the network and their parent switch.
This introduces some potential issues: the inability to respond to unexpected changes in flow state, delayed service of new flows, and risks that flow states become outdated.
At their worst, these risks present additional attack surface to an adversary.
To adapt to these problems, we make use of \emph{timed random sequential} updates.

Each agent begins with an empty work list.
For the set of flows active in any timestep, we shuffle the list and perform as many action calculations and updates as possible, within a set time limit.
Uncompleted work is passed on to the next timestep, until the list is emptied, at which point it is repopulated using the set of available measurements.
To ensure that flow control actions are made with recent information, we combine state vectors for unvisited flows in the current work set, and replace the stored vector for all others.
State vector combination is done by summing deltas and packet counts, updating means via weighted sums, and replacing all other fields.
Following \citeauthor{DBLP:conf/sigcomm/ChenL0L18}'s observations concerning short flows \cite{DBLP:conf/sigcomm/ChenL0L18}, we maintain a deadline of \SI{1}{\milli\second}---in tests, an agent is typically able to process around 3 flows in this time.
We expect this should be tuned based on the frequency at which statistics arrive.
Naturally, this implies that an agent must carry work forward (and coalesce state updates) when \emph{host density} is $n>3$ (\cref{sec:evaluation}); this behaviour is not explicitly a property of network size.