
There has been significant research and development on real-time analysis of operational Internet traffic.
Accurate flow characterisation (or \emph{classification}) can drive intrusion detection, detecting unusual or illegal patterns of network traffic, or to prioritize traffic for certain customers, to provide path-diversity as well as to mark Quality of Service (QoS) of various users and protocols~\parencite{DBLP:journals/ccr/BernailleTASS06,DBLP:conf/lisa/Roesch99}.
However, flow classification solutions today can usually only rely on sampled data provided by routers, such as sFlow, Netflow, or IPFIX, along with imprecise timing (\si{\micro\second} and \si{\milli\second}-level)~\parencite{rfc7011,rfc3954}.
While sampled, low-precision telemetry can be used to classify network traffic based on some flow properties (such as port and protocol numbers)~\parencite{DBLP:conf/iwcmc/RossiV10}, it cannot be used to classify based on fine temporal properties (\eg, identifying bursty flows and senders that can cause microbursts and buffer overflow on the network).

On the other hand, full-software solutions for traffic classification have been proposed by commercial vendors (\eg, Barracuda DPI\footnote{https://www.barracuda.com/glossary/deep-packet-inspection}), the open-source community (\eg, Snort~\parencite{DBLP:conf/lisa/Roesch99}, Zeek (formerly Bro)~\parencite{DBLP:conf/uss/Paxson98,zeek}), and the research community, with extensible feature sets and algorithms for classification~\parencite{DBLP:conf/icccn/HagosEYK18}.
Unfortunately, these software solutions designed for commodity hardware do not provide accurate timing of packets, and therefore make certain time-critical events hard or impossible to detect (\eg, microbursts~\parencite{DBLP:conf/sigcomm/ChenFKRR18} or congestion control properties~\parencite{DBLP:conf/icccn/HagosEYK18}).
Moreover, even the most sophisticated software solutions process packets 10 orders of magnitude slower than current backbone traffic of large operators, making them unusable for large-scale operational analysis~\parencite{DBLP:journals/wpc/ParkA17}.

At the same time, programming and fast reconfiguration of network devices is being explored in all types of networks: datacenter and cloud networks, CDNs and WANs.
Specifically, with the recent developments of generalized dataplanes (\eg, the \emph{Portable Switch Architecture}~\parencite{p4-psa}), target devices (\eg, Barefoot Tofino and Netronome SmartNICs) along with the high-level programming languages presented for them (\eg, P4~\parencite{DBLP:journals/ccr/BosshartDGIMRSTVVW14}), operators can now express in-network functionality running on their devices, including accurate nanosecond-precision packet timing.
However, programming in-network services has its own challenges (\eg, restricted instruction sets, data types and memory), prohibiting the implementation of a fully in-network classification solution.

% To solve the aforementioned challenges, this paper presents an architecture that marries the precision timing and fast data aggregation capabilities of the dataplane with software classifiers that can run complex classification models due on a the reduced data rate.

To solve the aforementioned challenges, we present \seidr{}\sidenote{Pronounced ``SAY-ther''.}, a dataplane assisted flow classification solution.
Our design philosophy of \seidr{} keeps functionality where it belongs: dataplane devices create accurately timestamped, aggregated data structures for our analysis, and we let a scalable software stack perform ML-based classification on commodity machines.
As in-network aggregation reduces the data rate by a factor of $\sim$\num{740}, our solution can analyse aggregated data from a total rate of \SI{10}{\tera\bit\per\second} original traffic using a single commodity processing machine.

As a concrete use-case, we look at fine dynamics of TCP congestion control algorithms.
Understanding and classifying them is important for network providers as inadequate choices have severe effects on transfer rates, especially in networks with high bandwidth-delay product~\parencite{DBLP:journals/queue/CardwellCGYJ16} and in networks where multiple congestion control algorithms are used~\parencite{DBLP:conf/imc/WareMSS19}. 
By using accurate congestion control diagnostics, operators will be able to infer sender problems (\eg, backlogged or application-limited senders), network inefficiencies (\eg, increased path latency and congestion), as well as receiver issues (\eg, delayed acknowledgements, small receiver windows) and fairness issues between delay-based and loss-based algorithms~\parencite{DBLP:conf/imc/WareMSS19}.

The contributions of this paper are summarized below:
\begin{itemize}
	\item A flexible dataplane-assisted architecture compatible with the \emph{Portable Switch Architecture} (PSA)~\parencite{p4-psa} that allows data aggregation in the form of histograms with nanosecond-accurate timing (\Cref{sec:architecture}),
	\item A high-accuracy method for telling apart timer-based (\eg, BBR) and cwnd-based TCP flavours using our system with machine learning algorithms (\Cref{sec:tcpcc}),
	\item An extensive evaluation of TCP congestion control classification using our solution (\Cref{sec:evaluation}).
\end{itemize}

% The rest of this paper is organised as follows. Section~\ref{sec:introduction} highlights the motivation of our work, while Section~\ref{sec:architecture} presents our telemetry architecture. Section~\ref{sec:analysis} provides information about the collector and the data analysis the collector performs. Section~\ref{sec:evaluation} given an evaluation, focusing on the TCP congestion control determination. Section~\ref{sec:related} related work provides a summary of related papers, while Section~\ref{sec:conclustion} concludes the paper.