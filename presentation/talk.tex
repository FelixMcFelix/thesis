% !TeX program = lualatex

\documentclass[aspectratio=169,xcolor={dvipsnames}]{beamer}
\usetheme[background=light, numbering=fraction]{metropolis}

\usepackage{appendixnumberbeamer}
\usepackage{pgfpages}
\usepackage{FiraMono}

\setmonofont[
Scale=MatchLowercase,
Contextuals={Alternate}
]{Fira Mono}

\usepackage[newfloat]{minted}
% \usemintedstyle{autumn}
\usemintedstyle{tango}

% You will need to modify these, the authors down below,
% and likely the \addbibresource statements below.
\newcommand{\mytitle}{Online Learning on the Programmable Dataplane}
\newcommand{\myemail}{k.simpson.1@research.gla.ac.uk}
\newcommand{\myurl}{https://mcfelix.me}
\newcommand{\mygithub}{FelixMcFelix}

%\usepackage[T1]{fontenc}

\usepackage{bm}
\usepackage{mathtools}

\usepackage[labelfont=bf,textfont={it}]{caption}
\usepackage{subcaption}
\captionsetup[figure]{justification=centering}
\captionsetup[subfigure]{justification=centering}

\usepackage{tikz}
\usepackage{varwidth}
\usetikzlibrary{arrows.meta, calc, fit, positioning, shapes}

\usepackage[title]{appendix}

\usepackage{etoolbox}
\usepackage[per-mode=symbol]{siunitx}
\robustify\bfseries
\robustify\emph
%\robustify\uline
\sisetup{detect-all, range-phrase=--, range-units=single, detect-weight=true, table-format=1.3}
\DeclareSIUnit{\packet}{p}

\usepackage[siunitx]{circuitikz}

%\usepackage{fontspec}
%\setsansfont{Fira Sans Mono}

\usepackage[UKenglish]{babel}
\usepackage{csquotes}

\usepackage{amssymb}

\usepackage{lipsum}
%\usepackage[basic]{complexity}
\usepackage[super,negative]{nth}

\usepackage{booktabs}

%bib
\usepackage[maxnames=3,maxbibnames=99,mincrossrefs=5,sortcites
%,backend=bibtex
,style=authortitle
]{biblatex}
\addbibresource{../bibliography/papers-off.bib}
\addbibresource{../bibliography/confs-off.bib}

% official colours
\definecolor{uofguniversityblue}{rgb}{0, 0.219608, 0.396078}

\definecolor{uofgheather}{rgb}{0.356863, 0.32549, 0.490196}
\definecolor{uofgaquamarine}{rgb}{0.603922, 0.72549, 0.678431}
\definecolor{uofgslate}{rgb}{0.309804, 0.34902, 0.380392}
\definecolor{uofgrose}{rgb}{0.823529, 0.470588, 0.709804}
\definecolor{uofgmocha}{rgb}{0.709804, 0.564706, 0.47451}

\definecolor{uofglawn}{rgb}{0.517647, 0.741176, 0}
\definecolor{uofgcobalt}{rgb}{0, 0.615686, 0.92549}
\definecolor{uofgturquoise}{rgb}{0, 0.709804, 0.819608}
\definecolor{uofgsunshine}{rgb}{1.0, 0.862745, 0.211765}
\definecolor{uofgpumpkin}{rgb}{1.0, 0.72549, 0.282353}
\definecolor{uofgthistle}{rgb}{0.584314, 0.070588, 0.447059}
\definecolor{uofgpillarbox}{rgb}{0.701961, 0.047059, 0}
\definecolor{uofglavendar}{rgb}{0.356863, 0.301961, 0.580392}

\definecolor{uofgsandstone}{rgb}{0.321569, 0.278431, 0.231373}
\definecolor{uofgforest}{rgb}{0, 0.317647, 0.2}
\definecolor{uofgburgundy}{rgb}{0.490196, 0.133333, 0.223529}
\definecolor{uofgrust}{rgb}{0.603922, 0.227451, 0.023529}

\definecolor{inferno0}{rgb}{0.001462 0.000466 0.013866}
\definecolor{inferno64}{rgb}{0.341500 0.062325 0.429425}
\definecolor{inferno128}{rgb}{0.735683 0.215906 0.330245}
\definecolor{inferno192}{rgb}{0.978422 0.557937 0.034931}
\definecolor{inferno255}{rgb}{0.988362 0.998364 0.644924}

%picky abt et al.
\usepackage{xpatch}

\makeatletter\let\expandableinput\@@input\makeatother

\xpatchbibmacro{name:andothers}{%
	\bibstring{andothers}%
}{%
	\bibstring[\emph]{andothers}%
}{}{}

%opening!

\usepackage{cleveref}
\newcommand{\crefrangeconjunction}{--}

\usepackage{fontawesome5}

\addtobeamertemplate{footnote}{\vspace{-6pt}\advance\hsize-0.5cm}{\vspace{6pt}}
\makeatletter
% Alternative A: footnote rule
\renewcommand*{\footnoterule}{\kern -3pt \hrule \@width 2in \kern 8.6pt}
% Alternative B: no footnote rule
% \renewcommand*{\footnoterule}{\kern 6pt}
\makeatother

\usepackage[export]{adjustbox}
\usetikzlibrary{arrows.meta, calc, fit, positioning, shapes.misc}

% \expandableinput is useful for some tables etc.
\makeatletter\let\expandableinput\@@input\makeatother
\newcommand{\cmark}{\ding{51}}%
\newcommand{\xmark}{\ding{55}}%

\usepackage{array}

\newcolumntype{C}[1]{>{\centering\let\newline\\\arraybackslash\hspace{0pt}}m{#1}}

%-------------------------------------%
%-------------------------------------%

\title{\mytitle}
\author{\vspace{-1em}\textbf{Kyle A.\ Simpson}\\
	\faEnvelopeOpen{} \href{mailto:\myemail}{\nolinkurl{\myemail}}\\
	\vspace{1em}\small{\faGithub{} \href{https://github.com/\mygithub}{\mygithub} \hspace{0.5em} \faGlobe{} \url{\myurl}}}
\institute{University of Glasgow}
\date{September 16, 2022}

\begin{document}
	\begin{frame}
		\maketitle
		\begin{tikzpicture}[overlay, remember picture]
			%		\node[above right=0.8cm and 0.9cm of current page.south west] (esnet-logo) {\includegraphics[width=2.75cm]{netlab-trim}};
			%		\node[right=1cm of esnet-logo] {\adjincludegraphics[height=2cm,trim={0 {.4\height} 0 {.05\height}},clip]{uofg}};
			\node[above right=-0.1cm and 0.8cm of current page.south west] (uofg-logo) {\adjincludegraphics[height=2cm,trim={0 {.4\height} 0 {.05\height}},clip]{branding/uofg}};
			\node[right=0.5cm of uofg-logo] {\includegraphics[width=2.75cm]{branding/netlab-trim}};
		\end{tikzpicture}
	\end{frame}
%	\section{First Section}
	\begin{frame}{Main Thesis -- Summarised}
		\strong{The sum of data-driven networking (DDN) and programmable dataplane (PDP) developments are key tools in improving management of computer networks.}
		% 
		
		\begin{itemize}
			\item DDN automates, \alert{increases efficacy} in control/optimisation problems.
			\item PDP hardware allows \alert{more precise input data} for DDN, faster.
			\item ML/RL in PDP hardware lets us \alert{learn from \& act on network events with less delay}.
			\item PDP useful to reduce data rates -- \alert{make host classification work at scale}.
		\end{itemize}
	\end{frame}

	\begin{frame}{Motivation (I)}
		Many control/design mechanisms are already heuristic -- CCAs, routing, network defence, content delivery/placement...
		\begin{itemize}
			\item DDN can offer improved mechanisms, or tailored control over tuning parameters.
			\item Catch: need these to run \alert{fast}!
		\end{itemize}
		
		Many problems already feedback-loop like:
		\begin{itemize}
			\item Constant measure-act cycles.
			\item Clear optimisation criteria. % no need to manually classify
		\end{itemize}
	
		\strong{Problem}: costs of ML inference and learning are high.
		
		\strong{Problem}: traditional network data are low-quality, sampled state.

	\end{frame}

	\begin{frame}{Motivation (II)}
		Advances in PDP and In-Network Compute help here:
		\begin{itemize}
			\item Architectures designed for line-rate processing.
			\item More fine-grained state -- timestamps, flow monitoring.
			\item Per-packet flow state \emph{must} be handled in-situ.
		\end{itemize}
	
		Host-based execution has steering, stack costs that add to latency and harm throughput.
		
%		\begin{itemize}
%			\item Need to choose exec environment, awareness of all accelerators and dataplane tooling.
%		\end{itemize}
%	
%		?? Precise timestamping, 
%		
%		?? fine-grained measurements?
%		
%		?? need awareness of all accelerators and dataplane tooling
%		
%		?? The intersection relies on exploiting the best features of both fields? PDP and DDN
%		
%		?? Need to choose exec environment. -- cuts into the stack, algo considerations, ... Cost of ML inference + learning can be high
	\end{frame}

	\begin{frame}{Supporting the thesis}
		How is this explored? \strong{Three strands of work:}
		
		\begin{table}
			\centering\begin{tabular}{@{}C{4cm}C{4cm}C{4cm}@{}}
			\toprule Pure DDN & ML \strong{\emph{in}} PDP & PDP \strong{\emph{for}} ML \\
			\midrule \alert{Ch.\ 4 \& 6} & \alert{Ch.\ 5} & \alert{Ch.\ 6} \\
			RL for anti-DDoS,\newline CCA detection & Online RL in the dataplane & In-dataplane telemetry reduction \\ % to suit large scale/BW networks
			\strong{More effective (env. dependent)} & \strong{Strong online latency + t'put benefits} & \strong{Significant packet rate reduction} \\
			\emph{Need to really understand environment, protocols} & \emph{Limited devices, sacrificed model capacity} & \emph{Data reduction needs tailored to problem} \\
			\bottomrule
		\end{tabular}
			\end{table}
	
		Other use cases and methods supported by recent literature -- \alert{Ch.\ 2--3}.
	
%		Networks for ML covered in lit.
	\end{frame}
	\begin{frame}{Wider takeaways and conclusions}
		\begin{itemize}
			\item DDN can be very effective \emph{when it works}:
			\begin{itemize}
				\item Hard to deploy and design.
				\item \alert{Deep knowledge of environments needed.}
			\end{itemize}
			\item Co-design is key!
			\begin{itemize}
				\item Expertise on system dynamics -- \strong{network and HW/SW architectures}.
				\item PDPs excellent for general offload...
				\item ...but ML offload still needs deep human insight.
				\item Future online works will need more insight at HW level \emph{\'a la} Taurus.
			\end{itemize}
			\item Security and explainability look bleak:
			\begin{itemize}
				\item Extraordinary care needed around input data, training experience, and scope of outputs.
			\end{itemize}
		\end{itemize}
	\end{frame}
\end{document}